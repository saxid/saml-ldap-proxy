\section{Übersicht}
\subsection{Infrastruktur}
Die virtuelle Test-Infrastruktur besteht initial aus zwei Servern:
\begin{itemize}
  \item 1x Identity Provider mit Shibboleth IdP 2.4.4
  \item 1x Service Provider mit Shibboleth SP Modul 2.5.2 für Apache 
\end{itemize} 
Als Virtualisierungsumgebung kommt die OpenSource Software
VirtualBox\footnote{https://www.virtualbox.org/} zum Einsatz, andere
Virtualisierungsprodukte sollten ebenfalls funktionieren.

\begin{figure}[h!]
  \centering
    \includegraphics[width=1.0\textwidth]{img/infrastruktur.png}
    \caption{VM Infrastruktur}
\end{figure}

Die Maschinen der Testumgebung kommunizieren über ein host-internes Netzwerk.
Zusätzlich wird jede Maschine per NAT über den Host mit dem Internet verbunden
um Pakete nachinstallieren zu können. Als Datenquelle für die
Shibboleth-Nutzerdaten dient auf IdP-Seite eine OpenLDAP-Server. Die Nutzerdaten
können entweder über das Webinterface phpldapadmin oder (besser) per Apache
Directory Studio (zB. auf dem VM Hostrechner) administriert werden. Der Zugriff
auf beide Maschinen kann entweder über das GUI erfolgen oder per SSH-Client. Für
die Namensauflösung innerhalb des VM-Netzes wird auf dem IdP ein DNS-Server
(bind9) konfiguriert, die IP-Vergabe erfolgt über einen DHCP-Server (ebenfalls
auf dem IdP). Um die für eine gesicherte Verbindung nötigen (selbst signierten)
Zertifikate erzeugen zu können, wird eine Certificate Authority auf dem
IdP-Server bereitgestellt. Dieser CA wird innerhalb der Testumgebung vertraut,
sodass eine durchgängige Zertifikatskette erzeugt werden kann.

\subsection{Software}
Auf beiden Maschinen kommt Ubuntu 14.04 LTS (Desktop) x64 zum Einsatz. Die
Desktop Version wurde für die Testumgebung gewählt, um für die Entwicklung von
Zusatzmodulen und die Konfiguration ein GUI "`out-of-the-box"' zur Verfügung zu
haben. Alternativ kann sicherlich auch die Ubuntu Server-Variante eingesetzt
werden. Die auszuführenden Kommandos sollten nahezu identisch sein. Es ist
jedoch darauf zu achten, dass ggf. weitere Pakete nachinstalliert werden müssen.
Für die Installation und Konfiguration unter anderen Linux-Distributionen können
die hier gezeigten Schritte als grobe Orientierungshilfe dienen.
\newline Neben der reinen Identity-Provider-Funktionalität soll die IdP-VM noch
weitere Aufgaben im Netz übernehmen:
\begin{itemize}
  \item DNS-Server
  \item DHCP-Server
  \item Bereitstellung von Nutzerdaten per LDAP
  \item Certificate Authority (CA)
\end{itemize}
Die Service-Provider-VM wird hier so weit Konfiguriert, dass eine per Shibboleth
gesicherte Webseite betrieben werden kann. Der SP kann dann für weitere
Testzwecke wie das Anbinden von Services an Shibboleth genutzt werden.
\newline Zur Basis-Installation auf beiden Maschinen zählen folgende
zusätzliche Pakete:
\begin{itemize}
  \item Apache 2.4
  \item MySQL 5.5
  \item PHP 5.5 (inkl. PHP-MySQL)
  \item Sublime Text 3 (aus PPA)
  \item vim
  \item openssh-Server
  \item Chromium-Browser
  \item NTP
  \item Gasterweiterungen für VirtualBox  
\end{itemize}
\subsection{Hinweise}
\begin{itemize}
  \item zum Editieren der Dateien bieten sich entweder SublimeText, gedit oder
  vim an.
  SublimeText ist schon installiert, aus dem Terminal einfach mit subl
  <filename> starten, vim und gedit natürlich analog
  \item Änderungen an Dateien sind in einer Notation mit Zeilennummer
  (Orginaldokument) und einem "`--"' für zu entfernende und "`++"' für
  hinzuzufügende Zeilen dokumentiert.
  \item als Basis für die Installation wurden die Anleitungen vom DFN sowie von
  SWITCH genutzt. Diese konnten aber oftmals nicht 1:1 angewendet werden, daher
  bitte vorrangig an dieses Dokument halten.
  \begin{itemize}
    \item https://www.aai.dfn.de/dokumentation/identity-provider/konfiguration/
    \item https://www.aai.dfn.de/dokumentation/service-provider/
    \item
    https://www.aai.dfn.de/fileadmin/documents/attributes/200811/Object\_Identifier\_DFN-AAI.pdf
    (ACHTUNG: Schreibfehler bei den Attributnamen!)
    \item https://www.switch.ch/aai/support/tools/uapprove/
  \end{itemize}
  \item vollständige Listings der umfangreicheren Konfigurationsdateien sind
  auch nochmal am Ende des Dokumentes angefügt.
\end{itemize}
