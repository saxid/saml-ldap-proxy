\documentclass[12pt, a4paper]{scrartcl}

\usepackage[ngerman]{babel}
\usepackage[T1]{fontenc}
\usepackage[utf8]{inputenc}
\usepackage{lmodern}
%\usepackage{minted}
\usepackage{graphicx}
\usepackage{multirow}
\usepackage{listings}
\usepackage{color}
\usepackage[usenames,dvipsnames]{xcolor}
\usepackage{nameref}
\usepackage{courier}

\definecolor{mygreen}{rgb}{0,0.6,0}
\definecolor{mygray}{rgb}{0.5,0.5,0.5}
\definecolor{mymauve}{rgb}{0.58,0,0.82}

\lstset{ %
  backgroundcolor=\color{white},   % choose the background color; you must add \usepackage{color} or \usepackage{xcolor}
  basicstyle=\footnotesize\ttfamily,        % the size of the fonts that are
  % used for the code
  breakatwhitespace=false,         % sets if automatic breaks should only happen at whitespace
  breaklines=true,                 % sets automatic line breaking
  captionpos=b,                    % sets the caption-position to bottom
  commentstyle=\color{mygreen},    % comment style
  deletekeywords={bind,type,local},            % if you want to delete keywords
  % from the given language
  escapeinside={\%*}{*)},          % if you want to add LaTeX within your code
  extendedchars=true,              % lets you use non-ASCII characters; for 8-bits encodings only, does not work with UTF-8
  frame=single,                    % adds a frame around the code
  keepspaces=true,                 % keeps spaces in text, useful for keeping indentation of code (possibly needs columns=flexible)
  keywordstyle=\color{blue},       % keyword style
  language=bash,                 % the language of the code
  alsoletter={-, +},
  keywords=[2]{sudo, apt-get, service, shib-idp@shibidp, dig, start, stop,
  a2ensite, a2enmod, restart, cp, mv, openssl, rm, ls, ll, la, grep, mkdir,
  chown, dpkg-reconfigure, ln, --, ++, ldapadd, ldapmodify, ifup, ifdown, scp,
   add-apt-repository, wget, unzip},
  % if you want to add more keywords to the set
  numbers=none,                    % where to put the line-numbers; possible
  % values are (none, left, right)
  numbersep=5pt,                   % how far the line-numbers are from the code
  numberstyle=\tiny\color{mygray}, % the style that is used for the line-numbers
  rulecolor=\color{black},         % if not set, the frame-color may be changed on line-breaks within not-black text (e.g. comments (green here))
  showspaces=false,                % show spaces everywhere adding particular underscores; it overrides 'showstringspaces'
  showstringspaces=false,          % underline spaces within strings only
  showtabs=false,                  % show tabs within strings adding particular underscores
  stepnumber=2,                    % the step between two line-numbers. If it's 1, each line will be numbered
  stringstyle=\color{mymauve},     % string literal style
  tabsize=2,                       % sets default tabsize to 2 spaces
  title=\lstname,                   % show the filename of files included with \lstinputlisting; also try caption instead of title
  literate=%
  {Ö}{{\"O}}1
  {Ä}{{\"A}}1
  {Ü}{{\"U}}1
  {ß}{{\ss}}2
  {ü}{{\"u}}1
  {ä}{{\"a}}1
  {ö}{{\"o}}1,
  emph=[1]{%  
   sudo%
    },emphstyle=[1]{\color{red}\bfseries},
  emph=[2] {
    shib-idp@shibidp,shib-sp@shibsp 
    },emphstyle=[2]{\bfseries},
  emph=[3] {
  	start, restart, ++
  },emphstyle=[3]{\color{green}},
  emph=[4] {
   stop, --
  },emphstyle=[4]{\color{red}}
}


\newcommand{\step}[1]{\begin{itemize} \item \textbf{#1} \end{itemize}}
\newcommand{\anno}[1]{\begin{itemize} \item \textit{#1} \end{itemize}}

\begin{document}

\title{Installation und Konfiguration einer Testumgebung für Shibboleth IdP und
SP}
\subtitle{Version 0.8}
\author{Steffen Dach\\ Rechenzentrum \\ TU Bergakademie Freiberg}
\maketitle

\newpage
\tableofcontents

\newpage
\section{Übersicht}
\subsection{Infrastruktur}
Die virtuelle Test-Infrastruktur besteht initial aus zwei Servern:
\begin{itemize}
  \item 1x Identity Provider mit Shibboleth IdP 2.4.4
  \item 1x Service Provider mit Shibboleth SP Modul 2.5.2 für Apache 
\end{itemize} 
Als Virtualisierungsumgebung kommt die OpenSource Software
VirtualBox\footnote{https://www.virtualbox.org/} zum Einsatz, andere
Virtualisierungsprodukte sollten ebenfalls funktionieren.

\begin{figure}[h!]
  \centering
    \includegraphics[width=1.0\textwidth]{img/infrastruktur.png}
    \caption{VM Infrastruktur}
\end{figure}

Die Maschinen der Testumgebung kommunizieren über ein host-internes Netzwerk.
Zusätzlich wird jede Maschine per NAT über den Host mit dem Internet verbunden
um Pakete nachinstallieren zu können. Als Datenquelle für die
Shibboleth-Nutzerdaten dient auf IdP-Seite eine OpenLDAP-Server. Die Nutzerdaten
können entweder über das Webinterface phpldapadmin oder (besser) per Apache
Directory Studio (zB. auf dem VM Hostrechner) administriert werden. Der Zugriff
auf beide Maschinen kann entweder über das GUI erfolgen oder per SSH-Client. Für
die Namensauflösung innerhalb des VM-Netzes wird auf dem IdP ein DNS-Server
(bind9) konfiguriert, die IP-Vergabe erfolgt über einen DHCP-Server (ebenfalls
auf dem IdP). Um die für eine gesicherte Verbindung nötigen (selbst signierten)
Zertifikate erzeugen zu können, wird eine Certificate Authority auf dem
IdP-Server bereitgestellt. Dieser CA wird innerhalb der Testumgebung vertraut,
sodass eine durchgängige Zertifikatskette erzeugt werden kann.

\subsection{Software}
Auf beiden Maschinen kommt Ubuntu 14.04 LTS (Desktop) x64 zum Einsatz. Die
Desktop Version wurde für die Testumgebung gewählt, um für die Entwicklung von
Zusatzmodulen und die Konfiguration ein GUI "`out-of-the-box"' zur Verfügung zu
haben. Alternativ kann sicherlich auch die Ubuntu Server-Variante eingesetzt
werden. Die auszuführenden Kommandos sollten nahezu identisch sein. Es ist
jedoch darauf zu achten, dass ggf. weitere Pakete nachinstalliert werden müssen.
Für die Installation und Konfiguration unter anderen Linux-Distributionen können
die hier gezeigten Schritte als grobe Orientierungshilfe dienen.
\newline Neben der reinen Identity-Provider-Funktionalität soll die IdP-VM noch
weitere Aufgaben im Netz übernehmen:
\begin{itemize}
  \item DNS-Server
  \item DHCP-Server
  \item Bereitstellung von Nutzerdaten per LDAP
  \item Certificate Authority (CA)
\end{itemize}
Die Service-Provider-VM wird hier so weit Konfiguriert, dass eine per Shibboleth
gesicherte Webseite betrieben werden kann. Der SP kann dann für weitere
Testzwecke wie das Anbinden von Services an Shibboleth genutzt werden.
\newline Zur Basis-Installation auf beiden Maschinen zählen folgende
zusätzliche Pakete:
\begin{itemize}
  \item Apache 2.4
  \item MySQL 5.5
  \item PHP 5.5 (inkl. PHP-MySQL)
  \item Sublime Text 3 (aus PPA)
  \item vim
  \item openssh-Server
  \item Chromium-Browser
  \item NTP
  \item Gasterweiterungen für VirtualBox  
\end{itemize}
\subsection{Hinweise}
\begin{itemize}
  \item zum Editieren der Dateien bieten sich entweder SublimeText, gedit oder
  vim an.
  SublimeText ist schon installiert, aus dem Terminal einfach mit subl
  <filename> starten, vim und gedit natürlich analog
  \item Änderungen an Dateien sind in einer Notation mit Zeilennummer
  (Orginaldokument) und einem "`--"' für zu entfernende und "`++"' für
  hinzuzufügende Zeilen dokumentiert.
  \item als Basis für die Installation wurden die Anleitungen vom DFN sowie von
  SWITCH genutzt. Diese konnten aber oftmals nicht 1:1 angewendet werden, daher
  bitte vorrangig an dieses Dokument halten.
  \begin{itemize}
    \item https://www.aai.dfn.de/dokumentation/identity-provider/konfiguration/
    \item https://www.aai.dfn.de/dokumentation/service-provider/
    \item
    https://www.aai.dfn.de/fileadmin/documents/attributes/200811/Object\_Identifier\_DFN-AAI.pdf
    (ACHTUNG: Schreibfehler bei den Attributnamen!)
    \item https://www.switch.ch/aai/support/tools/uapprove/
  \end{itemize}
  \item vollständige Listings der umfangreicheren Konfigurationsdateien sind
  auch nochmal am Ende des Dokumentes angefügt.
\end{itemize}
 
\section{Spickzettel}\label{sec:spickzettel}
\subsection{Nutzer und Passwörter}

\begin{center}
	\begin{tabular}{| l | l | p{4cm} | l | }
		\hline
		Host & Service / Objekt & Nutzername & Passwort\\ \hline \hline
		\multirow{11}{*}{shibidp} 
		 & Benutzer & shib-idp & idp \\ \cline{2-4}
		 & sudo & - & idp \\ \cline{2-4}
		 & MySQL & root & idp \\ \cline{2-4}
		 & MySQL & uApprove & idp \\ \cline{2-4}
		 & LDAP & cn=admin, dc=shib,dc=lan & idp \\ \cline{2-4}
		 & LDAP & uid=shib\_ldap, dc=shib,dc=lan & shib\_ldap \\ \cline{2-4}
		 & CA-Stammzertifikat & - & +\$hib-idp.C@\_p@\$\$phr@5e** \\ \cline{2-4}
		 & ldap.shib.lan-Zertifikat & - & +\$hib-idp.LD@P\_p@\$\$** \\ \cline{2-4}
		 & www.shib.lan-Zertifikat & - & +\$hib-idp.www\_p@\$\$** \\ \cline{2-4}
		 & idp.shib.lan-Zertifikat & - & +\$hib-idp.idp\_p@\$\$** \\ \cline{2-4}
		 & IdP-Keystore & - & *\$hib-idp.k3yst0r3\_p@\$\$** \\ \hline
		\hline
		\multirow{4}{*}{shibsp} 
		 & Benutzer & shib-sp & sp \\ \cline{2-4}
		 & sudo & - & sp \\ \cline{2-4}
		 & MySQL & root & sp \\ \cline{2-4}
		 & sp.shib.lan-Zertifikat & - & +\$hib-idp.sp\_p@\$\$** \\ \hline
	\end{tabular}
\end{center}

\subsection{Hosts}
\begin{center}
	\begin{tabular}{| l | l | l | p{7cm} |}
		\hline
		Host & DNS & (hier benötigte) erreichbare Dienste \\ \hline\hline
		 shibidp & shib.lan & bei HTTP(S): Weiterleitung auf
		https://www.shib.lan \\ \hline
		shibidp & idp.shib.lan & DNS, DHCP, www
		(allg.), MySQL, Shib-IdP (Tomcat), SSH
		\\
		\hline 
		shibidp & ldap.shib.lan & LDAP, phpldapadmin (https,
		vHost)
		\\
		\hline  shibsp & sp.shib.lan & www (allg.) inkl. Shib-SP (https,
		vHost)
		\\
		\hline
		
	\end{tabular}
\end{center}
\section{Installation von Identity Provider und Service Provider}
\subsection{VirtualBox}
Da der IdP die IPs per DHCP im VM-Netz übernehmen soll, muss der
VirtualBox-interne DHCP Server für das Host-Only-Netzwerk deaktiviert werden.
Laufen bereits VMs auf dem Host-System, die den VirtualBox-DHCP für das interne
Netz nutzen, sollte ein neues Host-Only-Netzwerk angelegt werden:\newline
Datei $\rightarrow$ Globale EInstellungen $\rightarrow$ Netzwerk $\rightarrow$
Reiter "`Host-Only-Netzwerke"' $\rightarrow$ Host-Only-Netzwerk hinzufügen
(kleine Buttons rechts) bzw. für den vorhandenen den DHCP-Server deaktivieren
(über "`Host-Only-Netzwerk ändern"'-Button) $\rightarrow$ IP-Adresse des
Adapters: 192.168.100.1/24
\begin{figure}[h!]
  \centering
    \includegraphics[width=0.75\textwidth]{img/vbox-network-global.png}
    \caption{Host-Only-Netzwerk - globale Einstellungen VirtualBox}
\end{figure}

Danach ist natürlich darauf zu achten, dass die beiden Shibboleth-VMs als
zweiten Netzwerkadapter (neben dem NAT-Adapter) noch einen "`Host-only Adapter"'
erhalten. Dieser muss als virtuellen Host-Adapter natürlich den gerade in den
globalen Einstellungen konfigurierten Host-Only-Adapter nutzen.
\begin{figure}[h!]
  \centering
    \includegraphics[width=0.75\textwidth]{img/vbox-network-vm.png}
    \caption{Host-Only-Netzwerk in den VMs}
\end{figure}

\newpage
\subsection{Grundkonfiguration auf beiden Systemen}
\textcolor{red}{\textbf{Kennwörter für sudo (siehe auch Kapitel
\ref{sec:spickzettel} \nameref{sec:spickzettel}):
\newline IDP: idp\newline SP: sp}}

PPA für SublimeText3 hinzufügen aktuelle Patches einspielen (hier für shibidp):
\begin{lstlisting}
shib-idp@shibidp:~$ sudo add-apt-repository ppa:webupd8team/sublime-text-3
shib-idp@shibidp:~$ sudo apt-get update
shib-idp@shibidp:~$ sudo apt-get upgrade
\end{lstlisting}

Pakete installieren:
\begin{lstlisting}
shib-idp@shibidp:~$ sudo apt-get install chromium-browser openssh-server apache2 php5 php5-mysql mysql-server vim ntp sublime-text-installer
\end{lstlisting}
Für MySQL wird das MySQL-root-Passwort abgefragt, hier wurde "`idp"' gewählt.
\newline Optional kann noch der SSH-Server angepasst werden, die Config findet
sich unter \mbox{/etc/ssh/sshd\_config}. Die Grundkonfiguration ist aber
ausreichend.
\subsection{Identity Provider - Teil 1}
Im ersten Teil des IdP-Setups werden DNS, DHCP, LDAP, CA und der Shibboleth IdP
installiert und konfiguriert. Um die Basiskonfiguration zu testen wird
anschließend der SP aufgesetzt. Im zweiten Teil des IdP-Setups erfolgt dann der
Feinschliff und die Konfiguration des Attribute-Resolvers und -Filters.
\subsubsection{DNS}
\step{/etc/network/interfaces anpassen (KEIN Gateway für eth1 setzen, alle was
  nicht nach 192.168.100.0 geht wird über NAT geroutet. Das Gateway wird per
  DHCP vom VM-Wirtsrechner verteilt!):}
\begin{lstlisting}
shib-idp@shibidp:~$ cat /etc/network/interfaces
	# interfaces(5) file used by ifup(8) and ifdown(8)
	auto lo
	iface lo inet loopback
		auto eth0
	iface eth0 inet dhcp
		auto eth1
	iface eth1 inet static
		address 192.168.100.100
		netmask 255.255.255.0
		dns-nameservers 192.168.100.100
		dns-search shib.lan
\end{lstlisting}
\step{bind9 installieren und konfigurieren}
\begin{lstlisting}
shib-idp@shibidp:~$ sudo apt-get install bind9
shib-idp@shibidp:~$ sudo service bind9 stop
\end{lstlisting}
\step{da wir nur IPv4 nutzen kann eibe "`-4"' in der bind-config in die Optionen
übernommen werden:}
\begin{lstlisting}
shib-idp@shibidp:~$ cat /etc/default/bind9
	# run resolvconf?
	RESOLVCONF=no

	# startup options for the server
	OPTIONS="-u bind -4"
\end{lstlisting}
\step{Zonendateien für Forward-Lookup erstellen}
\begin{lstlisting}
shib-idp@shibidp:~$ cat /etc/bind/db.shib.lan
	;; db.shib.lan
	;; Forwardlookupzone für domainname
	;;
	$TTL 2D
	@       IN      SOA     shib.lan. mail.shib.lan. (
							2015051801      ; Serial
									8H      ; Refresh
									2H      ; Retry
									4W      ; Expire
									3H )    ; NX (TTL Negativ Cache)
	@                               IN      NS      shib.lan.
									IN      MX      10 mailserver.shib.lan.
									IN      A       192.168.100.100
	idp                             IN      A       192.168.100.100
	localhost                       IN      A       127.0.0.1
	sp                              IN      A       192.168.100.200
	www                             IN      A       192.168.100.100

\end{lstlisting}
\step{Zonendateien für Reverse-Lookup erstellen}
\begin{lstlisting}
shib-idp@shibidp:~$ cat /etc/bind/db.100.168.192
	;; db.100.168.192
	;; Reverselookupzone für shib.lan
	;;
	$TTL 2D
	@       IN      SOA     shib.lan. mail.shib-idp.lan. (
									2015051801      ; Serial
											8H      ; Refresh
											2H      ; Retry
											4W      ; Expire
											2D )    ; TTL Negative Cache

	@       IN      NS      shib.lan.

	100     IN      PTR     shib.lan.
	200     IN      PTR     sp.shib.lan.
\end{lstlisting}

\step{neuen DNS "`bekannt machen"':}
\begin{lstlisting}
shib-idp@shibidp:~$ cat /etc/bind/named.conf.local
	//
	// Do any local configuration here
	//

	// Consider adding the 1918 zones here, if they are not used in your
	// organization
	//include "/etc/bind/zones.rfc1918";

	zone "shib.lan" {
	type master;
	file "/etc/bind/db.shib.lan";
	};

	zone "100.168.192.in-addr.arpa" {
	type master;
	file "/etc/bind/db.100.168.192";
\end{lstlisting}
\step{lokalen Nameserver als Default festlegen (immer der erste, der abgefragt
wird, head wird immer zuerst eingebunden in die resolv.conf):}
\begin{lstlisting}
shib-idp@shibidp:~$ cat /etc/resolvconf/resolv.conf.d/head
	# Dynamic resolv.conf(5) file for glibc resolver(3) generated by resolvconf(8)
	#     DO NOT EDIT THIS FILE BY HAND -- YOUR CHANGES WILL BE OVERWRITTEN
	nameserver 192.168.100.100
\end{lstlisting}
\step{DNS-Weiterleitung aktivieren (forwarders { ... } einkommentieren, 8.8.8.8
(google) ggf. durch lokalen DNS (der Uni) ersetzen):}
\begin{lstlisting}
shib-idp@shibidp:~$ cat /etc/bind/named.conf.options
	options {
		directory "/var/cache/bind";

		// If there is a firewall between you and nameservers you want
		// to talk to, you may need to fix the firewall to allow multiple
		// ports to talk.  See http://www.kb.cert.org/vuls/id/800113

		// If your ISP provided one or more IP addresses for stable
		// nameservers, you probably want to use them as forwarders.
		// Uncomment the following block, and insert the addresses replacing
		// the all-0's placeholder.

		forwarders {
			8.8.8.8;
		};

		//========================================================================
		// If BIND logs error messages about the root key being expired,
		// you will need to update your keys.  See https://www.isc.org/bind-keys
		//========================================================================
		dnssec-validation auto;

		auth-nxdomain no;    # conform to RFC1035
		listen-on-v6 { any; };
	};
\end{lstlisting}
\step{entweder jetzt einfach das System neu durchstarten (sudo reboot now)
oder:}
\begin{lstlisting}
shib-idp@shibidp:~$ sudo ifdown eth0
shib-idp@shibidp:~$ sudo ifdown eth1
shib-idp@shibidp:~$ sudo ifup eth0
shib-idp@shibidp:~$ sudo ifup eth1
shib-idp@shibidp:~$ sudo service bind9 restart
shib-idp@shibidp:~$ sudo resolvconf -u
\end{lstlisting}
\step{Config checken:}
Routing sollte so aussehen (10er Netz ist das NAT vom Wirt, kann ggf. anders sein):
\begin{lstlisting}
shib-idp@shibidp:~$ route -n
	Kernel-IP-Routentabelle
	Ziel            Router          Genmask         Flags Metric Ref    Use Iface
	0.0.0.0         10.0.2.2        0.0.0.0         UG    0      0        0 eth0
	10.0.2.0        0.0.0.0         255.255.255.0   U     0      0        0 eth0
	169.254.0.0     0.0.0.0         255.255.0.0     U     1000   0        0 eth1
	192.168.100.0    0.0.0.0         255.255.255.0   U     0      0        0 eth1
\end{lstlisting}
Nameserver-Config (der zweite nameserver 192... Eintrag kommt aus der interfaces-Datei, kann dort ggf. auch weggelassen werden):
\begin{lstlisting}
shib-idp@shibidp:~$ cat /etc/resolv.conf
	# Dynamic resolv.conf(5) file for glibc resolver(3) generated by resolvconf(8)
	#     DO NOT EDIT THIS FILE BY HAND -- YOUR CHANGES WILL BE OVERWRITTEN
	nameserver 192.168.100.100
	nameserver 139.20.64.1
	nameserver 139.20.64.2
	nameserver 192.168.100.100
	search shib.lan
\end{lstlisting}
DNS-Auflösung mit dig testen (Ergebnisse sollten zumindest ähnlich sein, was auf jeden Fall funktionieren muss ist die Auflösung der Domain in eine IP):
 Test nach "`außen"' über NAT:
\begin{lstlisting}
shib-idp@shibidp:~$ dig www.linux.org
	; <<>> DiG 9.9.5-3ubuntu0.2-Ubuntu <<>> www.linux.org
	;; global options: +cmd
	;; Got answer:
	;; ->>HEADER<<- opcode: QUERY, status: NOERROR, id: 9156
	;; flags: qr rd ra; QUERY: 1, ANSWER: 2, AUTHORITY: 13, ADDITIONAL: 1

	;; OPT PSEUDOSECTION:
	; EDNS: version: 0, flags:; udp: 4096
	;; QUESTION SECTION:
	;www.linux.org.			IN	A

	;; ANSWER SECTION:
	www.linux.org.		12488	IN	CNAME	linux.org.
	linux.org.		1055	IN	A	107.170.40.56

	;; AUTHORITY SECTION:
	.			927	IN	NS	g.root-servers.net.
	.			927	IN	NS	i.root-servers.net.
	.			927	IN	NS	m.root-servers.net.
	.			927	IN	NS	b.root-servers.net.
	.			927	IN	NS	c.root-servers.net.
	.			927	IN	NS	j.root-servers.net.
	.			927	IN	NS	d.root-servers.net.
	.			927	IN	NS	f.root-servers.net.
	.			927	IN	NS	a.root-servers.net.
	.			927	IN	NS	h.root-servers.net.
	.			927	IN	NS	l.root-servers.net.
	.			927	IN	NS	k.root-servers.net.
	.			927	IN	NS	e.root-servers.net.

	;; Query time: 77 msec
	;; SERVER: 192.168.56.150#53(192.168.56.150)
	;; WHEN: Tue May 19 11:54:52 CEST 2015
	;; MSG SIZE  rcvd: 283

\end{lstlisting}
Test des lokalen Netzes (shib.lan)
\begin{lstlisting}
shib-idp@shibidp:~$ dig idp.shib.lan
	; <<>> DiG 9.9.5-3ubuntu0.2-Ubuntu <<>> idp.shib.lan
	;; global options: +cmd
	;; Got answer:
	;; ->>HEADER<<- opcode: QUERY, status: NOERROR, id: 57983
	;; flags: qr aa rd ra; QUERY: 1, ANSWER: 1, AUTHORITY: 1, ADDITIONAL: 2

	;; OPT PSEUDOSECTION:
	; EDNS: version: 0, flags:; udp: 4096
	;; QUESTION SECTION:
	;idp.shib.lan.			IN	A

	;; ANSWER SECTION:
	idp.shib.lan.		172800	IN	A	192.168.100.100

	;; AUTHORITY SECTION:
	shib.lan.		172800	IN	NS	shib.lan.

	;; ADDITIONAL SECTION:
	shib.lan.		172800	IN	A	192.168.100.100

	;; Query time: 1 msec
	;; SERVER: 192.168.100.100#53(192.168.100.100)
	;; WHEN: Tue May 19 11:56:47 CEST 2015
	;; MSG SIZE  rcvd: 87
\end{lstlisting}
Reverse-Lookup (idp.shib-lan hat im Beispiel 192.168.100.100)
\begin{lstlisting}
shib-idp@shibidp:~$ dig -x 192.168.100.100
	; <<>> DiG 9.9.5-3ubuntu0.2-Ubuntu <<>> -x 192.168.100.100
	;; global options: +cmd
	;; Got answer:
	;; ->>HEADER<<- opcode: QUERY, status: NOERROR, id: 341
	;; flags: qr aa rd ra; QUERY: 1, ANSWER: 1, AUTHORITY: 1, ADDITIONAL: 2

	;; OPT PSEUDOSECTION:
	; EDNS: version: 0, flags:; udp: 4096
	;; QUESTION SECTION:
	;100.100.168.192.in-addr.arpa.	IN	PTR

	;; ANSWER SECTION:
	100.100.168.192.in-addr.arpa. 172800 IN	PTR	shib.lan.

	;; AUTHORITY SECTION:
	100.168.192.in-addr.arpa. 172800	IN	NS	shib.lan.

	;; ADDITIONAL SECTION:
	shib.lan.		172800	IN	A	192.168.100.100

	;; Query time: 0 msec
	;; SERVER: 192.168.100.100#53(192.168.100.100)
	;; WHEN: Tue May 19 11:59:32 CEST 2015
	;; MSG SIZE  rcvd: 108
\end{lstlisting}
Reverse-Lookup für 192.168.100.200 sollte auf den noch aufzusetzenden SP mit sp.shib.lan zeigen:
\begin{lstlisting}
shib-idp@shibidp:~$ dig -x 192.168.100.200
	; <<>> DiG 9.9.5-3ubuntu0.2-Ubuntu <<>> -x 192.168.100.200
	;; global options: +cmd
	;; Got answer:
	;; ->>HEADER<<- opcode: QUERY, status: NOERROR, id: 8338
	;; flags: qr aa rd ra; QUERY: 1, ANSWER: 1, AUTHORITY: 1, ADDITIONAL: 2

	;; OPT PSEUDOSECTION:
	; EDNS: version: 0, flags:; udp: 4096
	;; QUESTION SECTION:
	;200.100.168.192.in-addr.arpa.	IN	PTR

	;; ANSWER SECTION:
	200.100.168.192.in-addr.arpa. 172800 IN	PTR	sp.shib.lan.

	;; AUTHORITY SECTION:
	100.168.192.in-addr.arpa. 172800	IN	NS	shib.lan.

	;; ADDITIONAL SECTION:
	shib.lan.		172800	IN	A	192.168.100.100

	;; Query time: 0 msec
	;; SERVER: 192.168.100.100#53(192.168.100.100)
	;; WHEN: Tue May 19 11:58:30 CEST 2015
	;; MSG SIZE  rcvd: 111
\end{lstlisting}
\step{Der DNS-Server kann jetzt auch auf dem VM-Host als Standard-Nameserver
für das Host-Only-Netzwerk eingetragen werden (das, in welchem sich Host, IdP
und SP bewegen sollen).}
\subsubsection{DHCP}
\step{DHCP3 Installieren:}
\begin{lstlisting}
shib-idp@shibidp:~$ sudo apt-get install isc-dhcp-server
\end{lstlisting}
\step{DHCP Config (/etc/dhcp/dhcpd.conf) - "`authoritative"' aktivieren,
Domainname setzen, Bereich "`subnet"' anlegen und host "`sp"' auf IP
192.168.100.200 fixieren (hier muss natürlich die MAC des eth1 Interfaces der SP-VM eingetragen werden):}
\begin{lstlisting}
shib-idp@shibidp:~$ cat /etc/dhcp/dhcpd.conf
	ddns-update-style none;
	option domain-name "shib.lan";
	option domain-name-servers 192.168.100.100;
	default-lease-time 600;
	max-lease-time 7200;
	authoritative;
	log-facility local7;
	subnet 192.168.100.0 netmask 255.255.255.0 {
		range 192.168.100.101 192.168.100.200;
		interface eth1;
	}

	host sp {
		hardware ethernet 08:00:27:9e:43:d3;
		fixed-address 192.168.100.200;
	}
\end{lstlisting}

\step{ DHCP starten:}
\begin{lstlisting}
shib-idp@shibidp:~$ sudo /etc/init.d/isc-dhcp-server start
\end{lstlisting}
\subsubsection{Certificate Authority aufsetzen und Zertifikate erstellen}
\step{CA erstellen (im Home-Verzeichnis möglich):}
\begin{lstlisting}
shib-idp@shibidp:~$ mkdir CA
shib-idp@shibidp:~$ cd CA
shib-idp@shibidp:~/CA$ /usr/lib/ssl/misc/CA.pl -newca
\end{lstlisting}
\begin{itemize}
 \item für Filename: <Enter> (neue erstellen)
 \item CA-Passphrase (auch für cakey.pem): +\$hib-idp.C@\_p@\$\$phr@5e**
 \item Country: DE
 \item State: Saxony
 \item Common Name: shib.lan
 \item Rest leer.
\end{itemize}

\step{eigene Zertifikate erstellen, die von der erstellten CA zertifiziert
sind:}
\anno{Anm.: es werden drei Zertifikate erstellt: eines für den PHP-LDAP-Admin,
eines für den Shibboleth IdP und ein "default"-Zertifikat (www.shib.lan)}
\anno{Anm.: vorher prüfen, ob das .rnd-File im Home-Verzeichnis dem Nutzer
gehört:}
\begin{lstlisting}
shib-idp@shibidp:~$ ll ~/ | grep .rnd
	-rw-------  1 root     root      1024 Mai 20 08:49 .rnd
\end{lstlisting}
Ist das nicht so (hier gehört das File root), dann sollte dies dem Nutzer (hier
shib-idp) übergeben werden:
\begin{lstlisting}
shib-idp@shibidp:~$ sudo chown shib-idp:shib-idp ~/.rnd
\end{lstlisting}
\step{neuen Request für LDAP-Zertifikat erstellen und signieren:}
\begin{lstlisting}
shib-idp@shibidp:~/CA$ /usr/lib/ssl/misc/CA.pl -newreq
\end{lstlisting}
\begin{itemize}
\item Hier genutzt:
\item Passphrase: +\$hib-idp.LD@P\_p@\$\$**
\item Country: DE
\item State: Saxony
\item Common Name: ldap.shib.lan
\item Rest leer.
\end{itemize}

\begin{lstlisting}
shib-idp@shibidp:~/CA$ /usr/lib/ssl/misc/CA.pl -sign
	Passphrase: CA-Passphrase (von oben)
	Sign Sertificae: y
	Commit: y
\end{lstlisting}
\step{im Verzeichnis CA sollte es jetzt so aussehen:}
\begin{lstlisting}
shib-idp@shibidp:~/CA$ ll
	drwxrwxr-x  3 shib-idp shib-idp 4096 Mai 20 10:28 ./
	drwxr-xr-x 20 shib-idp shib-idp 4096 Mai 20 10:09 ../
	drwxrwxr-x  6 shib-idp shib-idp 4096 Mai 20 10:29 demoCA/
	-rw-rw-r--  1 shib-idp shib-idp 4481 Mai 20 10:29 newcert.pem
	-rw-rw-r--  1 shib-idp shib-idp 1834 Mai 20 10:26 newkey.pem
	-rw-rw-r--  1 shib-idp shib-idp  985 Mai 20 10:26 newreq.pem
\end{lstlisting}
\step{für die Nutzung im Apache muss noch die Passphrase aus dem newkey.pem-File
entfernt werden:}
\begin{lstlisting}
shib-idp@shibidp:~/CA$ openssl rsa -in newkey.pem > ldap.shib.lan_key.pem
\end{lstlisting}
Passphrase: die vom LDAP-Key oben\newline
\step{Schlüssel und Key verschieben ("`Backup"'):}
\begin{lstlisting}
shib-idp@shibidp:~/CA$ mv ldap.shib.lan_key.pem demoCA/private/
shib-idp@shibidp:~/CA$ mv newcert.pem demoCA/certs/ldap.shib.lan_cert.pem
\end{lstlisting}
\step{Aufräumen:}
\begin{lstlisting}
shib-idp@shibidp:~/CA$ rm newreq.pem newkey.pem
\end{lstlisting}
\step{CA und LDAP-Cert bereitstellen:}
\begin{lstlisting}
shib-idp@shibidp:~/CA$ sudo cp demoCA/private/ldap.shib.lan_key.pem /etc/ssl/private/
shib-idp@shibidp:~/CA$ sudo cp demoCA/certs/ldap.shib.lan_cert.pem /etc/ssl/certs/
shib-idp@shibidp:~/CA$ sudo cp demoCA/cacert.pem /usr/share/ca-certificates/shib.lan/shib.lan.crt
\end{lstlisting}

\step{Das Zertifikat beinhalten noch "`Überhang"' der entfernt werden
muss:\newline Aus /etc/ssl/certs/ldap.shib.lan\_cert.pem muss alles entfernt werden, was
AUSSERHALB von BEGIN CERTIFICATE ... END CERTIFICATE steht, am Ende beinhaltet
die Datei nur noch etwas wie:}
\begin{lstlisting}
	-----BEGIN CERTIFICATE-----
	MIIDqzCCApOgAwIBAgIJAM857DKIFfSRMA0GCSqGSIb3DQEBCwUAMFQxCzAJBgNV
	.....
	DjW8g+KdwDa09JetID+kHiwto3NB3IN2YzUfr7nhSQ==
	-----END CERTIFICATE-----
\end{lstlisting}

\step{CA über ca-certificates importieren:}
\begin{lstlisting}
shib-idp@shibidp:~/CA$ sudo dpkg-reconfigure ca-certificates
\end{lstlisting}
neue Zertifizierungsstellen vertrauen: JA \newline
in der Liste shib,lan/shib.lan.crt raussuchen und anhaken (Leertaste), mit Enter weiter

\step{Zertifikate für www.shib.lan und idp.shib.lan erstellen:}
\begin{lstlisting}
shib-idp@shibidp:~/CA$ /usr/lib/ssl/misc/CA.pl -newreq
\end{lstlisting}
\begin{itemize}
\item Passphrase: +\$hib-idp.www\_p@\$\$**
\item Country: DE
\item State: Saxony
\item Common Name: ldap.shib.lan
\item Rest leer.
\end{itemize}
\begin{lstlisting}
			shib-idp@shibidp:~/CA$ /usr/lib/ssl/misc/CA.pl -sign
\end{lstlisting}
\begin{itemize}
  \item cakey-Passphrase com CA (oben)
  \item sign und commit: y
\end{itemize}
\begin{lstlisting}
shib-idp@shibidp:~/CA$ openssl rsa -in newkey.pem > www.shib.lan_key.pem
shib-idp@shibidp:~/CA$ mv www.shib.lan_key.pem demoCA/private/
shib-idp@shibidp:~/CA$ mv newcert.pem demoCA/certs/www.shib.lan_cert.pem
shib-idp@shibidp:~/CA$ rm new*
shib-idp@shibidp:~/CA$ /usr/lib/ssl/misc/CA.pl -newreq
\end{lstlisting}
\begin{itemize}
\item Passphrase: +\$hib-idp.idp\_p@\$\$**
\item Country: DE
\item State: Saxony
\item Common Name: ldap.shib.lan
\item Rest leer.
\end{itemize}
\begin{lstlisting}
shib-idp@shibidp:~/CA$ /usr/lib/ssl/misc/CA.pl -sign
\end{lstlisting}
\begin{itemize}
\item cakey-Passphrase com CA (oben)
\item sign und commit: y
\end{itemize}
\begin{lstlisting}
shib-idp@shibidp:~/CA$ openssl rsa -in newkey.pem > idp.shib.lan_key.pem
shib-idp@shibidp:~/CA$ mv idp.shib.lan_key.pem demoCA/private/
shib-idp@shibidp:~/CA$ mv newcert.pem demoCA/certs/idp.shib.lan_cert.pem
shib-idp@shibidp:~/CA$ rm new*
shib-idp@shibidp:~/CA$ sudo cp demoCA/certs/www.shib.lan_cert.pem /etc/ssl/certs/
shib-idp@shibidp:~/CA$ sudo cp demoCA/certs/idp.shib.lan_cert.pem /etc/ssl/certs/
shib-idp@shibidp:~/CA$ sudo cp demoCA/private/www.shib.lan_key.pem /etc/ssl/private/
shib-idp@shibidp:~/CA$ sudo cp demoCA/private/idp.shib.lan_key.pem
/etc/ssl/private/
\end{lstlisting}
\step{www.shib.lan\_cert.pem und
idp.shib.lan\_cert.pem (/etc/ssl/certs/) noch vom "`Überhang"' bereinigen (wie
oben)}

\step{ wird Chromium / Chrome zum Browsen verwendet, so sollte hier die CA als
 vertrauenswürdig hinzugefügt werden:}
\begin{itemize}
  \item Chromium:
  \begin{itemize}
    \item  Chromium in der VM: Einstellungen $\rightarrow$
  erweiterte Einstellungen $\rightarrow$ HTTPS/SSL (Zertifikate verwalten...) $\rightarrow$ Zertifizierungsstellen (Chromium) $\rightarrow$ Importieren... $\rightarrow$ cacert.pem aus /home/shib-idp/CA/demoCA/ importieren $\rightarrow$ Haken bei "`Diesem Zertifikat zur Identifizierung von Webseiten vertrauen"' $\rightarrow$ OK
  \item  Chromium auf VM-Host: analog, nur muss hier die cacert.pem noch per
  (Win)SCP oder Shared Folder für den Zugriff bereitgestellt
  werden)\footnote{\textbf{{\color{red} ACHTUNG:} DAMIT BEWERTET DAS
  WIRTS-SYSTEM VON DER DEMO-CA SIGNIERTE CERTS ALS ABSOLUT VERTRAUENSWÜRDIG! WENN MÖGLICH
  HIER DIE DEMO-CA NUR FÜR DIE VMs ALS VALIDE STAMMZERTIFIZIERUNGSSTELLE
  NUTZEN.
  AUF DEM VM-WIRT SOLLTE IN KAUF GENOMMEN WERDEN, IM BROWSER DIE WARNUNG
  WEGKLICKEN ZU MÜSSEN.}}
  \end{itemize}
\end{itemize}

\step{Mozilla: hier sollte es genügen, einmalig das Zertifikat herunterzuladen
(auf dem Hinweisbildschirm) und als Ausnahme zu genehmigen}
\step{ggf. ist noch ein Browser-Neustart nötig.}

\subsubsection{Apache Grundkonfig}
\step{Apache sollte per default schon für SSL vorbereitet sein, die ports.conf
sollte so aussehen:}
\begin{lstlisting}
shib-idp@shibidp:~$ cat /etc/apache2/ports.conf
	# If you just change the port or add more ports here, you will likely also
	# have to change the VirtualHost statement in
	# /etc/apache2/sites-enabled/000-default.conf

	Listen 80

	<IfModule ssl_module>
		Listen 443
	</IfModule>

	<IfModule mod_gnutls.c>
		Listen 443
	</IfModule>
\end{lstlisting}

\step{ggf. Apache SSL Modul aktivieren und Apache neustarten:}
\begin{lstlisting}
shib-idp@shibidp:~$ sudo a2enmod ssl
shib-idp@shibidp:~$ sudo service apache2 restart
\end{lstlisting}

\step{VHost für www.shib.lan:}
\begin{lstlisting}
shib-idp@shibidp:~$ cat /etc/apache2/sites-available/www.conf
	<VirtualHost*:443>
		ServerAdmin admin@shib.lan
		ServerName www.shib.lan
		ServerAlias www.shib.lan
		SSLEngine On
		SSLCertificateFile /etc/ssl/certs/www.shib.lan_cert.pem
		SSLCertificateKeyFile /etc/ssl/private/www.shib.lan_key.pem
		DocumentRoot "/var/www/html"
		<Directory "/var/www/html">
				Options FollowSymLinks
				AllowOverride AuthConfig
				Order allow,deny
				Allow from all
		</Directory>
		DirectoryIndex index.html
		ErrorLog ${APACHE_LOG_DIR}/error.log
		CustomLog ${APACHE_LOG_DIR}/access.log combined
	</VirtualHost>
\end{lstlisting}
\step{Vhost für Redirect von https://shib.lan und http://shib.lan auf
https://www.shib.lan}
\begin{lstlisting}
shib-idp@shibidp:~$ cat /etc/apache2/sites-available/redirect.conf
	<VirtualHost*:443>
	ServerName shib.lan
		ServerAlias shib.lan
		SSLEngine On
		SSLCertificateFile /etc/ssl/certs/www.shib.lan_cert.pem
		SSLCertificateKeyFile /etc/ssl/private/www.shib.lan_key.pem
		Redirect 301 / https://www.shib.lan/
		ErrorLog ${APACHE_LOG_DIR}/error.log
		CustomLog ${APACHE_LOG_DIR}/access.log combined
	</VirtualHost>

	<VirtualHost *:80>
		ServerAdmin admin@shib.lan
		ServerName shib.lan
		ServerAlias shib.lan
		Redirect 301 / https://www.shib.lan/
		ErrorLog ${APACHE_LOG_DIR}/error.log
		CustomLog ${APACHE_LOG_DIR}/access.log combined
	</VirtualHost>
\end{lstlisting}
\step{VHosts aktivieren und Apache neu starten}
\begin{lstlisting}
shib-idp@shibidp:~$ sudo a2ensite ldap.conf
shib-idp@shibidp:~$ sudo a2ensite www.conf
shib-idp@shibidp:~$ sudo a2ensite redirect.conf
shib-idp@shibidp:~$ sudo service apache2 restart
\end{lstlisting}

\subsubsection{OpenLDAP-Server}
\step{Pakete  slapd, ldap-utils installieren}
\begin{lstlisting}
shib-idp@shibidp:~$ sudo apt-get install slapd ldap-utils
\end{lstlisting}
\step{OpenLDAP konfigurieren:}
\begin{lstlisting}
shib-idp@shibidp:~$ sudo dpkg-reconfigure slapd
\end{lstlisting}
\begin{itemize}
	\item DNS: shib.lan
	\item Organisation: shib
	\item Password: idp
	\item Backend: hdb
	\item Löschen: nein
	\item Backup alte DB: ja
	\item LDAPv2 erlauben: nein
\end{itemize}

\step{TLS aktivieren:}
\begin{itemize}
  	\item Zertifikate in LDAP-Config laden: folgendes LDIF erzeugen und dieses
  ins Backend einfügen:
\end{itemize}
\begin{lstlisting}
shib-idp@shibidp:~$ cat /etc/ldap/tls.ldif
	\#\#\#\#\#\#\#\#\#\#\#\#\#\#\#\#\#\#\#\#\#\#\#\#\#\#\#\#\#\#\#\#\#\#\#\#\#\#\#\#\#\#\#\#\#\#\#\#\#\#\#\#\#\#\#\#\#\#\#
	\# CONFIGURATION for Support of TLS
	\#\#\#\#\#\#\#\#\#\#\#\#\#\#\#\#\#\#\#\#\#\#\#\#\#\#\#\#\#\#\#\#\#\#\#\#\#\#\#\#\#\#\#\#\#\#\#\#\#\#\#\#\#\#\#\#\#\#\#
	\# Add TLS supported access to user passwords for LDAP clients
	\# to the LDAP config.

	dn: cn=config
	changetype: modify
	add: olcTLSCACertificateFile
	olcTLSCACertificateFile: /etc/ssl/certs/ldap.shib.lan_cert.pem

	dn: cn=config
	changetype: modify
	add: olcTLSCertificateKeyFile
	olcTLSCertificateKeyFile: /etc/ssl/private/ldap.shib.lan_key.pem

	\#dn: cn=config
	\#changetype: modify
	\#delete: olcTLSCertificateFile

	dn: cn=config
	changetype: modify
	add: olcTLSCertificateFile
	olcTLSCertificateFile: /etc/ssl/certs/ldap.shib.lan_cert.pem

shib-idp@shibidp:~$ sudo ldapmodify -Y EXTERNAL -H ldapi:/// -f /etc/ldap/tls.ldif
\end{lstlisting}

\step{slapd (openldap) benötigt natürlich noch Rechte, um die
Zertifikatschlüssel auslesen zu dürfen:}
\begin{lstlisting}
shib-idp@shibidp:/etc/ldap$ sudo adduser openldap ssl-cert
\end{lstlisting}

\subsubsection*{(OPTIONAL phpldapadmin - man kann alle Arbeiten am LDAP auch
über das Apache Directory Studio wesentlich komfortabler erledigen)} * phpldapadmin installieren und Link erzeugen \footnote{Bug:
https://bugs.launchpad.net/ubuntu/+source/phpldapadmin/+bug/1321831}
\begin{lstlisting}
shib-idp@shibidp:~$ sudo ln -s /etc/apache2/conf-enabled /etc/apache2/conf.d
shib-idp@shibidp:~$ sudo apt-get install phpldapadmin
shib-idp@shibidp:~$ sudo rm /etc/apache2/conf.d
shib-idp@shibidp:~$ sudo service apache2 restart
\end{lstlisting}
* phpldapadmin für PHP >=5.5 patchen\footnote{
(http://sourceforge.net/u/nihilisticz/phpldapadmin/ci/7e53dab990748c546b79f0610c3a7a58431e9ebc/\#diff-1}:\newline
Diese Patches sind im aktuellen Paket schon angewandt, was fehlt ist
in /usr/share/phpldapadmin/lib/TemplateRender.php:
\begin{lstlisting}
Z.2469:
-- $default = $this->getServer()->getValue('appearance','password_hash');
++ $default = $this->getServer()->getValue('appearance','password_hash_custom');
\end{lstlisting}

* Apache VHost für phpldapadmin und SSL:
\begin{lstlisting}
shib-idp@shibidp:~$ cat /etc/apache2/sites-available/ldap.conf
	<VirtualHost *:443>
		ServerAdmin admin@shib.lan
		ServerName ldap.shib.lan
		ServerAlias ldap.shib.lan
		SSLEngine On
		SSLCertificateFile /etc/ssl/certs/ldap.shib.lan_cert.pem
		SSLCertificateKeyFile /etc/ssl/private/ldap.shib.lan_key.pem
		DocumentRoot "/usr/share/phpldapadmin/htdocs"
		<Directory "/usr/share/phpldapadmin">
				Options FollowSymLinks
				AllowOverride AuthConfig
				Order allow,deny
				Allow from all
		</Directory>
		DirectoryIndex index.php
		ErrorLog "/var/log/apache2/phpldapadmin_error.log"
		CustomLog "/var/log/apache2/phpldapadmin_access.log" combined
	</VirtualHost>

	<VirtualHost *:80>
			ServerAdmin admin@shib.lan
			ServerName ldap.shib.lan
			ServerAlias ldap.shib.lan
			Redirect 301 / https://ldap.shib.lan/
			ErrorLog ${APACHE_LOG_DIR}/error.log
			CustomLog ${APACHE_LOG_DIR}/access.log combined
	</VirtualHost>
\end{lstlisting}

* VHost aktivieren und Apache neu starten (oder auch reload, geht beides):
\begin{lstlisting}
shib-idp@shibidp:~$ sudo a2ensite ldap.conf
shib-idp@shibidp:~$ sudo service apache2 restart
\end{lstlisting}

* Schemata für eduPerson, schac, dfnEduPerson herunterladen:
\begin{lstlisting}
shib-idp@shibidp:~$ mkdir -p LDAP/schema
shib-idp@shibidp:~$ cd LDAP/schema
\end{lstlisting}
* folgende Schemata herunterladen:
\begin{itemize}
  \item
  https://wiki.refeds.org/download/attachments/1606048/schac-20150413-1.5.0.schema.txt
  \item https://spaces.internet2.edu/display/macedir/OpenLDAP+eduPerson
  \item
  https://www.aai.dfn.de/fileadmin/documents/attributes/200811/dfneduperson-1.0.schema.txt
\end{itemize}

* bis auf eduperson liegen die Schemata im "`schema"' Format vor, es wird jedoch LDIF benötigt. Daher ist hier eine Umwandlung nötig:
\newline * Anlegen einer schema\_convert.conf mit folgendem Inhalt:
\begin{lstlisting}
shib-idp@shibidp:~/LDAP/schema$ cat schema_convert.conf
	include /etc/ldap/schema/core.schema
	include /etc/ldap/schema/collective.schema
	include /etc/ldap/schema/corba.schema
	include /etc/ldap/schema/cosine.schema
	include /etc/ldap/schema/duaconf.schema
	include /etc/ldap/schema/dyngroup.schema
	include /etc/ldap/schema/inetorgperson.schema
	include /etc/ldap/schema/java.schema
	include /etc/ldap/schema/misc.schema
	include /etc/ldap/schema/nis.schema
	include /etc/ldap/schema/openldap.schema
	include /etc/ldap/schema/ppolicy.schema
	include /etc/ldap/schema/ldapns.schema
	include /etc/ldap/schema/schac-1.5.0.schema
	include /etc/ldap/schema/dfneduperson-1.0.schema
\end{lstlisting}

* dfneduperson und schac schema kopieren:
\begin{lstlisting}
	shib-idp@shibidp:~/LDAP/schema$ sudo cp dfneduperson-1.0.schema /etc/ldap/schema/
	shib-idp@shibidp:~/LDAP/schema$ sudo cp schac-1.5.0.schema /etc/ldap/schema/
\end{lstlisting}
* eduperson war schon LDIF, dies muss natürlich auch in's Schema:
\begin{lstlisting}
	shib-idp@shibidp:~/LDAP/schema$ sudo cp eduperson.ldif /etc/ldap/schema/
\end{lstlisting}
* konvertierung mittels slapcat:
\begin{lstlisting}
	shib-idp@shibidp:~/LDAP/schema$ mkdir ldif
	shib-idp@shibidp:~/LDAP/schema$ slapcat -f schema_convert.conf -F ./ldif -n0
\end{lstlisting}
* in den cn=schema - Ordner wechseln
\begin{lstlisting}
	shib-idp@shibidp:~/LDAP/schema$ cd ldif/cn\=config/cn\=schema/
\end{lstlisting}
* neu erstellte LDIFs nach /etc/ldap/schema kopieren (die Zahlen in den {} können variieren!):
\begin{lstlisting}
	shib-idp@shibidp:~/LDAP/schema/ldif/cn=config/cn=schema$ sudo mv cn\=\{13\}schac-1.ldif /etc/ldap/schema/schac-1.5.0.ldif
	shib-idp@shibidp:~/LDAP/schema/ldif/cn=config/cn=schema$ sudo mv cn\=\{14\}dfneduperson-1.ldif /etc/ldap/schema/dfneduperson-1.0.ldif
\end{lstlisting}
* die LDIFs müssen noch nachbearbeitet werden:
\begin{lstlisting}
	# Zeile 3 in dfneduperson-1.0.ldif muss geändert werden:
	Z3 -- dn: cn={14}dfneduperson-1
	Z3 ++ dn: cn=dfneduperson,cn=schema,cn=config
	\# Zeile 5 in dfneduperson-1.0.ldif:
	Z5 -- cn: {14}dfneduperson-1
	Z5 ++ cn: dfneduperson

	# die letzten 7 Zeilen müssen entfernt werden (Werte können variieren):
	Zn-6 -- structuralObjectClass: olcSchemaConfig
	Zn-5 -- entryUUID: 571ca564-932e-1034-8286-bdb5afc52f0d
	Zn-4 -- creatorsName: cn=config
	Zn-3 -- createTimestamp: 20150520112310Z
	Zn-2 -- entryCSN: 20150520112310.319976Z#000000#000#000000
	Zn-1 -- modifiersName: cn=config
	Zn   -- modifyTimestamp: 20150520112310Z
	# analog gilt für schac-1.5.0.ldif für Zeilen3, 5 und die letzten 7:
	Z3 -- dn: cn={13}schac-1
	Z3 ++ dn: cn=schac,cn=schema,cn=config
	Z5 -- cn: {13}schac-1
	Z5 ++ cn: schac
	Zn-6 -- structuralObjectClass: olcSchemaConfig
	Zn-5 -- entryUUID: 571c9ff6-932e-1034-8285-bdb5afc52f0d
	Zn-4 -- creatorsName: cn=config
	Zn-3 -- createTimestamp: 20150520112310Z
	Zn-2 -- entryCSN: 20150520112310.319976Z#000000#000#000000
	Zn-1 -- modifiersName: cn=config
	Zn   -- modifyTimestamp: 20150520112310Z
\end{lstlisting}
* LDIFs zum Schema von LDAP hinzufügen:
\begin{lstlisting}
	shib-idp@shibidp:~/LDAP/schema$ sudo ldapadd -Y EXTERNAL -H ldapi:/// -f /etc/ldap/schema/eduperson.ldif
	shib-idp@shibidp:~/LDAP/schema$ sudo ldapadd -Y EXTERNAL -H ldapi:/// -f /etc/ldap/schema/dfneduperson-1.0.ldif
	shib-idp@shibidp:~/LDAP/schema$ sudo ldapadd -Y EXTERNAL -H ldapi:/// -f /etc/ldap/schema/schac-1.5.0.ldif
\end{lstlisting}
* phpldapadmin konfigurieren (in /etc/phpldapadmin):
\begin{lstlisting}
	# Backup Orginal-Konfig:
		shib-idp@shibidp:/etc/phpldapadmin$ sudo cp config.php config.php.bak
	# Server-DN:
		Z300 -- $servers->setValue('server','base',array('dc=example,dc=com'));
		Z300 ++ $servers->setValue('server','base',array('dc=shib,dc=lan'));
		Z326 -- $servers->setValue('login','bind_id','cn=admin,dc=example,dc=com');
		Z326 ++ $servers->setValue('login','bind_id','cn=admin,dc=shib,dc=lan');
\end{lstlisting}
* das Web-Interface ist jetzt über https://ldap.shib.lan/ erreichbar, Login ist wie oben angegeben cn=admin,dc=shib,dc=lan mit Passwort idp

\subsubsection*{Apache Directory Studio - verwendung idealerweise auf dem
Hostrechner:}
\begin{itemize}
  \item aktuelle Version hier raussuchen
  https://directory.apache.org/studio/download/
  \item ist der DNS auf dem Host eingetragen, so kann einfach per ldap.shib.lan auf den ldap server zugegriffen werden
  \begin{itemize}
		\item user: cn=admin,dc=shib,dc=lan
		\item pass: idp
\end{itemize}
\end{itemize}
\subsubsection*{Baum und Beispielnutzer:}
Wurzel: dc=shib,dc=lan\newline
OUs: students, staff, extern\newline
\begin{center}
\footnotesize
	\begin{tabular}{| l | l | l | p{10cm} | }
	\hline
	uid & Passwort & ou & Attribute \\ \hline\hline
	alice & alice & students &
sn: Alison\newline
givenName: Alice\newline
cn: Alice Alison\newline
dfnEduPersonFieldOfStudyString: BWL\newline
dfnEduPersonStudyBranch3: 021\newline
dfnEduPersonStudyBranch2: 30\newline
dfnEduPersonStudyBranch1: 03\newline
eduPersonAffiliation: student\newline
eduPersonAffiliation: member\newline
eduPersonPrimaryAffiliation: student\newline
eduPersonEntitlement: urn:mace:dir:common-lib-terms
	 \\ \hline
	 bob & bob & students &
sn: Bobinson\newline
givenName: Bob\newline
cn: Bob Bobinson\newline
dfnEduPersonFieldOfStudyString: Maschinenbau\newline
dfnEduPersonStudyBranch1: 08\newline
dfnEduPersonStudyBranch2: 63\newline
dfnEduPersonStudyBranch3: 104\newline
eduPersonAffiliation: student\newline
eduPersonAffiliation: member\newline
eduPersonPrimaryAffiliation: student\newline
eduPersonEntitlement: urn:mace:dir:common-lib-terms\\ \hline
	ed & ed & students &
sn: Edison\newline
givenName: Ed\newline
cn: Ed Edison\newline
dfnEduPersonFieldOfStudyString: Verfahrenstechnik\newline
dfnEduPersonStudyBranch1: 08\newline
dfnEduPersonStudyBranch2: 63\newline
dfnEduPersonStudyBranch3: 226\newline
eduPersonAffiliation: student\newline
eduPersonAffiliation: member\newline
eduPersonPrimaryAffiliation: student\newline
eduPersonEntitlement: urn:mace:dir:common-lib-terms \\ \hline
	carl & carl & staff &
sn: Carlson\newline
givenName: Carl\newline
cn: Carl Carlson\newline
eduPersonAffiliation: member\newline
eduPersonAffiliation: staff\newline
eduPersonPrimaryAffiliation: staff\newline
dfnEduPersonCostCenter: 01-INFO\newline
eduPersonEntitlement: urn:mace:dir:common-lib-terms\\ \hline
	gunhild & gunhild & staff &
sn: Gunhildson\newline
givenName: Gundhild\newline
cn: Gunhild Gunhildson\newline
eduPersonAffiliation: member\newline
eduPersonAffiliation: staff\newline
eduPersonPrimaryAffiliation: staff\newline
dfnEduPersonCostCenter: 02-CHEM\newline
eduPersonEntitlement: urn:mace:dir:common-lib-terms\\ \hline
	nils & nils & extern &
sn: Nilson	\newline
givenName: Nilson\newline
cn: Nils Nilson\newline
eduPersonAffiliation: affiliate\newline
eduPersonPrimaryAffiliation: affiliate\\ \hline
	\end{tabular}
\end{center}
\subsubsection{Identity Provider}
\step{Apache Module installieren:}
\begin{lstlisting}
shib-idp@shibidp:~$ sudo a2enmod headers
shib-idp@shibidp:~$ sudo a2enmod proxy_ajp
shib-idp@shibidp:~$ sudo service apache2 restart
\end{lstlisting}
\step{Tomcat 7 installieren:}
\begin{lstlisting}
shib-idp@shibidp:~$ sudo apt-get install openjdk-7-jre
shib-idp@shibidp:~$ sudo apt-get install tomcat7
\end{lstlisting}
\step{Tomcat Konfiguration anpassen:}
* editieren der /etc/default/tomcat7:
\begin{lstlisting}
Z21 -- JAVA_OPTS="-Djava.awt.headless=true -Xmx128m -XX:+UseConcMarkSweepGC"
Z21 ++ JAVA_OPTS="-Djava.awt.headless=true -Xmx512m -XX:+UseConcMarkSweepGC"
Z32 -- #TOMCAT7_SECURITY=no
Z32 ++ TOMCAT7_SECURITY=no
\end{lstlisting}
* Rechte für SSL-Zertifikate:
\begin{lstlisting}
shib-idp@shibidp:~$ sudo adduser tomcat7 ssl-cert
\end{lstlisting}
\step{Identity Provider 2.4.4 installieren:}
\begin{lstlisting}
shib-idp@shibidp:~$ mkdir IdP
shib-idp@shibidp:~$ cd IdP/
shib-idp@shibidp:~/IdP$ wget http://shibboleth.net/downloads/identity-provider/2.4.4/shibboleth-identityprovider-2.4.4-bin.tar.gz
shib-idp@shibidp:~/IdP$ tar -xzf shibboleth-identityprovider-2.4.4-bin.tar.gz
shib-idp@shibidp:~/IdP$ cd shibboleth-identityprovider-2.4.4/
shib-idp@shibidp:~/IdP/shibboleth-identityprovider-2.4.4$ sudo JAVA_HOME=/usr/lib/jvm/java-7-openjdk-amd64/ ./install.sh
\end{lstlisting}
\begin{itemize}
	\item Installationverzeichnis: /opt/shibboleth-idp
	\item FQDN: idp.shib.lan
	\item Keystore Pass: *\$hib-idp.k3yst0r3\_p@\$\$**
\end{itemize}
\begin{lstlisting}
shib-idp@shibidp:~$ sudo chown tomcat7 /opt/shibboleth-idp/{metadata,logs}
\end{lstlisting}
\step{Apache und Tomcat Konfigurieren \footnote{
siehe
auch https://www.aai.dfn.de/dokumentation/identity-provider/konfiguration/}:}
* Apache-Config (VHost für IDP anlegen (/etc/apache2/sites-available)):
\begin{lstlisting}
shib-idp@shibidp:~$ cat /etc/apache2/sites-available/idp.conf
	<VirtualHost *:443>
	  ServerName              idp.shib.lan
	  SSLEngine               on
	  SSLCertificateFile      /etc/ssl/certs/idp.shib.lan_cert.pem
	  SSLCertificateKeyFile   /etc/ssl/private/idp.shib.lan_key.pem
	  SSLProtocol All -SSLv2 -SSLv3
	  SSLHonorCipherOrder On
	  SSLCipherSuite 'ECDH+AESGCM:DH+AESGCM:ECDH+AES256:DH+AES256:ECDH+AES128:DH+AES:RSA+AESGCM:RSA+AES:ECDH+3DES:DH+3DES:RSA+3DES:!aNULL:!eNULL:!LOW:!RC4:!MD5:!EXP:!PSK:!DSS:!SEED:!ECDSA:!CAMELLIA'

	  <Location /idp>
		Allow from all
		ProxyPass ajp://localhost:8009/idp
		#verhindern, dass die Login Seite in iFrames eingebunden wird
		Header always append X-FRAME-OPTIONS "DENY"
	  </Location>
	</VirtualHost>

	Listen 8443

	<VirtualHost *:8443>
	  ServerName              idp.hib.lan
	  SSLEngine               on
	  SSLCertificateFile      /opt/shibboleth-idp/credentials/idp.crt
	  SSLCertificateKeyFile   /opt/shibboleth-idp/credentials/idp.key
	  SSLProtocol All -SSLv2 -SSLv3
	  SSLHonorCipherOrder On
	  SSLCipherSuite 'ECDH+AESGCM:DH+AESGCM:ECDH+AES256:DH+AES256:ECDH+AES128:DH+AES:RSA+AESGCM:RSA+AES:ECDH+3DES:DH+3DES:RSA+3DES:!aNULL:!eNULL:!LOW:!RC4:!MD5:!EXP:!PSK:!DSS:!SEED:!ECDSA:!CAMELLIA'
	  SSLVerifyClient optional_no_ca
	  SSLVerifyDepth  10
	  SSLOptions      +StdEnvVars +ExportCertData

	  <Location /idp>
		Allow from all
		ProxyPass ajp://localhost:8009/idp
	  </Location>
	</VirtualHost>
\end{lstlisting}
\step{VHosts aktivieren:}
\begin{lstlisting}
shib-idp@shibidp:~$ sudo a2ensite idp.conf
shib-idp@shibidp:~$ sudo service apache2 reload
\end{lstlisting}
\step{Tomcat Konfiguration:}
* server.xml anpasssen (/etc/tomcat7/) und in den Abschnitt <Service name="Catalina"> (Z.56) einen neuen Connector einfügen:
\begin{lstlisting}
<!-- Define an AJP 1.3 Connector on port 8009 -->
<Connector port="8009" address="127.0.0.1"
  enableLookups="false"
  redirectPort="8443"
  protocol="AJP/1.3"
  maxPostSize="100000" />
\end{lstlisting}

* idp.xml erzeugen (in /etc/tomcat7/Catalina/localhost/)
\begin{lstlisting}
shib-idp@shibidp:~$ cat /etc/tomcat7/Catalina/localhost/idp.xml
<Context docBase="/opt/shibboleth-idp/war/idp.war"
		 privileged="true"
		 antiResourceLocking="false"
		 antiJARLocking="false"
		 unpackWAR="false"
		 swallowOutput="true" />
\end{lstlisting}
* Tomcat neu starten:
\begin{lstlisting}
shib-idp@shibidp:~$ sudo service tomcat7 restart
\end{lstlisting}
* kurz warten, bis der IdP komplett geladen ist.
* Aufrufen: https://idp.shib.lan/idp/profile/Status, hier sollte einfach
"`ok"' stehen - das IdP Servlet läuft!
\step{Logs bei Fehlern:}
\begin{itemize}
  \item IdP: /opt/shibboleth-idp/logs/idp-process.log
  \item Tomcat: /var/log/tomcat7/catalina.XXX.log (XXX = Datum)
\end{itemize}
\step{IdP konfigurieren:}
* service.xml anpassen (/opt/shibboleth-idp/conf) / polling Intervall für Auslesen des Attribute-Resolvers und -Filters auf 5 Minuten setzen für den Test:
\begin{lstlisting}
Z14: -- <srv:Service id="shibboleth.AttributeResolver" xsi:type="attribute-resolver:ShibbolethAttributeResolver">
Z14: ++ <srv:Service id="shibboleth.AttributeResolver" xsi:type="attribute-resolver:ShibbolethAttributeResolver" configurationResourcePollingFrequency="PT5M">

Z18: -- <srv:Service id="shibboleth.AttributeFilterEngine" xsi:type="attribute-afp:ShibbolethAttributeFilteringEngine">
Z18: ++ <srv:Service id="shibboleth.AttributeFilterEngine" xsi:type="attribute-afp:ShibbolethAttributeFilteringEngine" configurationResourcePollingFrequency="PT5M">
\end{lstlisting}

* relying-party.xml anpassen (/opt/shibboleth-idp/conf) / wir sind NICHT im
DFN-AAI mit dem Testsystem, daher unter <metadata:MetadataProvider
id="ShibbolethMetadata" xsi:type="metadata:ChainingMetadataProvider"> (Z. 76) zusätzlich einfügen:
\begin{lstlisting}
  <metadata:MetadataProvider id="sp.shib.lan" xsi:type="metadata:FileBackedHTTPMetadataProvider"
      metadataURL="https://sp.shib.lan/Shibboleth.sso/Metadata"
      backingFile="/opt/shibboleth-idp/metadata/sp-metadata.xml" />
\end{lstlisting}
* handler.xml (/opt/shibboleth-idp/conf) / Login-Methode unter <ph:LoginHandler>
konfigurieren:
\begin{lstlisting}
		Z115: --    <ph:LoginHandler xsi:type="ph:RemoteUser">
		Z116: --		<ph:AuthenticationMethod>urn:oasis:names:tc:SAML:2.0:ac:classes:unspecified</ph:AuthenticationMethod>
		Z117: --	</ph:LoginHandler>
		Z115: ++    <!-- <ph:LoginHandler xsi:type="ph:RemoteUser">
		Z116: ++		<ph:AuthenticationMethod>urn:oasis:names:tc:SAML:2.0:ac:classes:unspecified</ph:AuthenticationMethod>
		Z117: ++	</ph:LoginHandler> -->
		* "Einkommentieren" von UsernamePassword:
		Z129: -- <!--
		Z134: -- -->
\end{lstlisting}
* login.config (/opt/shibboleth-idp/conf) - LDAP Auth definieren:
\begin{lstlisting}
	Z29: -- /*
	Z30: --	edu.vt.middleware.ldap.jaas.LdapLoginModule required
	Z31: --	  ldapUrl="ldap://ldap.example.org"
	Z32: --	  baseDn="ou=people,dc=example,dc=org"
	Z33: --	  ssl="true"
	Z34: --	  userFilter="uid={0}";
	Z35: --	*/

	Z29: ++
	Z30: ++ edu.vt.middleware.ldap.jaas.LdapLoginModule required
	Z31: ++ 	  ldapUrl="ldap://ldap.shib.lan"
	Z32: ++ 	  baseDn="dc=shib,dc=lan"
	Z33: ++	      subtreeSearch="true"
	Z34: ++ 	  tls="true"
	Z35: ++ 	  bindDn="cn=admin,dc=shib,dc=lan"
	Z36: ++ 	  bindCredential="idp"
	Z37: ++ 	  userFilter="uid={0}";
\end{lstlisting}
* logging.xml (/opt/shibboleth-idp/conf) anpassen: (Debugging für LDAP setzen)
\begin{lstlisting}
	Z15: -- <logger name="edu.vt.middleware.ldap" level="WARN"/>
	Z15: ++ <logger name="edu.vt.middleware.ldap" level="DEBUG"/>
\end{lstlisting}
\step{Um jetzt den IdP testen zu können, muss hier der Tomcat neu gestartet
werden. Außerdem wird jetzt natürlich der Service Provider gebraucht.}
\begin{lstlisting}
shib-idp@shibidp:~$ sudo service tomcat7 restart
\end{lstlisting}

\subsection{Service Provider}
\subsubsection{Netzwerk}
\step{Interface auf DHCP für NAT und Host-Only (in der VM-Config müssen sich IdP
und SP im selben Host-Only-Netzwerk befinden)}
\begin{lstlisting}
shib-sp@shib-sp:~$ cat /etc/network/interfaces
	# interfaces(5) file used by ifup(8) and ifdown(8)
	auto lo
	iface lo inet loopback

	auto eth0
	iface eth0 inet dhcp

	auto eth1
	iface eth1 inet dhcp
\end{lstlisting}

\step{DNS auf Reihenfolge 192.168.100.100 forcen\footnote{Ansonsten
wird versucht, über eth0 aufzulösen, was natürlich nicht klappt. Hier muss noch eine Lösung
gefunden werden!}:}
* anpassen der dhclient.conf (/etc/dhcp) und neues DHCP-Lease holen:
\begin{lstlisting}
Z20: ++ supersede domain-name-servers 192.168.100.100;

shib-sp@shib-sp:~$ sudo ifdown eth0
shib-sp@shib-sp:~$ sudo ifdown eth1
shib-sp@shib-sp:~$ sudo ifup eth0
shib-sp@shib-sp:~$ sudo ifup eth1
\end{lstlisting}
\step{wenn der DHCP und DNS auf dem IdP korrekt konfiguriert ist, sollte
folgendes funktionieren:}
* DNS-Abfrage über NAT:
\begin{lstlisting}
shib-sp@shib-sp:~$ dig google.de
	; <<>> DiG 9.9.5-3ubuntu0.2-Ubuntu <<>> google.de
	;; global options: +cmd
	;; Got answer:
	;; ->>HEADER<<- opcode: QUERY, status: NOERROR, id: 21335
	;; flags: qr rd ra; QUERY: 1, ANSWER: 4, AUTHORITY: 13, ADDITIONAL: 1

	;; OPT PSEUDOSECTION:
	; EDNS: version: 0, flags:; udp: 4096
	;; QUESTION SECTION:
	;google.de.			IN	A

	;; ANSWER SECTION:
	google.de.		117	IN	A	173.194.32.223
	google.de.		117	IN	A	173.194.32.215
	google.de.		117	IN	A	173.194.32.207
	google.de.		117	IN	A	173.194.32.216

	;; AUTHORITY SECTION:
	.			18625	IN	NS	c.root-servers.net.
	.			18625	IN	NS	d.root-servers.net.
	.			18625	IN	NS	i.root-servers.net.
	.			18625	IN	NS	b.root-servers.net.
	.			18625	IN	NS	h.root-servers.net.
	.			18625	IN	NS	e.root-servers.net.
	.			18625	IN	NS	m.root-servers.net.
	.			18625	IN	NS	g.root-servers.net.
	.			18625	IN	NS	j.root-servers.net.
	.			18625	IN	NS	l.root-servers.net.
	.			18625	IN	NS	a.root-servers.net.
	.			18625	IN	NS	f.root-servers.net.
	.			18625	IN	NS	k.root-servers.net.

	;; Query time: 9 msec
	;; SERVER: 192.168.100.100#53(192.168.100.100)
	;; WHEN: Tue May 19 12:59:48 CEST 2015
	;; MSG SIZE  rcvd: 313
\end{lstlisting}
* DNS-Abfrage auf shib.lan:
\begin{lstlisting}
shib-sp@shib-sp:~$ dig shib.lan
	; <<>> DiG 9.9.5-3ubuntu0.2-Ubuntu <<>> shib.lan
	;; global options: +cmd
	;; Got answer:
	;; ->>HEADER<<- opcode: QUERY, status: NOERROR, id: 16736
	;; flags: qr aa rd ra; QUERY: 1, ANSWER: 1, AUTHORITY: 1, ADDITIONAL: 1

	;; OPT PSEUDOSECTION:
	; EDNS: version: 0, flags:; udp: 4096
	;; QUESTION SECTION:
	;shib.lan.			IN	A

	;; ANSWER SECTION:
	shib.lan.		172800	IN	A	192.168.100.100

	;; AUTHORITY SECTION:
	shib.lan.		172800	IN	NS	shib.lan.

	;; Query time: 0 msec
	;; SERVER: 192.168.100.100#53(192.168.100.100)
	;; WHEN: Tue May 19 12:59:52 CEST 2015
	;; MSG SIZE  rcvd: 67
\end{lstlisting}
* DNS auf sp.shib.lan (SP-VM selbst):
\begin{lstlisting}
shib-sp@shib-sp:~$ dig sp.shib.lan
	; <<>> DiG 9.9.5-3ubuntu0.2-Ubuntu <<>> sp.shib.lan
	;; global options: +cmd
	;; Got answer:
	;; ->>HEADER<<- opcode: QUERY, status: NOERROR, id: 23654
	;; flags: qr aa rd ra; QUERY: 1, ANSWER: 1, AUTHORITY: 1, ADDITIONAL: 2

	;; OPT PSEUDOSECTION:
	; EDNS: version: 0, flags:; udp: 4096
	;; QUESTION SECTION:
	;sp.shib.lan.			IN	A

	;; ANSWER SECTION:
	sp.shib.lan.		172800	IN	A	192.168.100.200

	;; AUTHORITY SECTION:
	shib.lan.		172800	IN	NS	shib.lan.

	;; ADDITIONAL SECTION:
	shib.lan.		172800	IN	A	192.168.100.100

	;; Query time: 0 msec
	;; SERVER: 192.168.100.100#53(192.168.100.100)
	;; WHEN: Tue May 19 13:00:01 CEST 2015
	;; MSG SIZE  rcvd: 86
\end{lstlisting}

\subsubsection{Apache und Shibboleth Daemon (shibd)}
\step{Apache + Shibboleth-SP Daemon (shibd) installieren}
\begin{lstlisting}
shib-sp@shib-sp:~$ sudo apt-get install apache2
shib-sp@shib-sp:~$ sudo a2enmod ssl
shib-sp@shib-sp:~$ sudo apt-get install libapache2-mod-shib2
shib-sp@shib-sp:~$ sudo adduser _shibd ssl-cert
shib-sp@shib-sp:~$ sudo service apache2 restart
shib-sp@shib-sp:~$ sudo service shibd restart
\end{lstlisting}
\step{wir brauchen noch ein gültiges Zertifikat für den SP. Zufälligerweise
haben wir ja im IdP eine eigene CA...}
* im Homeverzeichnis im SP:
\begin{lstlisting}
shib-sp@shib-sp:~$ mkdir certs
\end{lstlisting}

\step{$\rightarrow$ Wechseln zur IdP-VM}
* Wechseln nach CA im Home-Verzeichnis (cd ~/CA)
\begin{lstlisting}
shib-idp@shibidp:~/CA$ /usr/lib/ssl/misc/CA.pl -newreq
\end{lstlisting}
\begin{itemize}
\item Passphrase: +\$hib-idp.sp\_p@\$\$**
\item Country: DE
\item State: Saxony
\item Common Name: ldap.shib.lan
\item Rest leer.
\end{itemize}
\begin{lstlisting}
shib-idp@shibidp:~/CA$ /usr/lib/ssl/misc/CA.pl -sign
\end{lstlisting}
\begin{itemize}
\item cakey-Passphrase vom CA: +\$hib-idp.C@\_p@\$\$phr@5e**
\item sign und commit: y
\end{itemize}
\begin{lstlisting}
shib-idp@shibidp:~/CA$ openssl rsa -in newkey.pem > sp.shib.lan_key.pem
shib-idp@shibidp:~/CA$ mv sp.shib.lan_key.pem demoCA/private/
shib-idp@shibidp:~/CA$ mv newcert.pem demoCA/certs/sp.shib.lan_cert.pem
shib-idp@shibidp:~/CA$ rm new*
shib-idp@shibidp:~/CA$ scp demoCA/private/sp.shib.lan_key.pem shib-sp@sp.shib.lan:/home/shib-sp/certs
	# connect: yes
	# passwort: sp
shib-idp@shibidp:~/CA$ scp demoCA/certs/sp.shib.lan_cert.pem  shib-sp@sp.shib.lan:/home/shib-sp/certs
shib-idp@shibidp:~/CA$ scp demoCA/cacert.pem  shib-sp@sp.shib.lan:/home/shib-sp/certs/shib.lan.crt
\end{lstlisting}

\step{$\rightarrow$ Wechseln zur SP-VM}
\step{Zertifikate installieren}
* aus dem Zertifikat (sp.shib.lan\_cert.pem) wieder alles AUSSERHALB von BEGIN
RSA PRIVATE KEY und END RSA PRIVATE KEY entfernen, danach:
\begin{lstlisting}
shib-sp@shib-sp:~$ sudo cp certs/sp.shib.lan\_key.pem /etc/ssl/private/
shib-sp@shib-sp:~$ sudo cp certs/sp.shib.lan\_cert.pem /etc/ssl/certs/
shib-sp@shib-sp:~$ sudo mkdir /usr/share/ca-certificates/shib.lan
shib-sp@shib-sp:~$ sudo cp certs/shib.lan.crt /usr/share/ca-certificates/shib.lan/
shib-sp@shib-sp:~$ sudo dpkg-reconfigure ca-certificates
	# yes
	# shib.lan suchen und anhaken -> OK
\end{lstlisting}

* VHost für sp.shib.lan erzeugen:
\begin{lstlisting}
shib-sp@shib-sp:~$ cat /etc/apache2/sites-available/sp.conf
	<VirtualHost *:443>
	  ServerName              sp.shib.lan
	  ServerAdmin             webmaster@example.org
	  DocumentRoot            /var/www/html
	  SSLEngine on
	  SSLCertificateFile      /etc/ssl/certs/sp.shib.lan_cert.pem
	  SSLCertificateKeyFile   /etc/ssl/private/sp.shib.lan_key.pem

	  SSLCertificateChainFile /etc/ssl/certs/shib.lan.pem

	  SSLProtocol All -SSLv2 -SSLv3
	  SSLHonorCipherOrder On
	  SSLCipherSuite 'ECDH+AESGCM:DH+AESGCM:ECDH+AES256:DH+AES256:ECDH+AES128:DH+AES:RSA+AESGCM:RSA+AES:ECDH+3DES:DH+3DES:RSA+3DES:!aNULL:!eNULL:!LOW:!RC4:!MD5:!EXP:!PSK:!DSS:!SEED:!ECDSA:!CAMELLIA'

	  ErrorLog ${APACHE_LOG_DIR}/error.log
	  CustomLog ${APACHE_LOG_DIR}/access.log combined

	  <Location /shibtest>
		AuthType shibboleth
		ShibRequireSession On
		require valid-user
		DirectoryIndex shib-info.php
	  </Location>

	  # optional (Metadata-Access at entityID-URL)
	  Redirect seeother /shibboleth https://sp.shib.lan/Shibboleth.sso/Metadata

	</VirtualHost>

	<VirtualHost *:80>
	  ServerAdmin admin@shib.lan
	  ServerName sp.shib.lan
	  ServerAlias sp.shib.lan
	  Redirect 301 / https://sp.shib.lan/
	  ErrorLog ${APACHE_LOG_DIR}/error.log
	  CustomLog ${APACHE_LOG_DIR}/access.log combined

	  <Location /shibtest>
		Redirect 301 / https://sp.shib.lan/shibtest
	  </Location>

	</VirtualHost>
\end{lstlisting}
* Verzeichnis shibtest im Apache-www-Root erstellen und einfache
Beispiel-index.html erzeugen:
\begin{lstlisting}
shib-sp@shib-sp:~$ sudo mkdir /var/www/html/shibtest
shib-sp@shib-sp:~$ cat /var/www/html/shibtest/index.html
	<html>
	<head>
	<title>SHIBTEST</title>
	</head>
	<body>
	<h1>SHIBTEST</h1>
	</body>
	</html>

shib-sp@shib-sp:~$ sudo a2ensite sp.conf
shib-sp@shib-sp:~$ sudo service apache2 restart
\end{lstlisting}

* SP konfigurieren (shibboleth2.xml  in /etc/shibboleth/)
\begin{lstlisting}
Z23: -- <ApplicationDefaults entityID="https://sp.example.org/shibboleth"
Z24: --      REMOTE_USER="eppn persistent-id targeted-id"
Z23: ++ <ApplicationDefaults entityID="https://sp.shib.lan/shibboleth"
Z24: ++      REMOTE_USER="eppn persistent-id targeted-id uniqueID"
Z25: ++      signing="back" requireTransportAuth="false">

Z35: -- <Sessions lifetime="28800" timeout="3600" relayState="ss:mem"
Z56: --           checkAddress="false" handlerSSL="false" cookieProps="http">
Z35: ++ <Sessions lifetime="28800" timeout="3600" relayState="ss:mem"
Z36: ++           checkAddress="false" consistentAddress="true" handlerSSL="true" cookieProps="https">

Z44: -- <SSO entityID="https://idp.example.org/idp/shibboleth"
Z45: --   discoveryProtocol="SAMLDS" discoveryURL="https://ds.example.org/DS/WAYF">
Z46: --  SAML2 SAML1
Z47: -- </SSO>

Z44: ++ <SSO entityID="https://idp.shib.lan/idp/shibboleth">
Z45: ++ 	SAML2
Z46: ++ </SSO>

Z59: -- <Handler type="Session" Location="/Session" showAttributeValues="false"/>
Z59: ++ <Handler type="Session" Location="/Session" showAttributeValues="true"/>

Z77: ++ <MetadataProvider type="XML" url="https://idp.xacml.lan/idp/shibboleth" />

Z98: -- <CredentialResolver type="File" key="sp-key.pem" certificate="sp-cert.pem"/>
Z98: ++ <CredentialResolver type="File" key="/etc/ssl/private/sp.shib.lan_key.pem" certificate="/etc/ssl/certs/sp.shib.lan_cert.pem"/>
\end{lstlisting}

* shibd neu starten:
\begin{lstlisting}
	shib-sp@shib-sp:~$ sudo service shibd restart
\end{lstlisting}
* jetzt sollte der Login über Shibboleth funktionieren:
\begin{itemize}
\item https://sp.shib.lan/shibtest/ \footnote{Achtung: shibtest ist nur für SSL mit Shibboleth-Auth, normales http geht auch ohne! (Es ist ja nur ein Test ;-) )}
\item User:Password: alice:alice
\end{itemize}

\subsection{Identity Provider - Teil 2}
\subsubsection{uApprove}
\step{$\rightarrow$ Wechsel in IdP-VM}
\step{uApprove 2.6.0 herunterladen und entpacken}
\begin{lstlisting}
shib-idp@shibidp:~$ mkdir IdP/uApprove
shib-idp@shibidp:~$ cd IdP/uApprove/
shib-idp@shibidp:~/IdP/uApprove$ wget
https://forge.switch.ch/redmine/attachments/download/1823/uApprove-2.6.0.zip shib-idp@shibidp:~/uApprove$ unzip uApprove-2.6.0.zip
shib-idp@shibidp:~/IdP/uApprove$ cd uApprove-2.6.0/
\end{lstlisting}
\step{uApprove Bibliotheken in die IdP Quellen kopieren}
\begin{lstlisting}
shib-idp@shibidp:~/IdP/uApprove/uApprove-2.6.0$ cp lib/*.jar /home/shib-idp/IdP/shibboleth-identityprovider-2.4.4/lib/
shib-idp@shibidp:~/IdP/uApprove/uApprove-2.6.0$ cp lib/jdbc/*.jar /home/shib-idp/IdP/shibboleth-identityprovider-2.4.4/lib/
shib-idp@shibidp:~/IdP/uApprove/uApprove-2.6.0$ cp lib/jdbc/optional/mysql-connector-java-5.1.34.jar /home/shib-idp/IdP/shibboleth-identityprovider-2.4.4/lib/
\end{lstlisting}
\subsubsection*{uApprove Konfiguration, Web-App bereit stellen und SQL-DB
bauen:}
\step{Config-Templates kopieren}
\begin{lstlisting}
shib-idp@shibidp:~/IdP/uApprove/uApprove-2.6.0$ sudo cp manual/configuration/uApprove.properties /opt/shibboleth-idp/conf/
shib-idp@shibidp:~/IdP/uApprove/uApprove-2.6.0$ sudo cp manual/configuration/uApprove.xml /opt/shibboleth-idp/conf/
\end{lstlisting}
\step{Webapp kopieren:}
\begin{lstlisting}
shib-idp@shibidp:~/IdP/uApprove/uApprove-2.6.0$ mkdir /home/shib-idp/IdP/shibboleth-identityprovider-2.4.4/src/main/webapp/uApprove
shib-idp@shibidp:~/IdP/uApprove/uApprove-2.6.0$ cp webapp/* /home/shib-idp/IdP/shibboleth-identityprovider-2.4.4/src/main/webapp/uApprove/
\end{lstlisting}
\step{Datenbank vorbereiten:}
\begin{lstlisting}
shib-idp@shibidp:~/IdP/uApprove/uApprove-2.6.0$ mysql -uroot -p
	# Passwort: idp
\end{lstlisting}
\begin{lstlisting}[language=sql]
mysql> set names 'utf8';
mysql> set character set utf8;
mysql> charset utf8;
mysql> create database if not exists uApprove character set=utf8;
mysql> create user 'uApprove'@'localhost' identified by 'idp';
mysql> grant insert, select, update, delete on uApprove.* to
   'uApprove'@'localhost';
mysql> flush privileges;
mysql> create table ToUAcceptance (userId varchar(104) not null, version varchar(104) not null, fingerprint varchar(256) not null, acceptanceDate timestamp default current_timestamp not null, primary key (userId,version));
mysql> create table AttributeReleaseConsent(userId varchar(104) not null, relyingPartyId varchar(104) not null, attributeId varchar(104) not null, valuesHash varchar(256) not null, consentDate timestamp default current_timestamp not null, primary key (userId, relyingPartyId, attributeId));
mysql> show fields from AttributeReleaseConsent;
mysql> quit
\end{lstlisting}
\step{uApprove konfigurieren:}
* web.xml anpassen (/home/shib-idp/IdP/shibboleth-identityprovider-2.4.4/src/main/webapp/WEB-INF/)
\begin{lstlisting}
Z14: -- <param-value>$IDP_HOME$/conf/internal.xml; $IDP_HOME$/conf/service.xml;</param-value>
Z14: ++ <param-value>$IDP_HOME$/conf/internal.xml; $IDP_HOME$/conf/service.xml; $IDP_HOME$/conf/uApprove.xml;</param-value>

ab Z53ff: hinzufügen:
<!-- uApprove -->
<filter>
	<filter-name>uApprove</filter-name>
	<filter-class>ch.SWITCH.aai.uApprove.Intercepter</filter-class>
</filter>
<filter-mapping>
	<filter-name>uApprove</filter-name>
	<url-pattern>/profile/Shibboleth/SSO</url-pattern>
	<url-pattern>/profile/SAML1/SOAP/AttributeQuery</url-pattern>
	<url-pattern>/profile/SAML1/SOAP/ArtifactResolution</url-pattern>
	<url-pattern>/profile/SAML2/POST/SSO</url-pattern>
	<url-pattern>/profile/SAML2/POST-SimpleSign/SSO</url-pattern>
	<url-pattern>/profile/SAML2/Redirect/SSO</url-pattern>
	<url-pattern>/profile/SAML2/Unsolicited/SSO</url-pattern>
	<url-pattern>/Authn/UserPassword</url-pattern>
</filter-mapping>

<servlet>
	<servlet-name>uApprove - Terms Of Use</servlet-name>
	<servlet-class>ch.SWITCH.aai.uApprove.tou.ToUServlet</servlet-class>
</servlet>

<servlet-mapping>
	<servlet-name>uApprove - Terms Of Use</servlet-name>
	<url-pattern>/uApprove/TermsOfUse</url-pattern>
</servlet-mapping>

<servlet>
	<servlet-name>uApprove - Attribute Release</servlet-name>
	<servlet-class>ch.SWITCH.aai.uApprove.ar.AttributeReleaseServlet</servlet-class>
</servlet>

<servlet-mapping>
	<servlet-name>uApprove - Attribute Release</servlet-name>
	<url-pattern>/uApprove/AttributeRelease</url-pattern>
</servlet-mapping>
\end{lstlisting}
* uApprove.xml anpassen (/opt/shibboleth-idp/conf/)
\begin{lstlisting}
Z10: -- <context:property-placeholder location="classpath:/configuration/uApprove.properties" />
Z10: ++ <context:property-placeholder location="file:/opt/shibboleth-idp/conf/uApprove.properties" />
\end{lstlisting}
* uApprove.properties anpassen (/opt/shibboleth-idp/conf/)
\begin{lstlisting}
Z26: -- view.defaultLocale          = en
Z26: ++ view.defaultLocale          = de

Z38: -- database.password           = secret
Z38: ++ database.password           = idp
\end{lstlisting}
* Checkbox, um Zustimmung zu uApprove am IdP-Portal aufheben zu können
einfügen, dazu login.jsp anpassen (/home/shib-idp/IdP/shibboleth-identityprovider-2.4.4/src/main/webapp)
\begin{lstlisting}
Z50: ++ <section>
Z51: ++	 <input type="checkbox" name="uApprove.consent-revocation" value="true"/>
Z52: ++	 Clear data release consent for this service
Z53: ++	</section>
\end{lstlisting}
\subsubsection*{uApprove deployen}
\step{Deployment geht mit einem neuen Bauen der .war Files einher, der IdP wird
 daher neu installiert, wobei die alte Config beibehalten wird. Es wird nur
 das .war File neu erstellt (und kopiert):}
 \begin{lstlisting}
shib-idp@shibidp:~/IdP/shibboleth-identityprovider-2.4.4$ sudo JAVA_HOME=/usr/lib/jvm/java-7-openjdk-amd64/ ./install.sh
	# Installation nach: /opt/shibboleth-idp/ (default, also Enter)
	# bestehende Config überschreiben: no (!!!)
\end{lstlisting}
\step{uApprove Logger zu logging.xml hinzufügen (/opt/shibboleth-idp/conf):}
\begin{lstlisting}
Z17: ++ <!-- uApprove Logger -->
Z18: ++ <logger name="ch.SWITCH.aai.uApprove" level="DEBUG"/>
\end{lstlisting}
\step{abschließend Tomcat neu starten}
\begin{lstlisting}
shib-idp@shibidp:~/IdP/shibboleth-identityprovider-2.4.4$ sudo service tomcat7 restart
\end{lstlisting}

\subsubsection{attribut-resolver.xml und attribute-filter.xml anpassen}
\step{anpassen der attribute-resolver.xml (/opt/shibboleth-idp/conf) und
folgende Attribute aus der Auskommentierung entfernen:}
\begin{itemize}
\item uid
\item commonName
\item surname
\item givenName
\item eduPersonAffiliation
\item eduPersonEntitlement
\item eduPersonPrimaryAffiliation
\item eduPersonPrincipalName
\item eduPersonScopedAffiliation
\end{itemize}
* neu kommen die dfnEduPerson-Attribute hinzu (hier schon fertig aus- und
einkommentiert), ab Z.261 einzufügen:
\begin{lstlisting}[language=xml]
<!-- DFN Attribute -->
<resolver:AttributeDefinition xsi:type="ad:Simple" id="dfnEduPersonCostCenter" sourceAttributeID="dfnEduPersonCostCenter">
	<resolver:Dependency ref="myLDAP" />
	<resolver:AttributeEncoder xsi:type="enc:SAML1String" name="urn:mace:dir:attribute-def:dfnEduPersonCostCenter" />
	<resolver:AttributeEncoder xsi:type="enc:SAML2String" name="urn:oid:1.3.6.1.4.1.22177.400.1.1.3.1" friendlyName="dfnEduPersonCostCenter" />
</resolver:AttributeDefinition>

<resolver:AttributeDefinition xsi:type="ad:Simple" id="dfnEduPersonStudyBranch1" sourceAttributeID="dfnEduPersonStudyBranch1">
	<resolver:Dependency ref="myLDAP" />
	<resolver:AttributeEncoder xsi:type="enc:SAML1String" name="urn:mace:dir:attribute-def:dfnEduPersonStudyBranch1" />
	<resolver:AttributeEncoder xsi:type="enc:SAML2String" name="urn:oid:1.3.6.1.4.1.22177.400.1.1.3.2" friendlyName="dfnEduPersonStudyBranch1" />
</resolver:AttributeDefinition>

<resolver:AttributeDefinition xsi:type="ad:Simple" id="dfnEduPersonStudyBranch2" sourceAttributeID="dfnEduPersonStudyBranch2">
	<resolver:Dependency ref="myLDAP" />
	<resolver:AttributeEncoder xsi:type="enc:SAML1String" name="urn:mace:dir:attribute-def:dfnEduPersonStudyBranch2" />
	<resolver:AttributeEncoder xsi:type="enc:SAML2String" name="urn:oid:1.3.6.1.4.1.22177.400.1.1.3.3" friendlyName="dfnEduPersonStudyBranch2" />
</resolver:AttributeDefinition>

<resolver:AttributeDefinition xsi:type="ad:Simple" id="dfnEduPersonStudyBranch3" sourceAttributeID="dfnEduPersonStudyBranch3">
	<resolver:Dependency ref="myLDAP" />
	<resolver:AttributeEncoder xsi:type="enc:SAML1String" name="urn:mace:dir:attribute-def:dfnEduPersonStudyBranch2" />
	<resolver:AttributeEncoder xsi:type="enc:SAML2String" name="urn:oid:1.3.6.1.4.1.22177.400.1.1.3.4" friendlyName="dfnEduPersonStudyBranch3" />
</resolver:AttributeDefinition>

<resolver:AttributeDefinition xsi:type="ad:Simple" id="dfnEduPersonFieldOfStudyString" sourceAttributeID="dfnEduPersonFieldOfStudyString">
	<resolver:Dependency ref="myLDAP" />
	<resolver:AttributeEncoder xsi:type="enc:SAML1String" name="urn:mace:dir:attribute-def:dfnEduPersonFieldOfStudyString" />
	<resolver:AttributeEncoder xsi:type="enc:SAML2String" name="urn:oid:1.3.6.1.4.1.22177.400.1.1.3.5" friendlyName="dfnEduPersonFieldOfStudyString" />
</resolver:AttributeDefinition>
<!--
<resolver:AttributeDefinition xsi:type="ad:Simple" id="dfnEduPersonFinalDegree" sourceAttributeID="dfnEduPersonFinalDegree">
	<resolver:Dependency ref="myLDAP" />
	<resolver:AttributeEncoder xsi:type="enc:SAML1String" name="urn:mace:dir:attribute-def:dfnEduPersonFinalDegree" />
	<resolver:AttributeEncoder xsi:type="enc:SAML2String" name="urn:oid:1.3.6.1.4.1.22177.400.1.1.3.6" friendlyName="dfnEduPersonFinalDegree" />
</resolver:AttributeDefinition>

<resolver:AttributeDefinition xsi:type="ad:Simple" id="dfnEduPersonTypeOfStudy" sourceAttributeID="dfnEduPersonTypeOfStudy">
	<resolver:Dependency ref="myLDAP" />
	<resolver:AttributeEncoder xsi:type="enc:SAML1String" name="urn:mace:dir:attribute-def:dfnEduPersonTypeOfStudy" />
	<resolver:AttributeEncoder xsi:type="enc:SAML2String" name="urn:oid:1.3.6.1.4.1.22177.400.1.1.3.7" friendlyName="dfnEduPersonTypeOfStudy" />
</resolver:AttributeDefinition>

<resolver:AttributeDefinition xsi:type="ad:Simple" id="dfnEduPersonTermsOfStudy" sourceAttributeID="dfnEduPersonTermsOfStudy">
	<resolver:Dependency ref="myLDAP" />
	<resolver:AttributeEncoder xsi:type="enc:SAML1String" name="urn:mace:dir:attribute-def:dfnEduPersonTermsOfStudy" />
	<resolver:AttributeEncoder xsi:type="enc:SAML2String" name="urn:oid:1.3.6.1.4.1.22177.400.1.1.3.8" friendlyName="dfnEduPersonTermsOfStudy" />
</resolver:AttributeDefinition>

<resolver:AttributeDefinition xsi:type="ad:Simple" id="dfnEduPersonBranchAndDegree" sourceAttributeID="dfnEduPersonBranchAndDegree">
	<resolver:Dependency ref="myLDAP" />
	<resolver:AttributeEncoder xsi:type="enc:SAML1String" name="urn:mace:dir:attribute-def:dfnEduPersonBranchAndDegree" />
	<resolver:AttributeEncoder xsi:type="enc:SAML2String" name="urn:oid:1.3.6.1.4.1.22177.400.1.1.3.9" friendlyName="dfnEduPersonBranchAndDegree" />
</resolver:AttributeDefinition>

<resolver:AttributeDefinition xsi:type="ad:Simple" id="dfnEduPersonBrachAndType" sourceAttributeID="dfnEduPersonBrachAndType">
	<resolver:Dependency ref="myLDAP" />
	<resolver:AttributeEncoder xsi:type="enc:SAML1String" name="urn:mace:dir:attribute-def:dfnEduPersonBrachAndType" />
	<resolver:AttributeEncoder xsi:type="enc:SAML2String" name="urn:oid:1.3.6.1.4.1.22177.400.1.1.3.10" friendlyName="dfnEduPersonBrachAndType" />
</resolver:AttributeDefinition>

<resolver:AttributeDefinition xsi:type="ad:Simple" id="dfnEduPersonFeaturesOfStudy" sourceAttributeID="dfnEduPersonFeaturesOfStudy">
	<resolver:Dependency ref="myLDAP" />
	<resolver:AttributeEncoder xsi:type="enc:SAML1String" name="urn:mace:dir:attribute-def:dfnEduPersonFeaturesOfStudy" />
	<resolver:AttributeEncoder xsi:type="enc:SAML2String" name="urn:oid:1.3.6.1.4.1.22177.400.1.1.3.11" friendlyName="dfnEduPersonFeaturesOfStudy" />
</resolver:AttributeDefinition>
-->
\end{lstlisting}
* LDAP-Connector einfügen (Anpassung des "`Example LDAP Connector"', einkommentieren nicht vergessen):
\begin{lstlisting}[language=xml]
<resolver:DataConnector id="myLDAP" xsi:type="dc:LDAPDirectory"
	ldapURL="ldap://ldap.shib.lan"
	baseDN="dc=shib,dc=lan"
	principal="uid=shib_ldap,dc=shib,dc=lan"
	principalCredential="shib_ldap">
	<dc:FilterTemplate>
		<![CDATA[
			(uid=$requestContext.principalName)
		]]>
	</dc:FilterTemplate>
</resolver:DataConnector>
\end{lstlisting}
\step{Anpassung attribute-filter.xml  (/opt/shibboleth-idp/conf)}
* folgende Policy ab Zeile 26 einfügen:
\begin{lstlisting}[language=xml]
<afp:AttributeFilterPolicy id="sp.shib.lan-Policy">

	<afp:PolicyRequirementRule xsi:type="basic:AttributeRequesterString" value="https://sp.shib.lan/shibboleth" />

	<afp:AttributeRule attributeID="eduPersonAffiliation">
		<afp:PermitValueRule xsi:type="basic:OR">
			<basic:Rule xsi:type="basic:AttributeValueString" value="faculty" ignoreCase="true" />
			<basic:Rule xsi:type="basic:AttributeValueString" value="student" ignoreCase="true" />
			<basic:Rule xsi:type="basic:AttributeValueString" value="staff" ignoreCase="true" />
			<basic:Rule xsi:type="basic:AttributeValueString" value="alum" ignoreCase="true" />
			<basic:Rule xsi:type="basic:AttributeValueString" value="member" ignoreCase="true" />
			<basic:Rule xsi:type="basic:AttributeValueString" value="affiliate" ignoreCase="true" />
			<basic:Rule xsi:type="basic:AttributeValueString" value="employee" ignoreCase="true" />
			<basic:Rule xsi:type="basic:AttributeValueString" value="library-walk-in" ignoreCase="true" />
		</afp:PermitValueRule>
	</afp:AttributeRule>

	<afp:AttributeRule attributeID="eduPersonPrimaryAffiliation">
		<afp:PermitValueRule xsi:type="basic:OR">
			<basic:Rule xsi:type="basic:AttributeValueString" value="faculty" ignoreCase="true" />
			<basic:Rule xsi:type="basic:AttributeValueString" value="student" ignoreCase="true" />
			<basic:Rule xsi:type="basic:AttributeValueString" value="staff" ignoreCase="true" />
			<basic:Rule xsi:type="basic:AttributeValueString" value="alum" ignoreCase="true" />
			<basic:Rule xsi:type="basic:AttributeValueString" value="member" ignoreCase="true" />
			<basic:Rule xsi:type="basic:AttributeValueString" value="affiliate" ignoreCase="true" />
			<basic:Rule xsi:type="basic:AttributeValueString" value="employee" ignoreCase="true" />
			<basic:Rule xsi:type="basic:AttributeValueString" value="library-walk-in" ignoreCase="true" />
		</afp:PermitValueRule>
	</afp:AttributeRule>

	<afp:AttributeRule attributeID="commonName">
		<afp:PermitValueRule xsi:type="basic:ANY" />
	</afp:AttributeRule>
	<afp:AttributeRule attributeID="surname">
		<afp:PermitValueRule xsi:type="basic:ANY" />
	</afp:AttributeRule>
	<afp:AttributeRule attributeID="givenName">
		<afp:PermitValueRule xsi:type="basic:ANY" />
	</afp:AttributeRule>
	<afp:AttributeRule attributeID="eduPersonPrincipalName">
		<afp:PermitValueRule xsi:type="basic:ANY" />
	</afp:AttributeRule>
	<afp:AttributeRule attributeID="eduPersonScopedAffiliation">
		<afp:PermitValueRule xsi:type="basic:ANY" />
	</afp:AttributeRule>
	<afp:AttributeRule attributeID="eduPersonPrimaryAffiliation">
		<afp:PermitValueRule xsi:type="basic:ANY" />
	</afp:AttributeRule>
	<afp:AttributeRule attributeID="dfnEduPersonStudyBranch1">
		<afp:PermitValueRule xsi:type="basic:ANY" />
	</afp:AttributeRule>
	<afp:AttributeRule attributeID="dfnEduPersonStudyBranch2">
		<afp:PermitValueRule xsi:type="basic:ANY" />
	</afp:AttributeRule>
	<afp:AttributeRule attributeID="dfnEduPersonStudyBranch3">
		<afp:PermitValueRule xsi:type="basic:ANY" />
	</afp:AttributeRule>
	<afp:AttributeRule attributeID="dfnEduPersonFieldOfStudyString">
		<afp:PermitValueRule xsi:type="basic:ANY" />
	</afp:AttributeRule>
	<afp:AttributeRule attributeID="eduPersonEntitlement">
		<afp:PermitValueRule xsi:type="basic:ANY" />
	</afp:AttributeRule>
	<afp:AttributeRule attributeID="dfnEduPersonCostCenter">
		<afp:PermitValueRule xsi:type="basic:ANY" />
	</afp:AttributeRule>

</afp:AttributeFilterPolicy>
\end{lstlisting}
\step{Abschließend Tomcat neu starten:}
\begin{lstlisting}
shib-idp@shibidp:~$ sudo service tomcat7 restart
\end{lstlisting}
\subsubsection*{Attribute-Mapping auf dem SP anpassen:}
$\rightarrow$ Wechsel zur SP-VM
\step{alle Attribute, die vom IdP kommen, müssen im SP noch gemappt werden. Dazu
die attribute-map.xml (/etc/shibboleth) anpassen:}
* einkommentieren von:
\begin{itemize}
\item eduPersonPrimaryAffiliation
\item cn
\item sn
\item givenName
\end{itemize}
\step{Attribute sowohl im alten NameID- als auch OID-Format einkommentieren!}
* zusätzlich ist für die eduPersonScopedAffiliation die oid-Variante einzufügen:
\begin{lstlisting}
<Attribute name="urn:oid:1.3.6.1.4.1.5923.1.1.1.5" id="primary-affiliation">
	<AttributeDecoder xsi:type="StringAttributeDecoder" caseSensitive="false"/>
</Attribute>
\end{lstlisting}
* außerdem neu hinzufügen:
\begin{lstlisting}[language=xml]
<!-- DFN Attribute dfnEduPerson -->
<Attribute name="urn:mace:dir:attribute-def:dfnEduPersonCostCenter" id="dfnEduPersonCostCenter">
	<AttributeDecoder xsi:type="StringAttributeDecoder" caseSensitive="false"/>
</Attribute>
<Attribute name="urn:oid:1.3.6.1.4.1.22177.400.1.1.3.1" id="dfnEduPersonCostCenter">
	<AttributeDecoder xsi:type="StringAttributeDecoder" caseSensitive="false"/>
</Attribute>

<Attribute name="urn:mace:dir:attribute-def:dfnEduPersonStudyBranch1" id="dfnEduPersonStudyBranch1">
	<AttributeDecoder xsi:type="StringAttributeDecoder" caseSensitive="false"/>
</Attribute>
<Attribute name="urn:oid:1.3.6.1.4.1.22177.400.1.1.3.2" id="dfnEduPersonStudyBranch1">
	<AttributeDecoder xsi:type="StringAttributeDecoder" caseSensitive="false"/>
</Attribute>

<Attribute name="urn:mace:dir:attribute-def:dfnEduPersonStudyBranch2" id="dfnEduPersonStudyBranch2">
	<AttributeDecoder xsi:type="StringAttributeDecoder" caseSensitive="false"/>
</Attribute>
<Attribute name="urn:oid:1.3.6.1.4.1.22177.400.1.1.3.3" id="dfnEduPersonStudyBranch2">
	<AttributeDecoder xsi:type="StringAttributeDecoder" caseSensitive="false"/>
</Attribute>

<Attribute name="urn:mace:dir:attribute-def:dfnEduPersonStudyBranch3" id="dfnEduPersonStudyBranch3">
	<AttributeDecoder xsi:type="StringAttributeDecoder" caseSensitive="false"/>
</Attribute>
<Attribute name="urn:oid:1.3.6.1.4.1.22177.400.1.1.3.4" id="dfnEduPersonStudyBranch3">
	<AttributeDecoder xsi:type="StringAttributeDecoder" caseSensitive="false"/>
</Attribute>

<Attribute name="urn:mace:dir:attribute-def:dfnEduPersonFieldOfStudyString" id="dfnEduPersonFieldOfStudyString">
	<AttributeDecoder xsi:type="StringAttributeDecoder" caseSensitive="false"/>
</Attribute>
<Attribute name="urn:oid:1.3.6.1.4.1.22177.400.1.1.3.5" id="dfnEduPersonFieldOfStudyString">
	<AttributeDecoder xsi:type="StringAttributeDecoder" caseSensitive="false"/>
</Attribute>

<!--
<Attribute name="urn:mace:dir:attribute-def:dfnEduPersonFinalDegree" id="dfnEduPersonFinalDegree">
	<AttributeDecoder xsi:type="StringAttributeDecoder" caseSensitive="false"/>
</Attribute>
<Attribute name="urn:oid:1.3.6.1.4.1.22177.400.1.1.3.6" id="dfnEduPersonFinalDegree">
	<AttributeDecoder xsi:type="StringAttributeDecoder" caseSensitive="false"/>
</Attribute>

<Attribute name="urn:mace:dir:attribute-def:dfnEduPersonTypeOfStudy" id="dfnEduPersonTypeOfStudy">
	<AttributeDecoder xsi:type="StringAttributeDecoder" caseSensitive="false"/>
</Attribute>
<Attribute name="urn:oid:1.3.6.1.4.1.22177.400.1.1.3.7" id="dfnEduPersonTypeOfStudy">
	<AttributeDecoder xsi:type="StringAttributeDecoder" caseSensitive="false"/>
</Attribute>

<Attribute name="urn:mace:dir:attribute-def:dfnEduPersonTermsOfStudy" id="dfnEduPersonTermsOfStudy">
	<AttributeDecoder xsi:type="StringAttributeDecoder" caseSensitive="false"/>
</Attribute>
<Attribute name="urn:oid:1.3.6.1.4.1.22177.400.1.1.3.8" id="dfnEduPersonTermsOfStudy">
	<AttributeDecoder xsi:type="StringAttributeDecoder" caseSensitive="false"/>
</Attribute>

<Attribute name="urn:mace:dir:attribute-def:dfnEduPersonBranchAndDegree" id="dfnEduPersonBranchAndDegree">
	<AttributeDecoder xsi:type="StringAttributeDecoder" caseSensitive="false"/>
</Attribute>
<Attribute name="urn:oid:1.3.6.1.4.1.22177.400.1.1.3.9" id="dfnEduPersonBranchAndDegree">
	<AttributeDecoder xsi:type="StringAttributeDecoder" caseSensitive="false"/>
</Attribute>

<Attribute name="urn:mace:dir:attribute-def:dfnEduPersonBrachAndType" id="dfnEduPersonBrachAndType">
	<AttributeDecoder xsi:type="StringAttributeDecoder" caseSensitive="false"/>
</Attribute>
<Attribute name="urn:oid:1.3.6.1.4.1.22177.400.1.1.3.10" id="dfnEduPersonBrachAndType">
	<AttributeDecoder xsi:type="StringAttributeDecoder" caseSensitive="false"/>
</Attribute>

<Attribute name="urn:mace:dir:attribute-def:dfnEduPersonFeaturesOfStudy" id="dfnEduPersonFeaturesOfStudy">
	<AttributeDecoder xsi:type="StringAttributeDecoder" caseSensitive="false"/>
</Attribute>
<Attribute name="urn:oid:1.3.6.1.4.1.22177.400.1.1.3.11" id="dfnEduPersonFeaturesOfStudy">
	<AttributeDecoder xsi:type="StringAttributeDecoder" caseSensitive="false"/>
</Attribute>
-->
\end{lstlisting}

\step{Auf den shibtest im SP ( https://sp.shib.lan/shibtest ) kann jetzt per IdP
Authentifizierung und uApprove zugegriffen werden}

\step{Optional: Anpassung der shibtest-Seite auf dem SP für die Ausgabe der
übertragenen Shibboleth-Attribute:}
$\rightarrow$ Wechsel in die SP-VM:
\step{sp.conf anpassen (/etc/apache2/sites-available)}
\begin{lstlisting}
Z19: ++    DirectoryIndex shib-info.php
\end{lstlisting}
\step{shib-info.php in /var/www/html/shibtest anlegen:}
\begin{lstlisting}[language=html]
shib-sp@shib-sp:/var/www/html/shibtest$ cat shib-info.php
	<html>
	<head>
	<title>SHIB-INFO</title>
	</head>
	<body>
	<h1>SHIBTEST</h1>
	<h2>$_SERVER:</h2>
	<?php
		foreach($_SERVER as $key => $value)
		{
			echo "<pre>".$key.": ".$value."</pre>";
		}
	?>
	</body>
\end{lstlisting}

\subsection{optionale Einstellungen}
\subsubsection{Statusseiten für IdP und SP verfügbar machen}
ACHTUNG: Das Verfügbarmachen der Statusseiten von "`außerhalb"' ist für den
Produktivbetrieb natürlich nicht sinnvoll, da hier z.T. interne Informationen
angezeigt werden!
\subsubsection*{IdP}
\step{Anpassung der web.xml (/home/shib-idp/IdP/shibboleth-identityprovider-2.4.4/src/main/webapp/WEB-INF)}:
\begin{lstlisting}
<!-- Servlet for displaying IdP status. -->
 <servlet>
   <servlet-name>Status</servlet-name>
    <servlet-class>edu.internet2.middleware.shibboleth.idp.StatusServlet</servlet-class>
     <!-- Space separated list of CIDR blocks allowed to access the status page -->
      <init-param>
        <param-name>AllowedIPs</param-name>
--      <param-value>127.0.0.1/32 ::1/128</param-value>
++      <param-value>127.0.0.1/32 ::1/128 192.168.100.0/24</param-value>
      </init-param>
    <load-on-startup>2</load-on-startup>
  </servlet>
\end{lstlisting}
\step{IdP neu deployen und Tomcat neu starten:}
\begin{lstlisting}
shib-idp@shibidp:~/IdP/shibboleth-identityprovider-2.4.4$ sudo JAVA\_HOME=/usr/lib/jvm/java-7-openjdk-amd64/ ./install.sh
	# bestehende config NICHT überschreiben!
shib-idp@shibidp:~$ sudo service tomcat7 restart
\end{lstlisting}
\step{IdP-Status-Seite: https://idp.shib.lan/idp/status}
\subsubsection{SP}
\step{shibboleth2.xml (/etc/shibboleth) anpassen:}
\begin{lstlisting}
-- <Handler type="Status" Location="/Status" acl="127.0.0.1 ::1"/>
++ <Handler type="Status" Location="/Status" acl="127.0.0.1 ::1  192.168.100.0/24"/>
\end{lstlisting}
\step{shibd neu starten:}
\begin{lstlisting}
shib-sp@shib-sp:~$ sudo service shibd restart
\end{lstlisting}
\step{SP-Status-Seite: https://sp.shib.lan/Shibboleth.sso/Status}

\section{Konfigurationsdateien}
Die Configs werden hier eher zum Nachschlagen als zum vollständigen Kopieren
gelistet. Kommentare und Definitionen - insbesondere bei den XML-Dateien -
erstrecken sich meist über eine lange Zeile und werden hier im PDF dann
umgebrochen ausgegeben. Also Vorsicht bei Copy-Paste! Die Configs spiegeln den
Stand der VMs nach der normalen Installation wider - die optionalen
Einstellungen sind hier \textbf{NICHT} aufgeführt!
\subsection{Identity Provider}
\subsubsection{/etc/network/interfaces}
\begin{lstlisting}
# interfaces(5) file used by ifup(8) and ifdown(8)
auto lo
iface lo inet loopback

auto eth0
iface eth0 inet dhcp

auto eth1
iface eth1 inet static
	address 192.168.100.100
	netmask 255.255.255.0
	dns-nameservers 192.168.100.100
	dns-search shib.lan
\end{lstlisting}
\subsubsection{/etc/default/bind9}
\begin{lstlisting}
# run resolvconf?
RESOLVCONF=no

# startup options for the server
OPTIONS="-u bind -4"
\end{lstlisting}

\subsubsection{/etc/bind/db.shib.lan}
\begin{lstlisting}
;; db.shib.lan
;; Forwardlookupzone für domainname
;;
$TTL 2D
@       IN      SOA     shib.lan. mail.shib.lan. (
                        2015051801      ; Serial
                                8H      ; Refresh
                                2H      ; Retry
                                4W      ; Expire
                                3H )    ; NX (TTL Negativ Cache)

@                               IN      NS      shib.lan.
                                IN      MX      10 mailserver.shib.lan.
                                IN      A       192.168.100.100

idp                             IN      A       192.168.100.100
localhost                       IN      A       127.0.0.1
sp                              IN      A       192.168.100.200
ldap                            IN      A       192.168.100.100
www                             IN      A       192.168.100.100

\end{lstlisting}

\subsubsection{/etc/bind/db.100.168.192}
\begin{lstlisting}
;; db.100.168.192
;; Reverselookupzone für shib.lan
;;
$TTL 2D
@       IN      SOA     shib.lan. mail.shib-idp.lan. (
                                2015051801      ; Serial
                                        8H      ; Refresh
                                        2H      ; Retry
                                        4W      ; Expire
                                        2D )    ; TTL Negative Cache

@       IN      NS      shib.lan.

100     IN      PTR     shib.lan.
200     IN      PTR     sp.shib.lan.

\end{lstlisting}

\subsubsection{/etc/bind/named.conf.local}
\begin{lstlisting}
//
// Do any local configuration here
//

// Consider adding the 1918 zones here, if they are not used in your
// organization
//include "/etc/bind/zones.rfc1918";

zone "shib.lan" {
type master;
file "/etc/bind/db.shib.lan";
};

zone "100.168.192.in-addr.arpa" {
type master;
file "/etc/bind/db.100.168.192";
};
\end{lstlisting}

\subsubsection{/etc/bind/named.conf.options}
\begin{lstlisting}
options {
	directory "/var/cache/bind";

	// If there is a firewall between you and nameservers you want
	// to talk to, you may need to fix the firewall to allow multiple
	// ports to talk.  See http://www.kb.cert.org/vuls/id/800113

	// If your ISP provided one or more IP addresses for stable 
	// nameservers, you probably want to use them as forwarders.  
	// Uncomment the following block, and insert the addresses replacing 
	// the all-0's placeholder.

	forwarders {
	 	8.8.8.8;
	};

	//========================================================================
	// If BIND logs error messages about the root key being expired,
	// you will need to update your keys.  See https://www.isc.org/bind-keys
	//========================================================================
	dnssec-validation auto;

	auth-nxdomain no;    # conform to RFC1035
	listen-on-v6 { any; };
};
\end{lstlisting}

\subsubsection{/etc/dhcp/dhcpd.conf}
\begin{lstlisting}
#
# Sample configuration file for ISC dhcpd for Debian
#
# Attention: If /etc/ltsp/dhcpd.conf exists, that will be used as
# configuration file instead of this file.
#
#

# The ddns-updates-style parameter controls whether or not the server will
# attempt to do a DNS update when a lease is confirmed. We default to the
# behavior of the version 2 packages ('none', since DHCP v2 didn't
# have support for DDNS.)
ddns-update-style none;

# option definitions common to all supported networks...
option domain-name "shib.lan";
option domain-name-servers 192.168.100.100;

default-lease-time 600;
max-lease-time 7200;

# If this DHCP server is the official DHCP server for the local
# network, the authoritative directive should be uncommented.
authoritative;

# Use this to send dhcp log messages to a different log file (you also
# have to hack syslog.conf to complete the redirection).
log-facility local7;

# No service will be given on this subnet, but declaring it helps the 
# DHCP server to understand the network topology.

subnet 192.168.100.0 netmask 255.255.255.0 {
	range 192.168.100.101 192.168.100.200;
	interface eth1;
}

#subnet 10.152.187.0 netmask 255.255.255.0 {
#}

# This is a very basic subnet declaration.

#subnet 10.254.239.0 netmask 255.255.255.224 {
#  range 10.254.239.10 10.254.239.20;
#  option routers rtr-239-0-1.example.org, rtr-239-0-2.example.org;
#}

# This declaration allows BOOTP clients to get dynamic addresses,
# which we don't really recommend.

#subnet 10.254.239.32 netmask 255.255.255.224 {
#  range dynamic-bootp 10.254.239.40 10.254.239.60;
#  option broadcast-address 10.254.239.31;
#  option routers rtr-239-32-1.example.org;
#}

# A slightly different configuration for an internal subnet.
#subnet 10.5.5.0 netmask 255.255.255.224 {
#  range 10.5.5.26 10.5.5.30;
#  option domain-name-servers ns1.internal.example.org;
#  option domain-name "internal.example.org";
#  option routers 10.5.5.1;
#  option broadcast-address 10.5.5.31;
#  default-lease-time 600;
#  max-lease-time 7200;
#}

# Hosts which require special configuration options can be listed in
# host statements.   If no address is specified, the address will be
# allocated dynamically (if possible), but the host-specific information
# will still come from the host declaration.

host sp {
	hardware ethernet 08:00:27:9e:43:d3;
	fixed-address 192.168.100.200;
}

#host passacaglia {
#  hardware ethernet 0:0:c0:5d:bd:95;
#  filename "vmunix.passacaglia";
#  server-name "toccata.fugue.com";
#}

# Fixed IP addresses can also be specified for hosts.   These addresses
# should not also be listed as being available for dynamic assignment.
# Hosts for which fixed IP addresses have been specified can boot using
# BOOTP or DHCP.   Hosts for which no fixed address is specified can only
# be booted with DHCP, unless there is an address range on the subnet
# to which a BOOTP client is connected which has the dynamic-bootp flag
# set.
#host fantasia {
#  hardware ethernet 08:00:07:26:c0:a5;
#  fixed-address fantasia.fugue.com;
#}

# You can declare a class of clients and then do address allocation
# based on that.   The example below shows a case where all clients
# in a certain class get addresses on the 10.17.224/24 subnet, and all
# other clients get addresses on the 10.0.29/24 subnet.

#class "foo" {
#  match if substring (option vendor-class-identifier, 0, 4) = "SUNW";
#}

#shared-network 224-29 {
#  subnet 10.17.224.0 netmask 255.255.255.0 {
#    option routers rtr-224.example.org;
#  }
#  subnet 10.0.29.0 netmask 255.255.255.0 {
#    option routers rtr-29.example.org;
#  }
#  pool {
#    allow members of "foo";
#    range 10.17.224.10 10.17.224.250;
#  }
#  pool {
#    deny members of "foo";
#    range 10.0.29.10 10.0.29.230;
#  }
#}
\end{lstlisting}

\subsubsection{/etc/apache2/ports.conf}
\begin{lstlisting}
# If you just change the port or add more ports here, you will likely also
# have to change the VirtualHost statement in
# /etc/apache2/sites-enabled/000-default.conf

Listen 80

<IfModule ssl_module>
	Listen 443
</IfModule>

<IfModule mod_gnutls.c>
	Listen 443
</IfModule>
\end{lstlisting}

\subsubsection{/etc/apache2/sites-available/www.conf}
\begin{lstlisting}
<VirtualHost *:443>
        ServerAdmin admin@shib.lan
        ServerName www.shib.lan
        ServerAlias www.shib.lan 
        SSLEngine On
        SSLCertificateFile /etc/ssl/certs/www.shib.lan_cert.pem
        SSLCertificateKeyFile /etc/ssl/private/www.shib.lan_key.pem
        DocumentRoot "/var/www/html"
        <Directory "/var/www/html">
                Options FollowSymLinks
                AllowOverride AuthConfig
                Order allow,deny
                Allow from all
        </Directory>
        DirectoryIndex index.html
        ErrorLog ${APACHE_LOG_DIR}/error.log
        CustomLog ${APACHE_LOG_DIR}/access.log combined
</VirtualHost>
\end{lstlisting}

\subsubsection{/etc/apache2/sites-available/redirect.conf}
\begin{lstlisting}
<VirtualHost *:443>
        ServerName shib.lan
        ServerAlias shib.lan 
        SSLEngine On
        SSLCertificateFile /etc/ssl/certs/www.shib.lan_cert.pem
        SSLCertificateKeyFile /etc/ssl/private/www.shib.lan_key.pem
        Redirect 301 / https://www.shib.lan/
        ErrorLog ${APACHE_LOG_DIR}/error.log
        CustomLog ${APACHE_LOG_DIR}/access.log combined
</VirtualHost>

<VirtualHost *:80>
        ServerAdmin admin@shib.lan
        ServerName shib.lan
        ServerAlias shib.lan 
        Redirect 301 / https://www.shib.lan/
        ErrorLog ${APACHE_LOG_DIR}/error.log
        CustomLog ${APACHE_LOG_DIR}/access.log combined
</VirtualHost>
\end{lstlisting}

\subsubsection{/etc/apache2/sites-available/ldap.conf}
\begin{lstlisting}
<VirtualHost *:443>
        ServerAdmin admin@shib.lan
        ServerName ldap.shib.lan
        ServerAlias ldap.shib.lan
        SSLEngine On
        SSLCertificateFile /etc/ssl/certs/ldap.shib.lan_cert.pem
        SSLCertificateKeyFile /etc/ssl/private/ldap.shib.lan_key.pem
        DocumentRoot "/usr/share/phpldapadmin/htdocs"
        <Directory "/usr/share/phpldapadmin">
                Options FollowSymLinks
                AllowOverride AuthConfig
                Order allow,deny
                Allow from all
        </Directory>
        DirectoryIndex index.php
        ErrorLog "/var/log/apache2/phpldapadmin_error.log"
        CustomLog "/var/log/apache2/phpldapadmin_access.log" combined
</VirtualHost>

<VirtualHost *:80>
        ServerAdmin admin@shib.lan
        ServerName ldap.shib.lan
        ServerAlias ldap.shib.lan 
        Redirect 301 / https://ldap.shib.lan/
        ErrorLog ${APACHE_LOG_DIR}/error.log
        CustomLog ${APACHE_LOG_DIR}/access.log combined
</VirtualHost>
\end{lstlisting}

\subsubsection{/etc/apache2/sites-available/idp.conf}
\begin{lstlisting}
<VirtualHost *:443>
  ServerName              idp.shib.lan
  SSLEngine               on
  SSLCertificateFile      /etc/ssl/certs/idp.shib.lan_cert.pem
  SSLCertificateKeyFile   /etc/ssl/private/idp.shib.lan_key.pem
 
  SSLProtocol All -SSLv2 -SSLv3
  SSLHonorCipherOrder On
  SSLCipherSuite 'ECDH+AESGCM:DH+AESGCM:ECDH+AES256:DH+AES256:ECDH+AES128:DH+AES:RSA+AESGCM:RSA+AES:ECDH+3DES:DH+3DES:RSA+3DES:!aNULL:!eNULL:!LOW:!RC4:!MD5:!EXP:!PSK:!DSS:!SEED:!ECDSA:!CAMELLIA'
 
  <Location /idp>
    Allow from all
    ProxyPass ajp://localhost:8009/idp
    #verhindern, dass die Login Seite in iFrames eingebunden wird
    Header always append X-FRAME-OPTIONS "DENY"
  </Location>
</VirtualHost>
 
Listen 8443
 
<VirtualHost *:8443>
  ServerName              idp.hib.lan
  SSLEngine               on
  SSLCertificateFile      /opt/shibboleth-idp/credentials/idp.crt
  SSLCertificateKeyFile   /opt/shibboleth-idp/credentials/idp.key

  SSLProtocol All -SSLv2 -SSLv3
  SSLHonorCipherOrder On
  SSLCipherSuite 'ECDH+AESGCM:DH+AESGCM:ECDH+AES256:DH+AES256:ECDH+AES128:DH+AES:RSA+AESGCM:RSA+AES:ECDH+3DES:DH+3DES:RSA+3DES:!aNULL:!eNULL:!LOW:!RC4:!MD5:!EXP:!PSK:!DSS:!SEED:!ECDSA:!CAMELLIA'
 
  SSLVerifyClient optional_no_ca
  # damit auch Zertifikate mit einer längeren CA-Kette funktionieren:
  SSLVerifyDepth  10
  # damit Apache das Client-Zertifkat an Tomcat weiterleitet:
  SSLOptions      +StdEnvVars +ExportCertData
 
  <Location /idp>
    Allow from all
    ProxyPass ajp://localhost:8009/idp
  </Location>
 
</VirtualHost>
\end{lstlisting}
\subsubsection{/etc/ldap/tls.ldif}
\begin{lstlisting}
###########################################################
# CONFIGURATION for Support of TLS
###########################################################
# Add TLS supported access to user passwords for LDAP clients
# to the LDAP config. 

dn: cn=config
changetype: modify
add: olcTLSCACertificateFile
olcTLSCACertificateFile: /etc/ssl/certs/ldap.shib.lan_cert.pem

dn: cn=config
changetype: modify
add: olcTLSCertificateKeyFile
olcTLSCertificateKeyFile: /etc/ssl/private/ldap.shib.lan_key.pem

#dn: cn=config
#changetype: modify
#delete: olcTLSCertificateFile

dn: cn=config
changetype: modify
add: olcTLSCertificateFile
olcTLSCertificateFile: /etc/ssl/certs/ldap.shib.lan_cert.pem
\end{lstlisting}

\subsubsection{/home/shib-idp/LDAP/schema/schema\_convert.conf}
\begin{lstlisting}
include /etc/ldap/schema/core.schema
include /etc/ldap/schema/collective.schema
include /etc/ldap/schema/corba.schema
include /etc/ldap/schema/cosine.schema
include /etc/ldap/schema/duaconf.schema
include /etc/ldap/schema/dyngroup.schema
include /etc/ldap/schema/inetorgperson.schema
include /etc/ldap/schema/java.schema
include /etc/ldap/schema/misc.schema
include /etc/ldap/schema/nis.schema
include /etc/ldap/schema/openldap.schema
include /etc/ldap/schema/ppolicy.schema
include /etc/ldap/schema/ldapns.schema
include /etc/ldap/schema/schac-1.5.0.schema
include /etc/ldap/schema/dfneduperson-1.0.schema
\end{lstlisting}

\subsubsection{/etc/default/tomcat7}
\begin{lstlisting}
# Run Tomcat as this user ID. Not setting this or leaving it blank will use the
# default of tomcat7.
TOMCAT7_USER=tomcat7

# Run Tomcat as this group ID. Not setting this or leaving it blank will use
# the default of tomcat7.
TOMCAT7_GROUP=tomcat7

# The home directory of the Java development kit (JDK). You need at least
# JDK version 1.5. If JAVA_HOME is not set, some common directories for 
# OpenJDK, the Sun JDK, and various J2SE 1.5 versions are tried.
#JAVA_HOME=/usr/lib/jvm/openjdk-6-jdk

# You may pass JVM startup parameters to Java here. If unset, the default
# options will be: -Djava.awt.headless=true -Xmx128m -XX:+UseConcMarkSweepGC
#
# Use "-XX:+UseConcMarkSweepGC" to enable the CMS garbage collector (improved
# response time). If you use that option and you run Tomcat on a machine with
# exactly one CPU chip that contains one or two cores, you should also add
# the "-XX:+CMSIncrementalMode" option.
JAVA_OPTS="-Djava.awt.headless=true -Xmx512m -XX:+UseConcMarkSweepGC"

# To enable remote debugging uncomment the following line.
# You will then be able to use a java debugger on port 8000.
#JAVA_OPTS="${JAVA_OPTS} -Xdebug -Xrunjdwp:transport=dt_socket,address=8000,server=y,suspend=n"

# Java compiler to use for translating JavaServer Pages (JSPs). You can use all
# compilers that are accepted by Ant's build.compiler property.
#JSP_COMPILER=javac

# Use the Java security manager? (yes/no, default: no)
TOMCAT7_SECURITY=no

# Number of days to keep logfiles in /var/log/tomcat7. Default is 14 days.
#LOGFILE_DAYS=14
# Whether to compress logfiles older than today's
#LOGFILE_COMPRESS=1

# Location of the JVM temporary directory
# WARNING: This directory will be destroyed and recreated at every startup !
#JVM_TMP=/tmp/tomcat7-temp

# If you run Tomcat on port numbers that are all higher than 1023, then you
# do not need authbind.  It is used for binding Tomcat to lower port numbers.
# NOTE: authbind works only with IPv4.  Do not enable it when using IPv6.
# (yes/no, default: no)
#AUTHBIND=no
\end{lstlisting}

\subsubsection{/etc/tomcat7/server.xml}
\begin{lstlisting}[language=xml]
<?xml version='1.0' encoding='utf-8'?>
<!--
  Licensed to the Apache Software Foundation (ASF) under one or more
  contributor license agreements.  See the NOTICE file distributed with
  this work for additional information regarding copyright ownership.
  The ASF licenses this file to You under the Apache License, Version 2.0
  (the "License"); you may not use this file except in compliance with
  the License.  You may obtain a copy of the License at

      http://www.apache.org/licenses/LICENSE-2.0

  Unless required by applicable law or agreed to in writing, software
  distributed under the License is distributed on an "AS IS" BASIS,
  WITHOUT WARRANTIES OR CONDITIONS OF ANY KIND, either express or implied.
  See the License for the specific language governing permissions and
  limitations under the License.
-->
<!-- Note:  A "Server" is not itself a "Container", so you may not
     define subcomponents such as "Valves" at this level.
     Documentation at /docs/config/server.html
 -->
<Server port="8005" shutdown="SHUTDOWN">
  <!-- Security listener. Documentation at /docs/config/listeners.html
  <Listener className="org.apache.catalina.security.SecurityListener" />
  -->
  <!--APR library loader. Documentation at /docs/apr.html -->
  <!--
  <Listener className="org.apache.catalina.core.AprLifecycleListener" SSLEngine="on" />
  -->
  <!--Initialize Jasper prior to webapps are loaded. Documentation at /docs/jasper-howto.html -->
  <Listener className="org.apache.catalina.core.JasperListener" />
  <!-- Prevent memory leaks due to use of particular java/javax APIs-->
  <Listener className="org.apache.catalina.core.JreMemoryLeakPreventionListener" />
  <Listener className="org.apache.catalina.mbeans.GlobalResourcesLifecycleListener" />
  <Listener className="org.apache.catalina.core.ThreadLocalLeakPreventionListener" />

  <!-- Global JNDI resources
       Documentation at /docs/jndi-resources-howto.html
  -->
  <GlobalNamingResources>
    <!-- Editable user database that can also be used by
         UserDatabaseRealm to authenticate users
    -->
    <Resource name="UserDatabase" auth="Container"
              type="org.apache.catalina.UserDatabase"
              description="User database that can be updated and saved"
              factory="org.apache.catalina.users.MemoryUserDatabaseFactory"
              pathname="conf/tomcat-users.xml" />
  </GlobalNamingResources>

  <!-- A "Service" is a collection of one or more "Connectors" that share
       a single "Container" Note:  A "Service" is not itself a "Container",
       so you may not define subcomponents such as "Valves" at this level.
       Documentation at /docs/config/service.html
   -->
  <Service name="Catalina">

      <!-- Define an AJP 1.3 Connector on port 8009 -->
      <Connector port="8009" address="127.0.0.1"
              enableLookups="false" 
              redirectPort="8443"
              protocol="AJP/1.3"
              maxPostSize="100000" />

     <!-- hinter einem Load-Balancer kann das z.B. so aussehen:
     <Connector port="8009" address="127.0.0.1" 
              scheme="https"
              enableLookups="false"
              redirectPort="8443" 
              proxyPort="443" 
              protocol="AJP/1.3"
              maxPostSize="100000" />
     -->


    <!--The connectors can use a shared executor, you can define one or more named thread pools-->
    <!--
    <Executor name="tomcatThreadPool" namePrefix="catalina-exec-"
        maxThreads="150" minSpareThreads="4"/>
    -->


    <!-- A "Connector" represents an endpoint by which requests are received
         and responses are returned. Documentation at :
         Java HTTP Connector: /docs/config/http.html (blocking & non-blocking)
         Java AJP  Connector: /docs/config/ajp.html
         APR (HTTP/AJP) Connector: /docs/apr.html
         Define a non-SSL HTTP/1.1 Connector on port 8080
    -->
    <Connector port="8080" protocol="HTTP/1.1"
               connectionTimeout="20000"
               URIEncoding="UTF-8"
               redirectPort="8443" />
    <!-- A "Connector" using the shared thread pool-->
    <!--
    <Connector executor="tomcatThreadPool"
               port="8080" protocol="HTTP/1.1"
               connectionTimeout="20000"
               redirectPort="8443" />
    -->
    <!-- Define a SSL HTTP/1.1 Connector on port 8443
         This connector uses the JSSE configuration, when using APR, the
         connector should be using the OpenSSL style configuration
         described in the APR documentation -->
    <!--
    <Connector port="8443" protocol="HTTP/1.1" SSLEnabled="true"
               maxThreads="150" scheme="https" secure="true"
               clientAuth="false" sslProtocol="TLS" />
    -->

    <!-- Define an AJP 1.3 Connector on port 8009 -->
    <!--
    <Connector port="8009" protocol="AJP/1.3" redirectPort="8443" />
    -->


    <!-- An Engine represents the entry point (within Catalina) that processes
         every request.  The Engine implementation for Tomcat stand alone
         analyzes the HTTP headers included with the request, and passes them
         on to the appropriate Host (virtual host).
         Documentation at /docs/config/engine.html -->

    <!-- You should set jvmRoute to support load-balancing via AJP ie :
    <Engine name="Catalina" defaultHost="localhost" jvmRoute="jvm1">
    -->
    <Engine name="Catalina" defaultHost="localhost">

      <!--For clustering, please take a look at documentation at:
          /docs/cluster-howto.html  (simple how to)
          /docs/config/cluster.html (reference documentation) -->
      <!--
      <Cluster className="org.apache.catalina.ha.tcp.SimpleTcpCluster"/>
      -->

      <!-- Use the LockOutRealm to prevent attempts to guess user passwords
           via a brute-force attack -->
      <Realm className="org.apache.catalina.realm.LockOutRealm">
        <!-- This Realm uses the UserDatabase configured in the global JNDI
             resources under the key "UserDatabase".  Any edits
             that are performed against this UserDatabase are immediately
             available for use by the Realm.  -->
        <Realm className="org.apache.catalina.realm.UserDatabaseRealm"
               resourceName="UserDatabase"/>
      </Realm>

      <Host name="localhost"  appBase="webapps"
            unpackWARs="true" autoDeploy="true">

        <!-- SingleSignOn valve, share authentication between web applications
             Documentation at: /docs/config/valve.html -->
        <!--
        <Valve className="org.apache.catalina.authenticator.SingleSignOn" />
        -->

        <!-- Access log processes all example.
             Documentation at: /docs/config/valve.html
             Note: The pattern used is equivalent to using pattern="common" -->
        <Valve className="org.apache.catalina.valves.AccessLogValve" directory="logs"
               prefix="localhost_access_log." suffix=".txt"
               pattern="%h %l %u %t &quot;%r&quot; %s %b" />

      </Host>
    </Engine>
  </Service>
</Server>
\end{lstlisting}

\subsubsection{/etc/tomcat7/Catalina/localhost/idp.xml}
\begin{lstlisting}[language=xml]
<Context docBase="/opt/shibboleth-idp/war/idp.war"
         privileged="true"
         antiResourceLocking="false"
         antiJARLocking="false"
         unpackWAR="false"
         swallowOutput="true" />
\end{lstlisting}

\subsubsection{/opt/shibboleth-idp/conf/service.xml}
\begin{lstlisting}[language=xml]
<?xml version="1.0" encoding="UTF-8"?>
<srv:Services xmlns:srv="urn:mace:shibboleth:2.0:services" xmlns:attribute-afp="urn:mace:shibboleth:2.0:afp" 
              xmlns:attribute-authority="urn:mace:shibboleth:2.0:attribute:authority" xmlns:attribute-resolver="urn:mace:shibboleth:2.0:resolver" 
              xmlns:profile="urn:mace:shibboleth:2.0:idp:profile-handler" xmlns:relyingParty="urn:mace:shibboleth:2.0:relying-party" 
              xmlns:resource="urn:mace:shibboleth:2.0:resource" xmlns:xsi="http://www.w3.org/2001/XMLSchema-instance" 
              xsi:schemaLocation="urn:mace:shibboleth:2.0:services classpath:/schema/shibboleth-2.0-services.xsd
                                  urn:mace:shibboleth:2.0:afp classpath:/schema/shibboleth-2.0-afp.xsd
                                  urn:mace:shibboleth:2.0:attribute:authority classpath:/schema/shibboleth-2.0-attribute-authority.xsd
                                  urn:mace:shibboleth:2.0:resolver classpath:/schema/shibboleth-2.0-attribute-resolver.xsd
                                  urn:mace:shibboleth:2.0:idp:profile-handler classpath:/schema/shibboleth-2.0-idp-profile-handler.xsd
                                  urn:mace:shibboleth:2.0:relying-party classpath:/schema/shibboleth-2.0-relying-party.xsd
                                  urn:mace:shibboleth:2.0:resource classpath:/schema/shibboleth-2.0-resource.xsd">

    <srv:Service id="shibboleth.AttributeResolver" xsi:type="attribute-resolver:ShibbolethAttributeResolver" configurationResourcePollingFrequency="PT5M">
        <srv:ConfigurationResource file="/opt/shibboleth-idp/conf/attribute-resolver.xml" xsi:type="resource:FilesystemResource"/>
    </srv:Service>

    <srv:Service id="shibboleth.AttributeFilterEngine" xsi:type="attribute-afp:ShibbolethAttributeFilteringEngine" configurationResourcePollingFrequency="PT5M">
        <srv:ConfigurationResource file="/opt/shibboleth-idp/conf/attribute-filter.xml" xsi:type="resource:FilesystemResource"/>
    </srv:Service>
    
    <srv:Service id="shibboleth.SAML1AttributeAuthority" xsi:type="attribute-authority:SAML1AttributeAuthority" 
                 depends-on="shibboleth.AttributeResolver shibboleth.AttributeFilterEngine" 
                 resolver="shibboleth.AttributeResolver" filter="shibboleth.AttributeFilterEngine"/>
             
    <srv:Service id="shibboleth.SAML2AttributeAuthority" xsi:type="attribute-authority:SAML2AttributeAuthority" 
                 depends-on="shibboleth.AttributeResolver shibboleth.AttributeFilterEngine" 
                 resolver="shibboleth.AttributeResolver" filter="shibboleth.AttributeFilterEngine"/>

    <srv:Service id="shibboleth.RelyingPartyConfigurationManager" xsi:type="relyingParty:SAMLMDRelyingPartyConfigurationManager" 
                 depends-on="shibboleth.SAML1AttributeAuthority shibboleth.SAML2AttributeAuthority">
        <srv:ConfigurationResource file="/opt/shibboleth-idp/conf/relying-party.xml" xsi:type="resource:FilesystemResource"/>
    </srv:Service>

    <srv:Service id="shibboleth.HandlerManager" depends-on="shibboleth.RelyingPartyConfigurationManager" xsi:type="profile:IdPProfileHandlerManager">
        <srv:ConfigurationResource file="/opt/shibboleth-idp/conf/handler.xml" xsi:type="resource:FilesystemResource"/>
    </srv:Service>
    
    <!-- 
        A special service that exports all services upon which it depends into the ServletContext as an attribute 
        with the same name as the service's ID.
    -->
    <srv:Service id="shibboleth.ServiceServletContextAttributeExporter" xsi:type="srv:ServletContextAttributeExporter"
                 depends-on="shibboleth.AttributeResolver shibboleth.AttributeFilterEngine 
                             shibboleth.SAML1AttributeAuthority shibboleth.SAML2AttributeAuthority
                             shibboleth.RelyingPartyConfigurationManager shibboleth.HandlerManager 
                             shibboleth.StorageService" />
</srv:Services>
\end{lstlisting}

\subsubsection{/opt/shibboleth-idp/conf/relying-party.xml}
\begin{lstlisting}[language=xml]
<?xml version="1.0" encoding="UTF-8"?>
<!--
    This file is an EXAMPLE configuration file.

    This file specifies relying party dependent configurations for the IdP, for example, whether SAML assertions to a 
    particular relying party should be signed.  It also includes metadata provider and credential definitions used 
    when answering requests to a relying party.
-->
<rp:RelyingPartyGroup xmlns:rp="urn:mace:shibboleth:2.0:relying-party" xmlns:saml="urn:mace:shibboleth:2.0:relying-party:saml" 
                      xmlns:metadata="urn:mace:shibboleth:2.0:metadata" xmlns:resource="urn:mace:shibboleth:2.0:resource" 
                      xmlns:security="urn:mace:shibboleth:2.0:security" xmlns:samlsec="urn:mace:shibboleth:2.0:security:saml" 
                      xmlns:samlmd="urn:oasis:names:tc:SAML:2.0:metadata" xmlns:xsi="http://www.w3.org/2001/XMLSchema-instance" 
                      xsi:schemaLocation="urn:mace:shibboleth:2.0:relying-party classpath:/schema/shibboleth-2.0-relying-party.xsd
                                          urn:mace:shibboleth:2.0:relying-party:saml classpath:/schema/shibboleth-2.0-relying-party-saml.xsd
                                          urn:mace:shibboleth:2.0:metadata classpath:/schema/shibboleth-2.0-metadata.xsd
                                          urn:mace:shibboleth:2.0:resource classpath:/schema/shibboleth-2.0-resource.xsd 
                                          urn:mace:shibboleth:2.0:security classpath:/schema/shibboleth-2.0-security.xsd
                                          urn:mace:shibboleth:2.0:security:saml classpath:/schema/shibboleth-2.0-security-policy-saml.xsd
                                          urn:oasis:names:tc:SAML:2.0:metadata classpath:/schema/saml-schema-metadata-2.0.xsd">
                                       
    <!-- ========================================== -->
    <!--      Relying Party Configurations          -->
    <!-- ========================================== -->
    <rp:AnonymousRelyingParty provider="https://idp.shib.lan/idp/shibboleth" defaultSigningCredentialRef="IdPCredential"/>
    
    <rp:DefaultRelyingParty provider="https://idp.shib.lan/idp/shibboleth" defaultSigningCredentialRef="IdPCredential">
        <!-- 
            Each attribute in these profiles configuration is set to its default value,
            that is, the values that would be in effect if those attributes were not present.
            We list them here so that people are aware of them (since they seem reluctant to 
            read the documentation).
        -->
        <rp:ProfileConfiguration xsi:type="saml:ShibbolethSSOProfile" includeAttributeStatement="false" 
                                 assertionLifetime="PT5M" signResponses="conditional" signAssertions="never"
                                 includeConditionsNotBefore="true"/>
                              
        <rp:ProfileConfiguration xsi:type="saml:SAML1AttributeQueryProfile" assertionLifetime="PT5M" 
                                 signResponses="conditional" signAssertions="never"
                                 includeConditionsNotBefore="true"/>
        
        <rp:ProfileConfiguration xsi:type="saml:SAML1ArtifactResolutionProfile" signResponses="conditional" 
                                 signAssertions="never"/>
        
        <rp:ProfileConfiguration xsi:type="saml:SAML2SSOProfile" includeAttributeStatement="true" 
                                 assertionLifetime="PT5M" assertionProxyCount="0" 
                                 signResponses="never" signAssertions="always" 
                                 encryptAssertions="conditional" encryptNameIds="never"
                                 includeConditionsNotBefore="true"/>

        <rp:ProfileConfiguration xsi:type="saml:SAML2ECPProfile" includeAttributeStatement="true" 
                                 assertionLifetime="PT5M" assertionProxyCount="0" 
                                 signResponses="never" signAssertions="always" 
                                 encryptAssertions="conditional" encryptNameIds="never"
                                 includeConditionsNotBefore="true"/>

        <rp:ProfileConfiguration xsi:type="saml:SAML2AttributeQueryProfile" 
                                 assertionLifetime="PT5M" assertionProxyCount="0" 
                                 signResponses="conditional" signAssertions="never" 
                                 encryptAssertions="conditional" encryptNameIds="never"
                                 includeConditionsNotBefore="true"/>
        
        <rp:ProfileConfiguration xsi:type="saml:SAML2ArtifactResolutionProfile" 
                                 signResponses="never" signAssertions="always" 
                                 encryptAssertions="conditional" encryptNameIds="never"/>

        <rp:ProfileConfiguration xsi:type="saml:SAML2LogoutRequestProfile"
                                 signResponses="conditional"/>
        
    </rp:DefaultRelyingParty>
        
    
    <!-- ========================================== -->
    <!--      Metadata Configuration                -->
    <!-- ========================================== -->
    <!-- MetadataProvider the combining other MetadataProviders -->
    <metadata:MetadataProvider id="ShibbolethMetadata" xsi:type="metadata:ChainingMetadataProvider">
    
    	<!-- Load the IdP's own metadata.  This is necessary for artifact support. -->
        <metadata:MetadataProvider id="IdPMD" xsi:type="metadata:FilesystemMetadataProvider"
                                   metadataFile="/opt/shibboleth-idp/metadata/idp-metadata.xml"
                                   maxRefreshDelay="P1D" />
        
        <metadata:MetadataProvider id="sp.shib.lan" xsi:type="metadata:FileBackedHTTPMetadataProvider"
                      metadataURL="https://sp.shib.lan/Shibboleth.sso/Metadata"
                      backingFile="/opt/shibboleth-idp/metadata/sp-metadata.xml" />


        <!-- Example metadata provider. -->
        <!-- Reads metadata from a URL and store a backup copy on the file system. -->
        <!-- Validates the signature of the metadata and filters out all by SP entities in order to save memory -->
        <!-- To use: fill in 'metadataURL' and 'backingFile' properties on MetadataResource element -->
        <!--
        <metadata:MetadataProvider id="URLMD" xsi:type="metadata:FileBackedHTTPMetadataProvider"
                          metadataURL="http://example.org/metadata.xml"
                          backingFile="/opt/shibboleth-idp/metadata/some-metadata.xml">
            <metadata:MetadataFilter xsi:type="metadata:ChainingFilter">
                <metadata:MetadataFilter xsi:type="metadata:RequiredValidUntil" 
                                maxValidityInterval="P7D" />
                <metadata:MetadataFilter xsi:type="metadata:SignatureValidation"
                                trustEngineRef="shibboleth.MetadataTrustEngine"
                                requireSignedMetadata="true" />
	            <metadata:MetadataFilter xsi:type="metadata:EntityRoleWhiteList">
                    <metadata:RetainedRole>samlmd:SPSSODescriptor</metadata:RetainedRole>
                </metadata:MetadataFilter>
            </metadata:MetadataFilter>
        </metadata:MetadataProvider>
        -->
        
    </metadata:MetadataProvider>

    
    <!-- ========================================== -->
    <!--     Security Configurations                -->
    <!-- ========================================== -->
    <security:Credential id="IdPCredential" xsi:type="security:X509Filesystem">
        <security:PrivateKey>/opt/shibboleth-idp/credentials/idp.key</security:PrivateKey>
        <security:Certificate>/opt/shibboleth-idp/credentials/idp.crt</security:Certificate>
    </security:Credential>
    
    <!-- Trust engine used to evaluate the signature on loaded metadata. -->
    <!--
    <security:TrustEngine id="shibboleth.MetadataTrustEngine" xsi:type="security:StaticExplicitKeySignature">
        <security:Credential id="MyFederation1Credentials" xsi:type="security:X509Filesystem">
            <security:Certificate>/opt/shibboleth-idp/credentials/federation1.crt</security:Certificate>
        </security:Credential>
    </security:TrustEngine>
     -->
     
    <!-- DO NOT EDIT BELOW THIS POINT -->
    <!-- 
        The following trust engines and rules control every aspect of security related to incoming messages. 
        Trust engines evaluate various tokens (like digital signatures) for trust worthiness while the 
        security policies establish a set of checks that an incoming message must pass in order to be considered
        secure.  Naturally some of these checks require the validation of the tokens evaluated by the trust 
        engines and so you'll see some rules that reference the declared trust engines.
    -->
    <security:TrustEngine id="shibboleth.SignatureTrustEngine" xsi:type="security:SignatureChaining">
        <security:TrustEngine id="shibboleth.SignatureMetadataExplicitKeyTrustEngine" xsi:type="security:MetadataExplicitKeySignature" metadataProviderRef="ShibbolethMetadata"/>                              
        <security:TrustEngine id="shibboleth.SignatureMetadataPKIXTrustEngine" xsi:type="security:MetadataPKIXSignature" metadataProviderRef="ShibbolethMetadata"/>
    </security:TrustEngine>
    
    <security:TrustEngine id="shibboleth.CredentialTrustEngine" xsi:type="security:Chaining">
        <security:TrustEngine id="shibboleth.CredentialMetadataExplictKeyTrustEngine" xsi:type="security:MetadataExplicitKey" metadataProviderRef="ShibbolethMetadata"/>
        <security:TrustEngine id="shibboleth.CredentialMetadataPKIXTrustEngine" xsi:type="security:MetadataPKIXX509Credential" metadataProviderRef="ShibbolethMetadata"/>
    </security:TrustEngine>
     
    <security:SecurityPolicy id="shibboleth.ShibbolethSSOSecurityPolicy" xsi:type="security:SecurityPolicyType">
        <security:Rule xsi:type="samlsec:Replay" required="false"/>
        <security:Rule xsi:type="samlsec:IssueInstant" required="false"/>
        <security:Rule xsi:type="samlsec:MandatoryIssuer"/>
    </security:SecurityPolicy>
    
    <security:SecurityPolicy id="shibboleth.SAML1AttributeQuerySecurityPolicy" xsi:type="security:SecurityPolicyType">
        <security:Rule xsi:type="samlsec:Replay"/>
        <security:Rule xsi:type="samlsec:IssueInstant"/>
        <security:Rule xsi:type="samlsec:ProtocolWithXMLSignature" trustEngineRef="shibboleth.SignatureTrustEngine"/>
        <security:Rule xsi:type="security:ClientCertAuth" trustEngineRef="shibboleth.CredentialTrustEngine"/>
        <security:Rule xsi:type="samlsec:MandatoryIssuer"/>
        <security:Rule xsi:type="security:MandatoryMessageAuthentication"/>
    </security:SecurityPolicy>
    
    <security:SecurityPolicy id="shibboleth.SAML1ArtifactResolutionSecurityPolicy" xsi:type="security:SecurityPolicyType">
        <security:Rule xsi:type="samlsec:Replay"/>
        <security:Rule xsi:type="samlsec:IssueInstant"/>
        <security:Rule xsi:type="samlsec:ProtocolWithXMLSignature" trustEngineRef="shibboleth.SignatureTrustEngine"/>
        <security:Rule xsi:type="security:ClientCertAuth" trustEngineRef="shibboleth.CredentialTrustEngine"/>
        <security:Rule xsi:type="samlsec:MandatoryIssuer"/>
        <security:Rule xsi:type="security:MandatoryMessageAuthentication"/>
    </security:SecurityPolicy>

    <security:SecurityPolicy id="shibboleth.SAML2SSOSecurityPolicy" xsi:type="security:SecurityPolicyType">
        <security:Rule xsi:type="samlsec:Replay"/>
        <security:Rule xsi:type="samlsec:IssueInstant"/>
        <security:Rule xsi:type="samlsec:SAML2AuthnRequestsSigned"/>
        <security:Rule xsi:type="samlsec:ProtocolWithXMLSignature" trustEngineRef="shibboleth.SignatureTrustEngine"/>
        <security:Rule xsi:type="samlsec:SAML2HTTPRedirectSimpleSign" trustEngineRef="shibboleth.SignatureTrustEngine"/>
        <security:Rule xsi:type="samlsec:SAML2HTTPPostSimpleSign" trustEngineRef="shibboleth.SignatureTrustEngine"/>
        <security:Rule xsi:type="samlsec:MandatoryIssuer"/>
    </security:SecurityPolicy>

    <security:SecurityPolicy id="shibboleth.SAML2AttributeQuerySecurityPolicy" xsi:type="security:SecurityPolicyType">
        <security:Rule xsi:type="samlsec:Replay"/>
        <security:Rule xsi:type="samlsec:IssueInstant"/>
        <security:Rule xsi:type="samlsec:ProtocolWithXMLSignature" trustEngineRef="shibboleth.SignatureTrustEngine"/>
        <security:Rule xsi:type="samlsec:SAML2HTTPRedirectSimpleSign" trustEngineRef="shibboleth.SignatureTrustEngine"/>
        <security:Rule xsi:type="samlsec:SAML2HTTPPostSimpleSign" trustEngineRef="shibboleth.SignatureTrustEngine"/>
        <security:Rule xsi:type="security:ClientCertAuth" trustEngineRef="shibboleth.CredentialTrustEngine"/>
        <security:Rule xsi:type="samlsec:MandatoryIssuer"/>
        <security:Rule xsi:type="security:MandatoryMessageAuthentication"/>
    </security:SecurityPolicy>
    
    <security:SecurityPolicy id="shibboleth.SAML2ArtifactResolutionSecurityPolicy" xsi:type="security:SecurityPolicyType">
        <security:Rule xsi:type="samlsec:Replay"/>
        <security:Rule xsi:type="samlsec:IssueInstant"/>
        <security:Rule xsi:type="samlsec:ProtocolWithXMLSignature" trustEngineRef="shibboleth.SignatureTrustEngine"/>
        <security:Rule xsi:type="samlsec:SAML2HTTPRedirectSimpleSign" trustEngineRef="shibboleth.SignatureTrustEngine"/>
        <security:Rule xsi:type="samlsec:SAML2HTTPPostSimpleSign" trustEngineRef="shibboleth.SignatureTrustEngine"/>
        <security:Rule xsi:type="security:ClientCertAuth" trustEngineRef="shibboleth.CredentialTrustEngine"/>
        <security:Rule xsi:type="samlsec:MandatoryIssuer"/>
        <security:Rule xsi:type="security:MandatoryMessageAuthentication"/>
    </security:SecurityPolicy>
    
    <security:SecurityPolicy id="shibboleth.SAML2SLOSecurityPolicy" xsi:type="security:SecurityPolicyType">
        <security:Rule xsi:type="samlsec:Replay"/>
        <security:Rule xsi:type="samlsec:IssueInstant"/>
        <security:Rule xsi:type="samlsec:ProtocolWithXMLSignature" trustEngineRef="shibboleth.SignatureTrustEngine"/>
        <security:Rule xsi:type="samlsec:SAML2HTTPRedirectSimpleSign" trustEngineRef="shibboleth.SignatureTrustEngine"/>
        <security:Rule xsi:type="samlsec:SAML2HTTPPostSimpleSign" trustEngineRef="shibboleth.SignatureTrustEngine"/>
        <security:Rule xsi:type="security:ClientCertAuth" trustEngineRef="shibboleth.CredentialTrustEngine"/>
        <security:Rule xsi:type="samlsec:MandatoryIssuer"/>
        <security:Rule xsi:type="security:MandatoryMessageAuthentication"/>
    </security:SecurityPolicy>
    
</rp:RelyingPartyGroup>
\end{lstlisting}

\subsubsection{/opt/shibboleth-idp/conf/handler.xml}
\begin{lstlisting}[language=xml]
<?xml version="1.0" encoding="UTF-8"?>

<ph:ProfileHandlerGroup xmlns:ph="urn:mace:shibboleth:2.0:idp:profile-handler" xmlns:xsi="http://www.w3.org/2001/XMLSchema-instance" 
                        xsi:schemaLocation="urn:mace:shibboleth:2.0:idp:profile-handler classpath:/schema/shibboleth-2.0-idp-profile-handler.xsd">

    <!-- Error Handler -->
    <ph:ErrorHandler xsi:type="ph:JSPErrorHandler" jspPagePath="/error.jsp"/>

    <!-- Profile Handlers -->
    <!-- 
        All profile handlers defined below are accessed via the Servlet path "/profile" so if your profile 
        handler's request path is "/Status" then the full path is "<servletContextName>/profile/Status"
     -->
    <ph:ProfileHandler xsi:type="ph:Status">
        <ph:RequestPath>/Status</ph:RequestPath>
    </ph:ProfileHandler>
    
    <ph:ProfileHandler xsi:type="ph:SAMLMetadata" metadataFile="/opt/shibboleth-idp/metadata/idp-metadata.xml">
        <ph:RequestPath>/Metadata/SAML</ph:RequestPath>
    </ph:ProfileHandler>    

    <ph:ProfileHandler xsi:type="ph:ShibbolethSSO" inboundBinding="urn:mace:shibboleth:1.0:profiles:AuthnRequest" 
                       outboundBindingEnumeration="urn:oasis:names:tc:SAML:1.0:profiles:browser-post
                                                   urn:oasis:names:tc:SAML:1.0:profiles:artifact-01">
        <ph:RequestPath>/Shibboleth/SSO</ph:RequestPath>
    </ph:ProfileHandler>
    
    <ph:ProfileHandler xsi:type="ph:SAML1AttributeQuery" inboundBinding="urn:oasis:names:tc:SAML:1.0:bindings:SOAP-binding"
                       outboundBindingEnumeration="urn:oasis:names:tc:SAML:1.0:bindings:SOAP-binding">
        <ph:RequestPath>/SAML1/SOAP/AttributeQuery</ph:RequestPath>
    </ph:ProfileHandler>
    
    <ph:ProfileHandler xsi:type="ph:SAML1ArtifactResolution" inboundBinding="urn:oasis:names:tc:SAML:1.0:bindings:SOAP-binding" 
                       outboundBindingEnumeration="urn:oasis:names:tc:SAML:1.0:bindings:SOAP-binding">
        <ph:RequestPath>/SAML1/SOAP/ArtifactResolution</ph:RequestPath>
    </ph:ProfileHandler>
    
    <ph:ProfileHandler xsi:type="ph:SAML2SSO" inboundBinding="urn:oasis:names:tc:SAML:2.0:bindings:HTTP-POST"
                       outboundBindingEnumeration="urn:oasis:names:tc:SAML:2.0:bindings:HTTP-POST-SimpleSign
                                                   urn:oasis:names:tc:SAML:2.0:bindings:HTTP-POST
                                                   urn:oasis:names:tc:SAML:2.0:bindings:HTTP-Artifact">
        <ph:RequestPath>/SAML2/POST/SSO</ph:RequestPath>
    </ph:ProfileHandler>

    <ph:ProfileHandler xsi:type="ph:SAML2SSO" inboundBinding="urn:oasis:names:tc:SAML:2.0:bindings:HTTP-POST-SimpleSign" 
                       outboundBindingEnumeration="urn:oasis:names:tc:SAML:2.0:bindings:HTTP-POST-SimpleSign
                                                   urn:oasis:names:tc:SAML:2.0:bindings:HTTP-POST
                                                   urn:oasis:names:tc:SAML:2.0:bindings:HTTP-Artifact">
        <ph:RequestPath>/SAML2/POST-SimpleSign/SSO</ph:RequestPath>
    </ph:ProfileHandler>

    <ph:ProfileHandler xsi:type="ph:SAML2SSO" inboundBinding="urn:oasis:names:tc:SAML:2.0:bindings:HTTP-Redirect"
                       outboundBindingEnumeration="urn:oasis:names:tc:SAML:2.0:bindings:HTTP-POST-SimpleSign
                                                   urn:oasis:names:tc:SAML:2.0:bindings:HTTP-POST
                                                   urn:oasis:names:tc:SAML:2.0:bindings:HTTP-Artifact">
        <ph:RequestPath>/SAML2/Redirect/SSO</ph:RequestPath>
    </ph:ProfileHandler>

    <ph:ProfileHandler xsi:type="ph:SAML2SSO" inboundBinding="urn:mace:shibboleth:2.0:profiles:AuthnRequest" 
                       outboundBindingEnumeration="urn:oasis:names:tc:SAML:2.0:bindings:HTTP-POST-SimpleSign
                                                   urn:oasis:names:tc:SAML:2.0:bindings:HTTP-POST
                                                   urn:oasis:names:tc:SAML:2.0:bindings:HTTP-Artifact">
        <ph:RequestPath>/SAML2/Unsolicited/SSO</ph:RequestPath>
    </ph:ProfileHandler>

    <ph:ProfileHandler xsi:type="ph:SAML2ECP" inboundBinding="urn:oasis:names:tc:SAML:2.0:bindings:SOAP" 
                       outboundBindingEnumeration="urn:oasis:names:tc:SAML:2.0:bindings:SOAP">
        <ph:RequestPath>/SAML2/SOAP/ECP</ph:RequestPath>
    </ph:ProfileHandler>

    <ph:ProfileHandler xsi:type="ph:SAML2SLO" inboundBinding="urn:oasis:names:tc:SAML:2.0:bindings:HTTP-Redirect"
                       outboundBindingEnumeration="urn:oasis:names:tc:SAML:2.0:bindings:HTTP-Redirect
                                                   urn:oasis:names:tc:SAML:2.0:bindings:HTTP-POST-SimpleSign
                                                   urn:oasis:names:tc:SAML:2.0:bindings:HTTP-POST
                                                   urn:oasis:names:tc:SAML:2.0:bindings:HTTP-Artifact">
        <ph:RequestPath>/SAML2/Redirect/SLO</ph:RequestPath>
    </ph:ProfileHandler>

    <ph:ProfileHandler xsi:type="ph:SAML2SLO" inboundBinding="urn:oasis:names:tc:SAML:2.0:bindings:HTTP-POST"
                       outboundBindingEnumeration="urn:oasis:names:tc:SAML:2.0:bindings:HTTP-Redirect
                                                   urn:oasis:names:tc:SAML:2.0:bindings:HTTP-POST-SimpleSign
                                                   urn:oasis:names:tc:SAML:2.0:bindings:HTTP-POST
                                                   urn:oasis:names:tc:SAML:2.0:bindings:HTTP-Artifact">
        <ph:RequestPath>/SAML2/POST/SLO</ph:RequestPath>
    </ph:ProfileHandler>

    <ph:ProfileHandler xsi:type="ph:SAML2SLO" inboundBinding="urn:oasis:names:tc:SAML:2.0:bindings:HTTP-POST-SimpleSign" 
                       outboundBindingEnumeration="urn:oasis:names:tc:SAML:2.0:bindings:HTTP-Redirect
                                                   urn:oasis:names:tc:SAML:2.0:bindings:HTTP-POST-SimpleSign
                                                   urn:oasis:names:tc:SAML:2.0:bindings:HTTP-POST
                                                   urn:oasis:names:tc:SAML:2.0:bindings:HTTP-Artifact">
        <ph:RequestPath>/SAML2/POST-SimpleSign/SLO</ph:RequestPath>
    </ph:ProfileHandler>

    <ph:ProfileHandler xsi:type="ph:SAML2SLO" inboundBinding="urn:oasis:names:tc:SAML:2.0:bindings:SOAP"
                       outboundBindingEnumeration="urn:oasis:names:tc:SAML:2.0:bindings:SOAP">
        <ph:RequestPath>/SAML2/SOAP/SLO</ph:RequestPath>
    </ph:ProfileHandler>
    
    <ph:ProfileHandler xsi:type="ph:SAML2SLO" inboundBinding="urn:mace:shibboleth:2.0:profiles:LocalLogout">
        <ph:RequestPath>/Logout</ph:RequestPath>
    </ph:ProfileHandler>
    
    <ph:ProfileHandler xsi:type="ph:SAML2AttributeQuery" inboundBinding="urn:oasis:names:tc:SAML:2.0:bindings:SOAP" 
                       outboundBindingEnumeration="urn:oasis:names:tc:SAML:2.0:bindings:SOAP">
        <ph:RequestPath>/SAML2/SOAP/AttributeQuery</ph:RequestPath>
    </ph:ProfileHandler>
    
    <ph:ProfileHandler xsi:type="ph:SAML2ArtifactResolution" inboundBinding="urn:oasis:names:tc:SAML:2.0:bindings:SOAP" 
                       outboundBindingEnumeration="urn:oasis:names:tc:SAML:2.0:bindings:SOAP">
        <ph:RequestPath>/SAML2/SOAP/ArtifactResolution</ph:RequestPath>
    </ph:ProfileHandler>
    
    <!-- Login Handlers -->
    <!-- <ph:LoginHandler xsi:type="ph:RemoteUser">
        <ph:AuthenticationMethod>urn:oasis:names:tc:SAML:2.0:ac:classes:unspecified</ph:AuthenticationMethod>
    </ph:LoginHandler> -->
    
    <!-- Login handler that delegates the act of authentication to an external system. -->
    <!-- This login handler and the RemoteUser login handler will be merged in the next major release. -->
    <!--
    <ph:LoginHandler xsi:type="ph:ExternalAuthn">
        <ph:AuthenticationMethod>urn:oasis:names:tc:SAML:2.0:ac:classes:unspecified</ph:AuthenticationMethod>
        <ph:QueryParam name="foo" value="bar" />
    </ph:LoginHandler>
    -->
    
    <!--  Username/password login handler -->

    <ph:LoginHandler xsi:type="ph:UsernamePassword" 
                  jaasConfigurationLocation="file:///opt/shibboleth-idp/conf/login.config">
        <ph:AuthenticationMethod>urn:oasis:names:tc:SAML:2.0:ac:classes:PasswordProtectedTransport</ph:AuthenticationMethod>
    </ph:LoginHandler>

    
    <!-- 
        Removal of this login handler will disable SSO support, that is it will require the user to authenticate 
        on every request.
    -->
    <ph:LoginHandler xsi:type="ph:PreviousSession">
        <ph:AuthenticationMethod>urn:oasis:names:tc:SAML:2.0:ac:classes:PreviousSession</ph:AuthenticationMethod>
    </ph:LoginHandler>

</ph:ProfileHandlerGroup>
\end{lstlisting}

\subsubsection{/opt/shibboleth-idp/conf/login.config}
\begin{lstlisting}
/*
  This is the JAAS configuration file used by the Shibboleth IdP.
  
  A JAAS configuration file is a grouping of LoginModules defined in the following manner:
  <LoginModuleClass> <Flag> <ModuleOptions>;
  
  LoginModuleClass - fully qualified class name of the LoginModule class
  Flag             - indicates whether the requirement level for the modules; 
                         allowed values: required, requisite, sufficient, optional
  ModuleOptions    - a space delimited list of name="value" options
  
  For complete documentation on the format of this file see:
  http://java.sun.com/j2se/1.5.0/docs/api/javax/security/auth/login/Configuration.html
  
  For LoginModules available within the Sun JVM see:
  http://java.sun.com/j2se/1.5.0/docs/guide/security/jaas/tutorials/LoginConfigFile.html
  
  Warning: Do NOT use Sun's JNDI LoginModule to authentication against an LDAP directory,
  Use the LdapLoginModule that ships with Shibboleth and is demonstrated below.

  Note, the application identifier MUST be ShibUserPassAuth
*/


ShibUserPassAuth {

// Example LDAP authentication
// See: https://wiki.shibboleth.net/confluence/display/SHIB2/IdPAuthUserPass

   edu.vt.middleware.ldap.jaas.LdapLoginModule required
      ldapUrl="ldap://ldap.shib.lan:389"
      baseDn="dc=shib,dc=lan"
      subtreeSearch="true"
      tls="true"
      bindDn="uid=shib_ldap,dc=shib,dc=lan"
      bindCredential="shib_ldap"
      userFilter="uid={0}";


// Example Kerberos authentication, requires Oracle's JVM or OpenJDK.
// Warning: The module does not support verifying the KDC using a keytab.
// See: https://wiki.shibboleth.net/confluence/display/SHIB2/IdPAuthUserPass
/*
   com.sun.security.auth.module.Krb5LoginModule required;
*/

};
\end{lstlisting}

\subsubsection{/opt/shibboleth-idp/conf/logging.xml}
\begin{lstlisting}[language=xml]
<?xml version="1.0" encoding="UTF-8"?>
<configuration>
    
    <!--
        Loggers define indicate which packages/categories are logged, at which level, and to which appender.
        Levels: OFF, ERROR, WARN, INFO, DEBUG, TRACE, ALL
    -->
    <!-- Logs IdP, but not OpenSAML, messages -->
    <logger name="edu.internet2.middleware.shibboleth" level="INFO"/>

    <!-- Logs OpenSAML, but not IdP, messages -->
    <logger name="org.opensaml" level="WARN"/>
    
    <!-- Logs LDAP related messages -->
    <logger name="edu.vt.middleware.ldap" level="DEBUG"/>

    <!-- uApprove Logger -->
    <logger name="ch.SWITCH.aai.uApprove" level="DEBUG"/>
    
    <!-- Logs inbound and outbound protocols messages at DEBUG level -->
    <!--
    <logger name="PROTOCOL_MESSAGE" level="DEBUG" />
    -->
    
    <!-- 
        Normally you should not edit below this point.  These default configurations are sufficient for 
        almost every system.
    -->

    <!-- 
        Logging appenders define where and how logging messages are logged.
     -->
    <appender name="IDP_ACCESS" class="ch.qos.logback.core.rolling.RollingFileAppender">
        <File>/opt/shibboleth-idp/logs/idp-access.log</File>

        <rollingPolicy class="ch.qos.logback.core.rolling.TimeBasedRollingPolicy">
            <FileNamePattern>/opt/shibboleth-idp/logs/idp-access-%d{yyyy-MM-dd}.log</FileNamePattern>
        </rollingPolicy>

        <encoder class="ch.qos.logback.classic.encoder.PatternLayoutEncoder">
            <charset>UTF-8</charset>
            <Pattern>%msg%n</Pattern>
        </encoder>
    </appender>

    <appender name="IDP_AUDIT" class="ch.qos.logback.core.rolling.RollingFileAppender">
        <File>/opt/shibboleth-idp/logs/idp-audit.log</File>

        <rollingPolicy class="ch.qos.logback.core.rolling.TimeBasedRollingPolicy">
            <FileNamePattern>/opt/shibboleth-idp/logs/idp-audit-%d{yyyy-MM-dd}.log</FileNamePattern>
        </rollingPolicy>

        <encoder class="ch.qos.logback.classic.encoder.PatternLayoutEncoder">
            <charset>UTF-8</charset>
            <Pattern>%msg%n</Pattern>
        </encoder>
    </appender>

    <appender name="IDP_PROCESS" class="ch.qos.logback.core.rolling.RollingFileAppender">
        <File>/opt/shibboleth-idp/logs/idp-process.log</File>
        
        <rollingPolicy class="ch.qos.logback.core.rolling.TimeBasedRollingPolicy">
            <FileNamePattern>/opt/shibboleth-idp/logs/idp-process-%d{yyyy-MM-dd}.log</FileNamePattern>
        </rollingPolicy>

        <encoder class="ch.qos.logback.classic.encoder.PatternLayoutEncoder">
            <charset>UTF-8</charset>
            <Pattern>%date{HH:mm:ss.SSS} - %level [%logger:%line] - %msg%n</Pattern>
        </encoder>
    </appender>
  
    <logger name="Shibboleth-Access" level="ALL">
        <appender-ref ref="IDP_ACCESS"/>
    </logger>
    
    <logger name="Shibboleth-Audit" level="ALL">
        <appender-ref ref="IDP_AUDIT"/>
    </logger>
        
    <logger name="org.springframework" level="ERROR"/>
    
    <logger name="org.apache.catalina" level="ERROR"/>

    <root level="ERROR">
        <appender-ref ref="IDP_PROCESS"/>
    </root>

</configuration>
\end{lstlisting}

\subsubsection{/opt/shibboleth-idp/conf/attribute-resolver.xml}
\begin{lstlisting}[language=xml]
<?xml version="1.0" encoding="UTF-8"?>
<!-- 
    This file is an EXAMPLE configuration file.  While the configuration presented in this 
    example file is functional, it isn't very interesting.  However, there are lots of example
    attributes, encoders, and a couple example data connectors.
    
    Not all attribute definitions, data connectors, or principal connectors are demonstrated.
    Deployers should refer to the Shibboleth 2 documentation for a complete list of components 
    and their options.
-->
<resolver:AttributeResolver xmlns:resolver="urn:mace:shibboleth:2.0:resolver" xmlns:xsi="http://www.w3.org/2001/XMLSchema-instance" 
                            xmlns:pc="urn:mace:shibboleth:2.0:resolver:pc" xmlns:ad="urn:mace:shibboleth:2.0:resolver:ad" 
                            xmlns:dc="urn:mace:shibboleth:2.0:resolver:dc" xmlns:enc="urn:mace:shibboleth:2.0:attribute:encoder" 
                            xmlns:sec="urn:mace:shibboleth:2.0:security" 
                            xsi:schemaLocation="urn:mace:shibboleth:2.0:resolver classpath:/schema/shibboleth-2.0-attribute-resolver.xsd
                                               urn:mace:shibboleth:2.0:resolver:pc classpath:/schema/shibboleth-2.0-attribute-resolver-pc.xsd
                                               urn:mace:shibboleth:2.0:resolver:ad classpath:/schema/shibboleth-2.0-attribute-resolver-ad.xsd
                                               urn:mace:shibboleth:2.0:resolver:dc classpath:/schema/shibboleth-2.0-attribute-resolver-dc.xsd
                                               urn:mace:shibboleth:2.0:attribute:encoder classpath:/schema/shibboleth-2.0-attribute-encoder.xsd
                                               urn:mace:shibboleth:2.0:security classpath:/schema/shibboleth-2.0-security.xsd">

    <!-- ========================================== -->
    <!--      Attribute Definitions                 -->
    <!-- ========================================== -->

    <!-- Schema: Core schema attributes-->
    
    <resolver:AttributeDefinition xsi:type="ad:Simple" id="uid" sourceAttributeID="uid">
        <resolver:Dependency ref="myLDAP" />
        <resolver:AttributeEncoder xsi:type="enc:SAML1String" name="urn:mace:dir:attribute-def:uid" />
        <resolver:AttributeEncoder xsi:type="enc:SAML2String" name="urn:oid:0.9.2342.19200300.100.1.1" friendlyName="uid" />
    </resolver:AttributeDefinition>

    <!--
    <resolver:AttributeDefinition xsi:type="ad:Simple" id="email" sourceAttributeID="mail">
        <resolver:Dependency ref="myLDAP" />
        <resolver:AttributeEncoder xsi:type="enc:SAML1String" name="urn:mace:dir:attribute-def:mail" />
        <resolver:AttributeEncoder xsi:type="enc:SAML2String" name="urn:oid:0.9.2342.19200300.100.1.3" friendlyName="mail" />
    </resolver:AttributeDefinition>

    <resolver:AttributeDefinition xsi:type="ad:Simple" id="homePhone" sourceAttributeID="homePhone">
        <resolver:Dependency ref="myLDAP" />
        <resolver:AttributeEncoder xsi:type="enc:SAML1String" name="urn:mace:dir:attribute-def:homePhone" />
        <resolver:AttributeEncoder xsi:type="enc:SAML2String" name="urn:oid:0.9.2342.19200300.100.1.20" friendlyName="homePhone" />
    </resolver:AttributeDefinition>

    <resolver:AttributeDefinition xsi:type="ad:Simple" id="homePostalAddress" sourceAttributeID="homePostalAddress">
        <resolver:Dependency ref="myLDAP" />
        <resolver:AttributeEncoder xsi:type="enc:SAML1String" name="urn:mace:dir:attribute-def:homePostalAddress" />
        <resolver:AttributeEncoder xsi:type="enc:SAML2String" name="urn:oid:0.9.2342.19200300.100.1.39" friendlyName="homePostalAddress" />
    </resolver:AttributeDefinition>

    <resolver:AttributeDefinition xsi:type="ad:Simple" id="mobileNumber" sourceAttributeID="mobile">
        <resolver:Dependency ref="myLDAP" />
        <resolver:AttributeEncoder xsi:type="enc:SAML1String" name="urn:mace:dir:attribute-def:mobile" />
        <resolver:AttributeEncoder xsi:type="enc:SAML2String" name="urn:oid:0.9.2342.19200300.100.1.41" friendlyName="mobile" />
    </resolver:AttributeDefinition>

    <resolver:AttributeDefinition xsi:type="ad:Simple" id="pagerNumber" sourceAttributeID="pager">
        <resolver:Dependency ref="myLDAP" />
        <resolver:AttributeEncoder xsi:type="enc:SAML1String" name="urn:mace:dir:attribute-def:pager" />
        <resolver:AttributeEncoder xsi:type="enc:SAML2String" name="urn:oid:0.9.2342.19200300.100.1.42" friendlyName="pager" />
    </resolver:AttributeDefinition>
    -->
    <resolver:AttributeDefinition xsi:type="ad:Simple" id="commonName" sourceAttributeID="cn">
        <resolver:Dependency ref="myLDAP" />
        <resolver:AttributeEncoder xsi:type="enc:SAML1String" name="urn:mace:dir:attribute-def:cn" />
        <resolver:AttributeEncoder xsi:type="enc:SAML2String" name="urn:oid:2.5.4.3" friendlyName="cn" />
    </resolver:AttributeDefinition>

    <resolver:AttributeDefinition xsi:type="ad:Simple" id="surname" sourceAttributeID="sn">
        <resolver:Dependency ref="myLDAP" />
        <resolver:AttributeEncoder xsi:type="enc:SAML1String" name="urn:mace:dir:attribute-def:sn" />
        <resolver:AttributeEncoder xsi:type="enc:SAML2String" name="urn:oid:2.5.4.4" friendlyName="sn" />
    </resolver:AttributeDefinition>
    <!--
    <resolver:AttributeDefinition xsi:type="ad:Simple" id="locality" sourceAttributeID="l">
        <resolver:Dependency ref="myLDAP" />
        <resolver:AttributeEncoder xsi:type="enc:SAML1String" name="urn:mace:dir:attribute-def:l" />
        <resolver:AttributeEncoder xsi:type="enc:SAML2String" name="urn:oid:2.5.4.7" friendlyName="l" />
    </resolver:AttributeDefinition>

    <resolver:AttributeDefinition xsi:type="ad:Simple" id="stateProvince" sourceAttributeID="st">
        <resolver:Dependency ref="myLDAP" />
        <resolver:AttributeEncoder xsi:type="enc:SAML1String" name="urn:mace:dir:attribute-def:st" />
        <resolver:AttributeEncoder xsi:type="enc:SAML2String" name="urn:oid:2.5.4.8" friendlyName="st" />
    </resolver:AttributeDefinition>

    <resolver:AttributeDefinition xsi:type="ad:Simple" id="street" sourceAttributeID="street">
        <resolver:Dependency ref="myLDAP" />
        <resolver:AttributeEncoder xsi:type="enc:SAML1String" name="urn:mace:dir:attribute-def:street" />
        <resolver:AttributeEncoder xsi:type="enc:SAML2String" name="urn:oid:2.5.4.9" friendlyName="street" />
    </resolver:AttributeDefinition>

    <resolver:AttributeDefinition xsi:type="ad:Simple" id="organizationName" sourceAttributeID="o">
        <resolver:Dependency ref="myLDAP" />
        <resolver:AttributeEncoder xsi:type="enc:SAML1String" name="urn:mace:dir:attribute-def:o" />
        <resolver:AttributeEncoder xsi:type="enc:SAML2String" name="urn:oid:2.5.4.10" friendlyName="o" />
    </resolver:AttributeDefinition>

    <resolver:AttributeDefinition xsi:type="ad:Simple" id="organizationalUnit" sourceAttributeID="ou">
        <resolver:Dependency ref="myLDAP" />
        <resolver:AttributeEncoder xsi:type="enc:SAML1String" name="urn:mace:dir:attribute-def:ou" />
        <resolver:AttributeEncoder xsi:type="enc:SAML2String" name="urn:oid:2.5.4.11" friendlyName="ou" />
    </resolver:AttributeDefinition>

    <resolver:AttributeDefinition xsi:type="ad:Simple" id="title" sourceAttributeID="title">
        <resolver:Dependency ref="myLDAP" />
        <resolver:AttributeEncoder xsi:type="enc:SAML1String" name="urn:mace:dir:attribute-def:title" />
        <resolver:AttributeEncoder xsi:type="enc:SAML2String" name="urn:oid:2.5.4.12" friendlyName="title" />
    </resolver:AttributeDefinition>

    <resolver:AttributeDefinition xsi:type="ad:Simple" id="postalAddress" sourceAttributeID="postalAddress">
        <resolver:Dependency ref="myLDAP" />
        <resolver:AttributeEncoder xsi:type="enc:SAML1String" name="urn:mace:dir:attribute-def:postalAddress" />
        <resolver:AttributeEncoder xsi:type="enc:SAML2String" name="urn:oid:2.5.4.16" friendlyName="postalAddress" />
    </resolver:AttributeDefinition>

    <resolver:AttributeDefinition xsi:type="ad:Simple" id="postalCode" sourceAttributeID="postalCode">
        <resolver:Dependency ref="myLDAP" />
        <resolver:AttributeEncoder xsi:type="enc:SAML1String" name="urn:mace:dir:attribute-def:postalCode" />
        <resolver:AttributeEncoder xsi:type="enc:SAML2String" name="urn:oid:2.5.4.17" friendlyName="postalCode" />
    </resolver:AttributeDefinition>

    <resolver:AttributeDefinition xsi:type="ad:Simple" id="postOfficeBox" sourceAttributeID="postOfficeBox">
        <resolver:Dependency ref="myLDAP" />
        <resolver:AttributeEncoder xsi:type="enc:SAML1String" name="urn:mace:dir:attribute-def:postOfficeBox" />
        <resolver:AttributeEncoder xsi:type="enc:SAML2String" name="urn:oid:2.5.4.18" friendlyName="postOfficeBox" />
    </resolver:AttributeDefinition>

    <resolver:AttributeDefinition xsi:type="ad:Simple" id="telephoneNumber" sourceAttributeID="telephoneNumber">
        <resolver:Dependency ref="myLDAP" />
        <resolver:AttributeEncoder xsi:type="enc:SAML1String" name="urn:mace:dir:attribute-def:telephoneNumber" />
        <resolver:AttributeEncoder xsi:type="enc:SAML2String" name="urn:oid:2.5.4.20" friendlyName="telephoneNumber" />
    </resolver:AttributeDefinition>
    -->
    <resolver:AttributeDefinition xsi:type="ad:Simple" id="givenName" sourceAttributeID="givenName">
        <resolver:Dependency ref="myLDAP" />
        <resolver:AttributeEncoder xsi:type="enc:SAML1String" name="urn:mace:dir:attribute-def:givenName" />
        <resolver:AttributeEncoder xsi:type="enc:SAML2String" name="urn:oid:2.5.4.42" friendlyName="givenName" />
    </resolver:AttributeDefinition>
    <!--
    <resolver:AttributeDefinition xsi:type="ad:Simple" id="initials" sourceAttributeID="initials">
        <resolver:Dependency ref="myLDAP" />
        <resolver:AttributeEncoder xsi:type="enc:SAML1String" name="urn:mace:dir:attribute-def:initials" />
        <resolver:AttributeEncoder xsi:type="enc:SAML2String" name="urn:oid:2.5.4.43" friendlyName="initials" />
    </resolver:AttributeDefinition>
     -->

    <!-- Schema: inetOrgPerson attributes-->
    <!--
    <resolver:AttributeDefinition xsi:type="ad:Simple" id="departmentNumber" sourceAttributeID="departmentNumber">
        <resolver:Dependency ref="myLDAP" />
        <resolver:AttributeEncoder xsi:type="enc:SAML1String" name="urn:mace:dir:attribute-def:departmentNumber" />
        <resolver:AttributeEncoder xsi:type="enc:SAML2String" name="urn:oid:2.16.840.1.113730.3.1.2" friendlyName="departmentNumber" />
    </resolver:AttributeDefinition>
    
    <resolver:AttributeDefinition xsi:type="ad:Simple" id="displayName" sourceAttributeID="displayName">
        <resolver:Dependency ref="myLDAP" />
        <resolver:AttributeEncoder xsi:type="enc:SAML1String" name="urn:mace:dir:attribute-def:displayName" />
        <resolver:AttributeEncoder xsi:type="enc:SAML2String" name="urn:oid:2.16.840.1.113730.3.1.241" friendlyName="displayName" />
    </resolver:AttributeDefinition> 

    <resolver:AttributeDefinition xsi:type="ad:Simple" id="employeeNumber" sourceAttributeID="employeeNumber">
        <resolver:Dependency ref="myLDAP" />
        <resolver:AttributeEncoder xsi:type="enc:SAML1String" name="urn:mace:dir:attribute-def:employeeNumber" />
        <resolver:AttributeEncoder xsi:type="enc:SAML2String" name="urn:oid:2.16.840.1.113730.3.1.3" friendlyName="employeeNumber" />
    </resolver:AttributeDefinition>

    <resolver:AttributeDefinition xsi:type="ad:Simple" id="employeeType" sourceAttributeID="employeeType">
        <resolver:Dependency ref="myLDAP" />
        <resolver:AttributeEncoder xsi:type="enc:SAML1String" name="urn:mace:dir:attribute-def:employeeType" />
        <resolver:AttributeEncoder xsi:type="enc:SAML2String" name="urn:oid:2.16.840.1.113730.3.1.4" friendlyName="employeeType" />
    </resolver:AttributeDefinition>

    <resolver:AttributeDefinition xsi:type="ad:Simple" id="jpegPhoto" sourceAttributeID="jpegPhoto">
        <resolver:Dependency ref="myLDAP" />
        <resolver:AttributeEncoder xsi:type="enc:SAML1String" name="urn:mace:dir:attribute-def:jpegPhoto" />
        <resolver:AttributeEncoder xsi:type="enc:SAML2String" name="urn:oid:0.9.2342.19200300.100.1.60" friendlyName="jpegPhoto" />
    </resolver:AttributeDefinition>

    <resolver:AttributeDefinition xsi:type="ad:Simple" id="preferredLanguage" sourceAttributeID="preferredLanguage">
        <resolver:Dependency ref="myLDAP" />
        <resolver:AttributeEncoder xsi:type="enc:SAML1String" name="urn:mace:dir:attribute-def:preferredLanguage" />
        <resolver:AttributeEncoder xsi:type="enc:SAML2String" name="urn:oid:2.16.840.1.113730.3.1.39" friendlyName="preferredLanguage" />
    </resolver:AttributeDefinition>
    -->

    <!-- Schema: eduPerson attributes -->
    
    <resolver:AttributeDefinition xsi:type="ad:Simple" id="eduPersonAffiliation" sourceAttributeID="eduPersonAffiliation">
        <resolver:Dependency ref="myLDAP" />
        <resolver:AttributeEncoder xsi:type="enc:SAML1String" name="urn:mace:dir:attribute-def:eduPersonAffiliation" />
        <resolver:AttributeEncoder xsi:type="enc:SAML2String" name="urn:oid:1.3.6.1.4.1.5923.1.1.1.1" friendlyName="eduPersonAffiliation" />
    </resolver:AttributeDefinition>

    <resolver:AttributeDefinition xsi:type="ad:Simple" id="eduPersonEntitlement" sourceAttributeID="eduPersonEntitlement">
        <resolver:Dependency ref="myLDAP" />
        <resolver:AttributeEncoder xsi:type="enc:SAML1String" name="urn:mace:dir:attribute-def:eduPersonEntitlement" />
        <resolver:AttributeEncoder xsi:type="enc:SAML2String" name="urn:oid:1.3.6.1.4.1.5923.1.1.1.7" friendlyName="eduPersonEntitlement" />
    </resolver:AttributeDefinition>
    <!--
    <resolver:AttributeDefinition xsi:type="ad:Simple" id="eduPersonNickname" sourceAttributeID="eduPersonNickname">
        <resolver:Dependency ref="myLDAP" />
        <resolver:AttributeEncoder xsi:type="enc:SAML1String" name="urn:mace:dir:attribute-def:eduPersonNickname" />
        <resolver:AttributeEncoder xsi:type="enc:SAML2String" name="urn:oid:1.3.6.1.4.1.5923.1.1.1.2" friendlyName="eduPersonNickname" />
    </resolver:AttributeDefinition>

    <resolver:AttributeDefinition xsi:type="ad:Simple" id="eduPersonOrgDN" sourceAttributeID="eduPersonOrgDN">
        <resolver:Dependency ref="myLDAP" />
        <resolver:AttributeEncoder xsi:type="enc:SAML1String" name="urn:mace:dir:attribute-def:eduPersonOrgDN" />
        <resolver:AttributeEncoder xsi:type="enc:SAML2String" name="urn:oid:1.3.6.1.4.1.5923.1.1.1.3" friendlyName="eduPersonOrgDN" />
    </resolver:AttributeDefinition>

    <resolver:AttributeDefinition xsi:type="ad:Simple" id="eduPersonOrgUnitDN" sourceAttributeID="eduPersonOrgUnitDN">
        <resolver:Dependency ref="myLDAP" />
        <resolver:AttributeEncoder xsi:type="enc:SAML1String" name="urn:mace:dir:attribute-def:eduPersonOrgUnitDN" />
        <resolver:AttributeEncoder xsi:type="enc:SAML2String" name="urn:oid:1.3.6.1.4.1.5923.1.1.1.4" friendlyName="eduPersonOrgUnitDN" />
    </resolver:AttributeDefinition>
    -->
    <resolver:AttributeDefinition xsi:type="ad:Simple" id="eduPersonPrimaryAffiliation" sourceAttributeID="eduPersonPrimaryAffiliation">
        <resolver:Dependency ref="myLDAP" />
        <resolver:AttributeEncoder xsi:type="enc:SAML1String" name="urn:mace:dir:attribute-def:eduPersonPrimaryAffiliation" />
        <resolver:AttributeEncoder xsi:type="enc:SAML2String" name="urn:oid:1.3.6.1.4.1.5923.1.1.1.5" friendlyName="eduPersonPrimaryAffiliation" />
    </resolver:AttributeDefinition>
    <!--
    <resolver:AttributeDefinition xsi:type="ad:Simple" id="eduPersonPrimaryOrgUnitDN" sourceAttributeID="eduPersonPrimaryOrgUnitDN">
        <resolver:Dependency ref="myLDAP" />
        <resolver:AttributeEncoder xsi:type="enc:SAML1String" name="urn:mace:dir:attribute-def:eduPersonPrimaryOrgUnitDN" />
        <resolver:AttributeEncoder xsi:type="enc:SAML2String" name="urn:oid:1.3.6.1.4.1.5923.1.1.1.8" friendlyName="eduPersonPrimaryOrgUnitDN" />
    </resolver:AttributeDefinition>
    -->
    <resolver:AttributeDefinition xsi:type="ad:Scoped" id="eduPersonPrincipalName" scope="shib.lan" sourceAttributeID="uid">
        <resolver:Dependency ref="myLDAP" />
        <resolver:AttributeEncoder xsi:type="enc:SAML1ScopedString" name="urn:mace:dir:attribute-def:eduPersonPrincipalName" />
        <resolver:AttributeEncoder xsi:type="enc:SAML2ScopedString" name="urn:oid:1.3.6.1.4.1.5923.1.1.1.6" friendlyName="eduPersonPrincipalName" />
    </resolver:AttributeDefinition>

    <resolver:AttributeDefinition xsi:type="ad:Scoped" id="eduPersonScopedAffiliation" scope="shib.lan" sourceAttributeID="eduPersonAffiliation">
        <resolver:Dependency ref="myLDAP" />
        <resolver:AttributeEncoder xsi:type="enc:SAML1ScopedString" name="urn:mace:dir:attribute-def:eduPersonScopedAffiliation" />
        <resolver:AttributeEncoder xsi:type="enc:SAML2ScopedString" name="urn:oid:1.3.6.1.4.1.5923.1.1.1.9" friendlyName="eduPersonScopedAffiliation" />
    </resolver:AttributeDefinition>
    <!--
    <resolver:AttributeDefinition xsi:type="ad:Simple" id="eduPersonAssurance" sourceAttributeID="eduPersonAssurance">
        <resolver:Dependency ref="myLDAP" />
        <resolver:AttributeEncoder xsi:type="enc:SAML1String" name="urn:mace:dir:attribute-def:eduPersonAssurance" />
        <resolver:AttributeEncoder xsi:type="enc:SAML2String" name="urn:oid:1.3.6.1.4.1.5923.1.1.1.11" friendlyName="eduPersonAssurance" />
    </resolver:AttributeDefinition>
    -->
        
    <!--
    <resolver:AttributeDefinition xsi:type="ad:SAML2NameID" id="eduPersonTargetedID" 
                                  nameIdFormat="urn:oasis:names:tc:SAML:2.0:nameid-format:persistent" sourceAttributeID="computedID">
        <resolver:Dependency ref="computedID" />
        <resolver:AttributeEncoder xsi:type="enc:SAML1XMLObject" name="urn:oid:1.3.6.1.4.1.5923.1.1.1.10" />
        <resolver:AttributeEncoder xsi:type="enc:SAML2XMLObject" name="urn:oid:1.3.6.1.4.1.5923.1.1.1.10" friendlyName="eduPersonTargetedID" />
    </resolver:AttributeDefinition>
    -->
    
    <!-- DFN Attribute -->
    <resolver:AttributeDefinition xsi:type="ad:Simple" id="dfnEduPersonCostCenter" sourceAttributeID="dfnEduPersonCostCenter">
        <resolver:Dependency ref="myLDAP" />
        <resolver:AttributeEncoder xsi:type="enc:SAML1String" name="urn:mace:dir:attribute-def:dfnEduPersonCostCenter" />
        <resolver:AttributeEncoder xsi:type="enc:SAML2String" name="urn:oid:1.3.6.1.4.1.22177.400.1.1.3.1" friendlyName="dfnEduPersonCostCenter" />
    </resolver:AttributeDefinition>

    <resolver:AttributeDefinition xsi:type="ad:Simple" id="dfnEduPersonStudyBranch1" sourceAttributeID="dfnEduPersonStudyBranch1">
        <resolver:Dependency ref="myLDAP" />
        <resolver:AttributeEncoder xsi:type="enc:SAML1String" name="urn:mace:dir:attribute-def:dfnEduPersonStudyBranch1" />
        <resolver:AttributeEncoder xsi:type="enc:SAML2String" name="urn:oid:1.3.6.1.4.1.22177.400.1.1.3.2" friendlyName="dfnEduPersonStudyBranch1" />
    </resolver:AttributeDefinition>

    <resolver:AttributeDefinition xsi:type="ad:Simple" id="dfnEduPersonStudyBranch2" sourceAttributeID="dfnEduPersonStudyBranch2">
        <resolver:Dependency ref="myLDAP" />
        <resolver:AttributeEncoder xsi:type="enc:SAML1String" name="urn:mace:dir:attribute-def:dfnEduPersonStudyBranch2" />
        <resolver:AttributeEncoder xsi:type="enc:SAML2String" name="urn:oid:1.3.6.1.4.1.22177.400.1.1.3.3" friendlyName="dfnEduPersonStudyBranch2" />
    </resolver:AttributeDefinition>

    <resolver:AttributeDefinition xsi:type="ad:Simple" id="dfnEduPersonStudyBranch3" sourceAttributeID="dfnEduPersonStudyBranch3">
        <resolver:Dependency ref="myLDAP" />
        <resolver:AttributeEncoder xsi:type="enc:SAML1String" name="urn:mace:dir:attribute-def:dfnEduPersonStudyBranch2" />
        <resolver:AttributeEncoder xsi:type="enc:SAML2String" name="urn:oid:1.3.6.1.4.1.22177.400.1.1.3.4" friendlyName="dfnEduPersonStudyBranch3" />
    </resolver:AttributeDefinition>

    <resolver:AttributeDefinition xsi:type="ad:Simple" id="dfnEduPersonFieldOfStudyString" sourceAttributeID="dfnEduPersonFieldOfStudyString">
        <resolver:Dependency ref="myLDAP" />
        <resolver:AttributeEncoder xsi:type="enc:SAML1String" name="urn:mace:dir:attribute-def:dfnEduPersonFieldOfStudyString" />
        <resolver:AttributeEncoder xsi:type="enc:SAML2String" name="urn:oid:1.3.6.1.4.1.22177.400.1.1.3.5" friendlyName="dfnEduPersonFieldOfStudyString" />
    </resolver:AttributeDefinition>
    <!--
    <resolver:AttributeDefinition xsi:type="ad:Simple" id="dfnEduPersonFinalDegree" sourceAttributeID="dfnEduPersonFinalDegree">
        <resolver:Dependency ref="myLDAP" />
        <resolver:AttributeEncoder xsi:type="enc:SAML1String" name="urn:mace:dir:attribute-def:dfnEduPersonFinalDegree" />
        <resolver:AttributeEncoder xsi:type="enc:SAML2String" name="urn:oid:1.3.6.1.4.1.22177.400.1.1.3.6" friendlyName="dfnEduPersonFinalDegree" />
    </resolver:AttributeDefinition>

    <resolver:AttributeDefinition xsi:type="ad:Simple" id="dfnEduPersonTypeOfStudy" sourceAttributeID="dfnEduPersonTypeOfStudy">
        <resolver:Dependency ref="myLDAP" />
        <resolver:AttributeEncoder xsi:type="enc:SAML1String" name="urn:mace:dir:attribute-def:dfnEduPersonTypeOfStudy" />
        <resolver:AttributeEncoder xsi:type="enc:SAML2String" name="urn:oid:1.3.6.1.4.1.22177.400.1.1.3.7" friendlyName="dfnEduPersonTypeOfStudy" />
    </resolver:AttributeDefinition>

    <resolver:AttributeDefinition xsi:type="ad:Simple" id="dfnEduPersonTermsOfStudy" sourceAttributeID="dfnEduPersonTermsOfStudy">
        <resolver:Dependency ref="myLDAP" />
        <resolver:AttributeEncoder xsi:type="enc:SAML1String" name="urn:mace:dir:attribute-def:dfnEduPersonTermsOfStudy" />
        <resolver:AttributeEncoder xsi:type="enc:SAML2String" name="urn:oid:1.3.6.1.4.1.22177.400.1.1.3.8" friendlyName="dfnEduPersonTermsOfStudy" />
    </resolver:AttributeDefinition>

    <resolver:AttributeDefinition xsi:type="ad:Simple" id="dfnEduPersonBranchAndDegree" sourceAttributeID="dfnEduPersonBranchAndDegree">
        <resolver:Dependency ref="myLDAP" />
        <resolver:AttributeEncoder xsi:type="enc:SAML1String" name="urn:mace:dir:attribute-def:dfnEduPersonBranchAndDegree" />
        <resolver:AttributeEncoder xsi:type="enc:SAML2String" name="urn:oid:1.3.6.1.4.1.22177.400.1.1.3.9" friendlyName="dfnEduPersonBranchAndDegree" />
    </resolver:AttributeDefinition>

    <resolver:AttributeDefinition xsi:type="ad:Simple" id="dfnEduPersonBrachAndType" sourceAttributeID="dfnEduPersonBrachAndType">
        <resolver:Dependency ref="myLDAP" />
        <resolver:AttributeEncoder xsi:type="enc:SAML1String" name="urn:mace:dir:attribute-def:dfnEduPersonBrachAndType" />
        <resolver:AttributeEncoder xsi:type="enc:SAML2String" name="urn:oid:1.3.6.1.4.1.22177.400.1.1.3.10" friendlyName="dfnEduPersonBrachAndType" />
    </resolver:AttributeDefinition>

    <resolver:AttributeDefinition xsi:type="ad:Simple" id="dfnEduPersonFeaturesOfStudy" sourceAttributeID="dfnEduPersonFeaturesOfStudy">
        <resolver:Dependency ref="myLDAP" />
        <resolver:AttributeEncoder xsi:type="enc:SAML1String" name="urn:mace:dir:attribute-def:dfnEduPersonFeaturesOfStudy" />
        <resolver:AttributeEncoder xsi:type="enc:SAML2String" name="urn:oid:1.3.6.1.4.1.22177.400.1.1.3.11" friendlyName="dfnEduPersonFeaturesOfStudy" />
    </resolver:AttributeDefinition>
    -->
    <!-- Do NOT use the version of eduPersonTargetedID defined below unless you understand 
         why it was deprecated and know that this reason does not apply to you. -->
    <!--
    <resolver:AttributeDefinition xsi:type="ad:Scoped" id="eduPersonTargetedID.old" scope="shib.lan" sourceAttributeID="computedID">
        <resolver:Dependency ref="computedID" />
        <resolver:AttributeEncoder xsi:type="enc:SAML1ScopedString" name="urn:mace:dir:attribute-def:eduPersonTargetedID" />
    </resolver:AttributeDefinition>
    -->

    <!-- Name Identifier related attributes -->
    <resolver:AttributeDefinition id="transientId" xsi:type="ad:TransientId">
        <resolver:AttributeEncoder xsi:type="enc:SAML1StringNameIdentifier" nameFormat="urn:mace:shibboleth:1.0:nameIdentifier"/>
        <resolver:AttributeEncoder xsi:type="enc:SAML2StringNameID" nameFormat="urn:oasis:names:tc:SAML:2.0:nameid-format:transient"/>
    </resolver:AttributeDefinition>

    <!-- ========================================== -->
    <!--      Data Connectors                       -->
    <!-- ========================================== -->

    <!-- Example Static Connector -->
    <!--
    <resolver:DataConnector id="staticAttributes" xsi:type="dc:Static">
        <dc:Attribute id="eduPersonAffiliation">
            <dc:Value>member</dc:Value>
        </dc:Attribute>
        <dc:Attribute id="eduPersonEntitlement">
            <dc:Value>urn:example.org:entitlement:entitlement1</dc:Value>
            <dc:Value>urn:mace:dir:entitlement:common-lib-terms</dc:Value>
        </dc:Attribute>
    </resolver:DataConnector>
    -->

    <!-- Example Relational Database Connector -->
    <!--
    <resolver:DataConnector id="mySIS" xsi:type="dc:RelationalDatabase">
        <dc:ApplicationManagedConnection jdbcDriver="oracle.jdbc.driver.OracleDriver"
                                         jdbcURL="jdbc:oracle:thin:@db.example.org:1521:SomeDB" 
                                         jdbcUserName="myid" 
                                         jdbcPassword="mypassword" />
        <dc:QueryTemplate>
            <![CDATA[
                SELECT * FROM student WHERE gzbtpid = '$requestContext.principalName'
            ]]>
        </dc:QueryTemplate>

        <dc:Column columnName="gzbtpid" attributeID="uid" />
        <dc:Column columnName="fqlft" attributeID="gpa" type="Float" />
    </resolver:DataConnector>
     -->

    <!-- Example LDAP Connector -->
    
    <resolver:DataConnector id="myLDAP" xsi:type="dc:LDAPDirectory"
        ldapURL="ldap://ldap.shib.lan" 
        baseDN="dc=shib,dc=lan" 
        principal="uid=shib_ldap,dc=shib,dc=lan"
        principalCredential="shib_ldap">
        <dc:FilterTemplate>
            <![CDATA[
                (uid=$requestContext.principalName)
            ]]>
        </dc:FilterTemplate>
    </resolver:DataConnector>
    
    
    <!-- Computed targeted ID connector -->
    <!--
    <resolver:DataConnector xsi:type="dc:ComputedId"
                            id="computedID"
                            generatedAttributeID="computedID"
                            sourceAttributeID="uid"
                            salt="your random string here">
        <resolver:Dependency ref="myLDAP" />
    </resolver:DataConnector> 
    -->

    <!-- ========================================== -->
    <!--      Principal Connectors                  -->
    <!-- ========================================== -->
    <resolver:PrincipalConnector xsi:type="pc:Transient" id="shibTransient" nameIDFormat="urn:mace:shibboleth:1.0:nameIdentifier"/>
    <resolver:PrincipalConnector xsi:type="pc:Transient" id="saml1Unspec" nameIDFormat="urn:oasis:names:tc:SAML:1.1:nameid-format:unspecified"/>
    <resolver:PrincipalConnector xsi:type="pc:Transient" id="saml2Transient" nameIDFormat="urn:oasis:names:tc:SAML:2.0:nameid-format:transient"/>

</resolver:AttributeResolver>
\end{lstlisting}

\subsubsection{/opt/shibboleth-idp/conf/attribute-filter.xml}
\begin{lstlisting}[language=xml]
<?xml version="1.0" encoding="UTF-8"?>
<!-- 
    This file is an EXAMPLE policy file.  While the policy presented in this 
    example file is functional, it isn't very interesting.
    
    Deployers should refer to the Shibboleth 2 documentation for a complete list of components 
    and their options.
-->
<afp:AttributeFilterPolicyGroup id="ShibbolethFilterPolicy"
                                xmlns:afp="urn:mace:shibboleth:2.0:afp" xmlns:basic="urn:mace:shibboleth:2.0:afp:mf:basic" 
                                xmlns:saml="urn:mace:shibboleth:2.0:afp:mf:saml" xmlns:xsi="http://www.w3.org/2001/XMLSchema-instance" 
                                xsi:schemaLocation="urn:mace:shibboleth:2.0:afp classpath:/schema/shibboleth-2.0-afp.xsd
                                                    urn:mace:shibboleth:2.0:afp:mf:basic classpath:/schema/shibboleth-2.0-afp-mf-basic.xsd
                                                    urn:mace:shibboleth:2.0:afp:mf:saml classpath:/schema/shibboleth-2.0-afp-mf-saml.xsd">

    <!--  Release the transient ID to anyone -->
    <afp:AttributeFilterPolicy id="releaseTransientIdToAnyone">
        <afp:PolicyRequirementRule xsi:type="basic:ANY"/>

        <afp:AttributeRule attributeID="transientId">
            <afp:PermitValueRule xsi:type="basic:ANY"/>
        </afp:AttributeRule>

    </afp:AttributeFilterPolicy>

    <afp:AttributeFilterPolicy id="sp.shib.lan-Policy">

        <afp:PolicyRequirementRule xsi:type="basic:AttributeRequesterString" value="https://sp.shib.lan/shibboleth" />

        <afp:AttributeRule attributeID="eduPersonAffiliation">
            <afp:PermitValueRule xsi:type="basic:OR">
                <basic:Rule xsi:type="basic:AttributeValueString" value="faculty" ignoreCase="true" />
                <basic:Rule xsi:type="basic:AttributeValueString" value="student" ignoreCase="true" />
                <basic:Rule xsi:type="basic:AttributeValueString" value="staff" ignoreCase="true" />
                <basic:Rule xsi:type="basic:AttributeValueString" value="alum" ignoreCase="true" />
                <basic:Rule xsi:type="basic:AttributeValueString" value="member" ignoreCase="true" />
                <basic:Rule xsi:type="basic:AttributeValueString" value="affiliate" ignoreCase="true" />
                <basic:Rule xsi:type="basic:AttributeValueString" value="employee" ignoreCase="true" />
                <basic:Rule xsi:type="basic:AttributeValueString" value="library-walk-in" ignoreCase="true" />
            </afp:PermitValueRule>
        </afp:AttributeRule>

        <afp:AttributeRule attributeID="eduPersonPrimaryAffiliation">
            <afp:PermitValueRule xsi:type="basic:OR">
                <basic:Rule xsi:type="basic:AttributeValueString" value="faculty" ignoreCase="true" />
                <basic:Rule xsi:type="basic:AttributeValueString" value="student" ignoreCase="true" />
                <basic:Rule xsi:type="basic:AttributeValueString" value="staff" ignoreCase="true" />
                <basic:Rule xsi:type="basic:AttributeValueString" value="alum" ignoreCase="true" />
                <basic:Rule xsi:type="basic:AttributeValueString" value="member" ignoreCase="true" />
                <basic:Rule xsi:type="basic:AttributeValueString" value="affiliate" ignoreCase="true" />
                <basic:Rule xsi:type="basic:AttributeValueString" value="employee" ignoreCase="true" />
                <basic:Rule xsi:type="basic:AttributeValueString" value="library-walk-in" ignoreCase="true" />
            </afp:PermitValueRule>
        </afp:AttributeRule>
        
        <afp:AttributeRule attributeID="commonName">
            <afp:PermitValueRule xsi:type="basic:ANY" />
        </afp:AttributeRule>
        <afp:AttributeRule attributeID="surname">
            <afp:PermitValueRule xsi:type="basic:ANY" />
        </afp:AttributeRule>
        <afp:AttributeRule attributeID="givenName">
            <afp:PermitValueRule xsi:type="basic:ANY" />
        </afp:AttributeRule>
        <afp:AttributeRule attributeID="eduPersonPrincipalName">
            <afp:PermitValueRule xsi:type="basic:ANY" />
        </afp:AttributeRule>
        <afp:AttributeRule attributeID="eduPersonScopedAffiliation">
            <afp:PermitValueRule xsi:type="basic:ANY" />
        </afp:AttributeRule>
        <afp:AttributeRule attributeID="eduPersonPrimaryAffiliation">
            <afp:PermitValueRule xsi:type="basic:ANY" />
        </afp:AttributeRule>
        <afp:AttributeRule attributeID="dfnEduPersonStudyBranch1">
            <afp:PermitValueRule xsi:type="basic:ANY" />
        </afp:AttributeRule>
        <afp:AttributeRule attributeID="dfnEduPersonStudyBranch2">
            <afp:PermitValueRule xsi:type="basic:ANY" />
        </afp:AttributeRule>
        <afp:AttributeRule attributeID="dfnEduPersonStudyBranch3">
            <afp:PermitValueRule xsi:type="basic:ANY" />
        </afp:AttributeRule>
        <afp:AttributeRule attributeID="dfnEduPersonFieldOfStudyString">
            <afp:PermitValueRule xsi:type="basic:ANY" />
        </afp:AttributeRule>
        <afp:AttributeRule attributeID="eduPersonEntitlement">
            <afp:PermitValueRule xsi:type="basic:ANY" />
        </afp:AttributeRule>
        <afp:AttributeRule attributeID="dfnEduPersonCostCenter">
            <afp:PermitValueRule xsi:type="basic:ANY" />
        </afp:AttributeRule>

    </afp:AttributeFilterPolicy>

    <!-- 
        Release eduPersonEntitlement and the permissible values of eduPersonAffiliation
        to three specific SPs
    -->
    <!--
    <afp:AttributeFilterPolicy>
        <afp:PolicyRequirementRule xsi:type="basic:OR">
            <basic:Rule xsi:type="basic:AttributeRequesterString" value="urn:example.org:sp:Portal" />
            <basic:Rule xsi:type="basic:AttributeRequesterString" value="urn:example.org:sp:SIS" />
            <basic:Rule xsi:type="basic:AttributeRequesterString" value="urn:example.org:sp:LMS" />
        </afp:PolicyRequirementRule>

        <afp:AttributeRule attributeID="eduPersonAffiliation">
            <afp:PermitValueRule xsi:type="basic:OR">
                <basic:Rule xsi:type="basic:AttributeValueString" value="faculty" ignoreCase="true" />
                <basic:Rule xsi:type="basic:AttributeValueString" value="student" ignoreCase="true" />
                <basic:Rule xsi:type="basic:AttributeValueString" value="staff" ignoreCase="true" />
                <basic:Rule xsi:type="basic:AttributeValueString" value="alum" ignoreCase="true" />
                <basic:Rule xsi:type="basic:AttributeValueString" value="member" ignoreCase="true" />
                <basic:Rule xsi:type="basic:AttributeValueString" value="affiliate" ignoreCase="true" />
                <basic:Rule xsi:type="basic:AttributeValueString" value="employee" ignoreCase="true" />
                <basic:Rule xsi:type="basic:AttributeValueString" value="library-walk-in" ignoreCase="true" />
            </afp:PermitValueRule>
        </afp:AttributeRule>

    </afp:AttributeFilterPolicy>
    -->

    <!-- 
        Release the given name of the user to our portal service provider
    -->
    <!--
    <afp:AttributeFilterPolicy>
        <afp:PolicyRequirementRule xsi:type="basic:AttributeRequesterString" value="urn:example.org:sp:myPortal" />

        <afp:AttributeRule attributeID="givenName">
            <afp:PermitValueRule xsi:type="basic:ANY" />
        </afp:AttributeRule>
    </afp:AttributeFilterPolicy>
    -->

</afp:AttributeFilterPolicyGroup>
\end{lstlisting}

\subsubsection{/opt/shibboleth-idp/conf/uApprove.xml}
\begin{lstlisting}[language=xml]
<?xml version="1.0" encoding="UTF-8"?>
<beans xmlns="http://www.springframework.org/schema/beans" xmlns:xsi="http://www.w3.org/2001/XMLSchema-instance"
    xmlns:p="http://www.springframework.org/schema/p" xmlns:util="http://www.springframework.org/schema/util"
    xmlns:context="http://www.springframework.org/schema/context"
    xsi:schemaLocation="http://www.springframework.org/schema/beans http://www.springframework.org/schema/beans/spring-beans-2.5.xsd
                           http://www.springframework.org/schema/util http://www.springframework.org/schema/util/spring-util-2.5.xsd
                           http://www.springframework.org/schema/context http://www.springframework.org/schema/context/spring-context-2.5.xsd">


    <context:property-placeholder location="file:/opt/shibboleth-idp/conf/uApprove.properties" />


    <bean id="uApprove.touModule" class="ch.SWITCH.aai.uApprove.tou.ToUModule" init-method="initialize"
        p:enabled="${tou.enabled}" p:auditLogEnabled="false" p:compareContent="false">
        <property name="tou">
            <bean class="ch.SWITCH.aai.uApprove.tou.ToU" init-method="initialize"
                p:version="${tou.version}" p:resource="${tou.resource}" />
        </property>
        <property name="relyingParties">
            <bean class="ch.SWITCH.aai.uApprove.RelyingPartyList"
                p:regularExpressions="${services}" p:blacklist="${services.blacklist}" />
        </property>
        <property name="storage">
            <!-- JDBC Storage, default -->
            <bean class="ch.SWITCH.aai.uApprove.tou.storage.JDBCStorage" init-method="initialize"
                p:dataSource-ref="uApprove.dataSource" p:sqlStatements="classpath:/storage/sql-statements.properties" p:graceful="false" />
            
            <!-- LDAP Storage, uncomment this bean and comment out above JDBC bean to activate LDAP -->
            <!--
            <bean class="ch.SWITCH.aai.uApprove.tou.storage.LDAPStorage"
                p:connection-ref="uApprove.ldapConnection" />
            -->
        </property>
    </bean>

    <bean id="uApprove.attributeReleaseModule" class="ch.SWITCH.aai.uApprove.ar.AttributeReleaseModule" init-method="initialize"
        p:enabled="${ar.enabled}" p:auditLogEnabled="false" p:allowGeneralConsent="${ar.allowGeneralConsent}" p:compareAttributeValues="false">
        <property name="relyingParties">
            <bean class="ch.SWITCH.aai.uApprove.RelyingPartyList"
                p:regularExpressions="${services}" p:blacklist="${services.blacklist}" />
        </property>
        <property name="storage">
            <!-- JDBC Storage, default -->
            <bean class="ch.SWITCH.aai.uApprove.ar.storage.JDBCStorage" init-method="initialize"
                p:dataSource-ref="uApprove.dataSource" p:sqlStatements="classpath:/storage/sql-statements.properties" p:graceful="false" />
            
            <!-- LDAP Storage, uncomment this bean and comment out above JDBC bean to activate LDAP -->
            <!--
            <bean class="ch.SWITCH.aai.uApprove.ar.storage.LDAPStorage"
                p:connection-ref="uApprove.ldapConnection" />
            -->
        </property>
    </bean>

    <!-- JDBC Data Connector, default -->
    <bean id="uApprove.dataSource" class="com.mchange.v2.c3p0.ComboPooledDataSource" destroy-method="close" depends-on="shibboleth.LogbackLogging"
        p:driverClass="${database.driver}" p:jdbcUrl="${database.url}"
        p:user="${database.username}" p:password="${database.password}"
        p:idleConnectionTestPeriod="300" />

    <!-- LDAP Data Connector, uncomment this bean and comment out above JDBC bean to activate LDAP -->
    <!--
    <bean id="uApprove.ldapConnection" class="ch.SWITCH.aai.uApprove.LDAPConnectionDetails"
        p:username="${ldap.username}" p:password="${ldap.password}" p:url="${ldap.url}" p:base="${ldap.base}" p:contextFactory="com.sun.jndi.ldap.LdapCtxFactory" p:authentification="simple" />
    -->

    <bean id="uApprove.viewHelper" class="ch.SWITCH.aai.uApprove.ViewHelper" init-method="initialize"
        p:defaultLocale="${view.defaultLocale}" p:forceDefaultLocale="${view.forceDefaultLocale}"
        p:messagesBase="messages" />

    <bean id="uApprove.samlHelper" class="ch.SWITCH.aai.uApprove.ar.SAMLHelper" init-method="initialize"
        p:attributeAuthority-ref="shibboleth.SAML2AttributeAuthority" p:relyingPartyConfigurationManager-ref="shibboleth.RelyingPartyConfigurationManager">
        <property name="attributeProcessor">
            <bean class="ch.SWITCH.aai.uApprove.ar.AttributeProcessor"
                p:blacklist="${ar.attributes.blacklist}" p:order="${ar.attributes.order}" />
        </property>
    </bean>
</beans>
\end{lstlisting}

\subsubsection{/opt/shibboleth-idp/conf/uApprove.properties}
\begin{lstlisting}
#######################################################################
### General uApprove configuration                                  ###
#######################################################################

#---------------------------------------------------------------------#
# Service Provider                                                    #
#---------------------------------------------------------------------#

# List of service provider entity IDs.
# The entries are interpreted as regular expression.
# http://myregexp.com/ can assist you creating such expressions.
services                    = ^https://.*\.example\.org/shibboleth$ \
                              ^https://sp\.other-example\.org/shibboleth$

# Indicates if the list above should be interpreted as blacklist (true)
# or as whitelist (false). If this value is set to true, users will not
# see uApprove when trying to access matching services. If this value is
# false, users will see uApprove only for the entities matching this list
# but not for others.
services.blacklist          = true

#---------------------------------------------------------------------#
# View and Localization                                               #
#---------------------------------------------------------------------#

# The default locale as 2-letter language code defined in ISO 639.
view.defaultLocale          = de

# Indicates whether the default locale is enforced or not.
view.forceDefaultLocale     = false

#---------------------------------------------------------------------#
# Database configuration                                              #
#---------------------------------------------------------------------#

database.driver             = com.mysql.jdbc.Driver
database.url                = jdbc:mysql://localhost:3306/uApprove?autoReconnect=true
database.username           = uApprove
database.password           = idp

#---------------------------------------------------------------------#
# LDAP configuration                                        	      #
#---------------------------------------------------------------------#
# Please consult the uApprove online manual for information on how to 
# configure uApprove to store data via LDAP instead the default JDBC
#ldap.url                    = ldap://ldap.example.org
#ldap.username               = uid=uapprove_admin, o=userBranch
#ldap.password               = secret
#ldap.base                   = ou=uApprove,dc=example,dc=org

#######################################################################
### Terms of Use configuration                                      ###
#######################################################################

# Indicates whether the Terms Of Use Module is enabled or not.
tou.enabled                 = true

# The Terms of Use version.
tou.version                 = 1.0

# Absolute path to the Terms Of Use HTML file.
# Use file:/path/to/file.html for a terms of use file on the file system.
# An example is provided in /manual/examples/terms-of-use.html.
tou.resource                = classpath:/examples/terms-of-use.html

#######################################################################
### Attribute Release configuration                                 ###
#######################################################################

# Indicates whether the Attribute Release Module is enabled or not.
ar.enabled                  = true

# Indicates whether general consent is allowed or not.
ar.allowGeneralConsent      = true

#---------------------------------------------------------------------#
# Attribute Processing                                                #
#---------------------------------------------------------------------#

# Defines the ordering of the attributes.
ar.attributes.order         = uid \
                              surname \
                              givenName \
                              eduPersonPrincipalName \
                              eduPersonAffiliation \
                              eduPersonEntitlement

# Defines a list of blacklisted attributes.              
ar.attributes.blacklist     = transientId \
                              persistentId \
                              eduPersonTargetedID
                          
\end{lstlisting}

\subsubsection{/home/shib-idp/IdP/shibboleth-identityprovider-2.4.4/src/main/webapp/WEB-INF/web.xml}
\begin{lstlisting}[language=xml]
<?xml version="1.0" encoding="UTF-8"?>
<web-app xmlns="http://java.sun.com/xml/ns/j2ee" xmlns:xsi="http://www.w3.org/2001/XMLSchema-instance" xsi:schemaLocation="http://java.sun.com/xml/ns/j2ee http://java.sun.com/xml/ns/j2ee/web-app_2_4.xsd" version="2.4">

    <display-name>Shibboleth Identity Provider</display-name>

    <!-- Parameter that allows the domain of all cookies to be explicitly set. If not set the domain is left empty
        which means that the cookie will only ever be sent to the IdP host. -->
    <!-- <context-param> <param-name>cookieDomain</param-name> <param-value>example.org</param-value> </context-param> -->

    <!-- Spring 2.0 application context files. Files are loaded in the order they appear with subsequent files overwriting 
        same named beans in previous files. -->
    <context-param>
        <param-name>contextConfigLocation</param-name>
        <param-value>$IDP_HOME$/conf/internal.xml; $IDP_HOME$/conf/service.xml; $IDP_HOME$/conf/uApprove.xml;</param-value>
    </context-param>

    <!-- Spring 2.0 listener used to load up the configuration -->
    <listener>
        <listener-class>org.springframework.web.context.ContextLoaderListener</listener-class>
    </listener>

    <!-- Add IdP SLF4J MDC cleanup filter to all requests -->
    <filter>
        <filter-name>SL4JCleanupFilter</filter-name>
        <filter-class>edu.internet2.middleware.shibboleth.common.log.SLF4JMDCCleanupFilter</filter-class>
    </filter>
    <filter-mapping>
        <filter-name>SL4JCleanupFilter</filter-name>
        <url-pattern>/*</url-pattern>
    </filter-mapping>


    <!-- Add IdP Session object to incoming profile requests -->
    <filter>
        <filter-name>IdPSessionFilter</filter-name>
        <filter-class>edu.internet2.middleware.shibboleth.idp.session.IdPSessionFilter</filter-class>
    </filter>
    <filter-mapping>
        <filter-name>IdPSessionFilter</filter-name>
        <url-pattern>/*</url-pattern>
    </filter-mapping>

    <!-- HTTP headers to every response in order to prevent response caching -->
    <filter>
        <filter-name>IdPNoCacheFilter</filter-name>
        <filter-class>edu.internet2.middleware.shibboleth.idp.util.NoCacheFilter</filter-class>
    </filter>
    <filter-mapping>
        <filter-name>IdPNoCacheFilter</filter-name>
        <url-pattern>/*</url-pattern>
    </filter-mapping>

    <!-- uApprove -->
     <filter>
        <filter-name>uApprove</filter-name>
        <filter-class>ch.SWITCH.aai.uApprove.Intercepter</filter-class>
    </filter>
    <filter-mapping>
        <filter-name>uApprove</filter-name>
        <url-pattern>/profile/Shibboleth/SSO</url-pattern>
        <url-pattern>/profile/SAML1/SOAP/AttributeQuery</url-pattern>
        <url-pattern>/profile/SAML1/SOAP/ArtifactResolution</url-pattern>
        <url-pattern>/profile/SAML2/POST/SSO</url-pattern>
        <url-pattern>/profile/SAML2/POST-SimpleSign/SSO</url-pattern>
        <url-pattern>/profile/SAML2/Redirect/SSO</url-pattern>
        <url-pattern>/profile/SAML2/Unsolicited/SSO</url-pattern>
        <url-pattern>/Authn/UserPassword</url-pattern>
    </filter-mapping>
    
    <servlet>
        <servlet-name>uApprove - Terms Of Use</servlet-name>
        <servlet-class>ch.SWITCH.aai.uApprove.tou.ToUServlet</servlet-class>
    </servlet>

    <servlet-mapping>
        <servlet-name>uApprove - Terms Of Use</servlet-name>
        <url-pattern>/uApprove/TermsOfUse</url-pattern>
    </servlet-mapping>

    <servlet>
        <servlet-name>uApprove - Attribute Release</servlet-name>
        <servlet-class>ch.SWITCH.aai.uApprove.ar.AttributeReleaseServlet</servlet-class>
    </servlet>

    <servlet-mapping>
        <servlet-name>uApprove - Attribute Release</servlet-name>
        <url-pattern>/uApprove/AttributeRelease</url-pattern>
    </servlet-mapping>

    <!-- Profile Request Dispatcher -->
    <servlet>
        <servlet-name>ProfileRequestDispatcher</servlet-name>
        <servlet-class>edu.internet2.middleware.shibboleth.common.profile.ProfileRequestDispatcherServlet</servlet-class>
        <load-on-startup>1</load-on-startup>
    </servlet>

    <servlet-mapping>
        <servlet-name>ProfileRequestDispatcher</servlet-name>
        <url-pattern>/profile/*</url-pattern>
    </servlet-mapping>

    <!-- Authentication Engine Entry Point -->
    <servlet>
        <servlet-name>AuthenticationEngine</servlet-name>
        <servlet-class>edu.internet2.middleware.shibboleth.idp.authn.AuthenticationEngine</servlet-class>

        <!-- Whether public credentials returned by a login handler are retained in the subject. -->
        <!-- <init-param> <param-name>retainSubjectsPublicCredentials</param-name> <param-value>false</param-value> </init-param> -->

        <!-- Whether private credentials returned by a login handler are retained in the subject. -->
        <!-- <init-param> <param-name>retainSubjectsPrivateCredentials</param-name> <param-value>false</param-value> </init-param> -->

        <load-on-startup>2</load-on-startup>

    </servlet>

    <servlet-mapping>
        <servlet-name>AuthenticationEngine</servlet-name>
        <url-pattern>/AuthnEngine</url-pattern>
    </servlet-mapping>

    <!-- Servlet protected by container used for RemoteUser authentication -->
    <servlet>
        <servlet-name>RemoteUserAuthHandler</servlet-name>
        <servlet-class>edu.internet2.middleware.shibboleth.idp.authn.provider.RemoteUserAuthServlet</servlet-class>
        <load-on-startup>3</load-on-startup>
    </servlet>

    <servlet-mapping>
        <servlet-name>RemoteUserAuthHandler</servlet-name>
        <url-pattern>/Authn/RemoteUser</url-pattern>
    </servlet-mapping>

    <!-- Servlet for doing Username/Password authentication -->
    <servlet>
        <servlet-name>UsernamePasswordAuthHandler</servlet-name>
        <servlet-class>edu.internet2.middleware.shibboleth.idp.authn.provider.UsernamePasswordLoginServlet</servlet-class>
        <load-on-startup>3</load-on-startup>
    </servlet>

    <servlet-mapping>
        <servlet-name>UsernamePasswordAuthHandler</servlet-name>
        <url-pattern>/Authn/UserPassword</url-pattern>
    </servlet-mapping>

    <!-- Servlet for displaying IdP status. -->
    <servlet>
        <servlet-name>Status</servlet-name>
        <servlet-class>edu.internet2.middleware.shibboleth.idp.StatusServlet</servlet-class>

        <!-- Space separated list of CIDR blocks allowed to access the status page -->
        <init-param>
            <param-name>AllowedIPs</param-name>
            <param-value>127.0.0.1/32 ::1/128</param-value>
        </init-param>

        <load-on-startup>2</load-on-startup>
    </servlet>

    <servlet-mapping>
        <servlet-name>Status</servlet-name>
        <url-pattern>/status</url-pattern>
    </servlet-mapping>


    <!-- Send request to the EntityID to the SAML metadata handler. -->
    <servlet>
        <servlet-name>shibboleth_jsp</servlet-name>
        <jsp-file>/shibboleth.jsp</jsp-file>
    </servlet>

    <servlet-mapping>
        <servlet-name>shibboleth_jsp</servlet-name>
        <url-pattern>/shibboleth</url-pattern>
    </servlet-mapping>

    <error-page>
        <error-code>500</error-code>
        <location>/error.jsp</location>
    </error-page>

    <error-page>
        <error-code>404</error-code>
        <location>/error-404.jsp</location>
    </error-page>

    <!-- Uncomment to use container managed authentication -->
    <!-- <security-constraint> <display-name>Shibboleth IdP</display-name> <web-resource-collection> <web-resource-name>user 
        authentication</web-resource-name> <url-pattern>/Authn/RemoteUser</url-pattern> <http-method>GET</http-method> <http-method>POST</http-method> 
        </web-resource-collection> <auth-constraint> <role-name>user</role-name> </auth-constraint> <user-data-constraint> <transport-guarantee>CONFIDENTIAL</transport-guarantee> 
        </user-data-constraint> </security-constraint> <security-role> <role-name>user</role-name> </security-role> -->

    <!-- Uncomment if you want BASIC auth managed by the container -->
    <!-- <login-config> <auth-method>BASIC</auth-method> <realm-name>IdP Password Authentication</realm-name> </login-config> -->

    <!-- Uncomment if you want form-based auth managed by the container -->
    <!-- <login-config> <auth-method>FORM</auth-method> <realm-name>IdP Password Authentication</realm-name> <form-login-config> 
        <form-login-page>/login.jsp</form-login-page> <form-error-page>/login-error.jsp</form-error-page> </form-login-config> </login-config> -->

</web-app>
\end{lstlisting}

\subsection{Service Provider}
\subsubsection{/etc/network/interfaces}
\begin{lstlisting}
# interfaces(5) file used by ifup(8) and ifdown(8)
auto lo
iface lo inet loopback

auto eth0
iface eth0 inet dhcp

auto eth1
iface eth1 inet dhcp
\end{lstlisting}

\subsubsection{/etc/dhcp/dhclient.conf}
\begin{lstlisting}
# Configuration file for /sbin/dhclient, which is included in Debian's
#	dhcp3-client package.
#
# This is a sample configuration file for dhclient. See dhclient.conf's
#	man page for more information about the syntax of this file
#	and a more comprehensive list of the parameters understood by
#	dhclient.
#
# Normally, if the DHCP server provides reasonable information and does
#	not leave anything out (like the domain name, for example), then
#	few changes must be made to this file, if any.
#

option rfc3442-classless-static-routes code 121 = array of unsigned integer 8;

#send host-name "andare.fugue.com";
send host-name = gethostname();
#send dhcp-client-identifier 1:0:a0:24:ab:fb:9c;
#send dhcp-lease-time 3600;
supersede domain-name-servers 192.168.100.100;
#supersede domain-name "fugue.com home.vix.com";
#prepend domain-name-servers 127.0.0.1;
request subnet-mask, broadcast-address, time-offset, routers,
	domain-name, domain-name-servers, domain-search, host-name,
	dhcp6.name-servers, dhcp6.domain-search,
	netbios-name-servers, netbios-scope, interface-mtu,
	rfc3442-classless-static-routes, ntp-servers,
	dhcp6.fqdn, dhcp6.sntp-servers;
#require subnet-mask, domain-name-servers;
#timeout 60;
#retry 60;
#reboot 10;
#select-timeout 5;
#initial-interval 2;
#script "/etc/dhcp3/dhclient-script";
#media "-link0 -link1 -link2", "link0 link1";
#reject 192.33.137.209;

#alias {
#  interface "eth0";
#  fixed-address 192.5.5.213;
#  option subnet-mask 255.255.255.255;
#}

#lease {
#  interface "eth0";
#  fixed-address 192.33.137.200;
#  medium "link0 link1";
#  option host-name "andare.swiftmedia.com";
#  option subnet-mask 255.255.255.0;
#  option broadcast-address 192.33.137.255;
#  option routers 192.33.137.250;
#  option domain-name-servers 127.0.0.1;
#  renew 2 2000/1/12 00:00:01;
#  rebind 2 2000/1/12 00:00:01;
#  expire 2 2000/1/12 00:00:01;
#}
\end{lstlisting}

\subsubsection{/etc/apache2/sites-available/sp.conf}
\begin{lstlisting}
<VirtualHost *:443>
  ServerName              sp.shib.lan
  ServerAdmin             webmaster@example.org
  DocumentRoot            /var/www/html
  SSLEngine on
  SSLCertificateFile      /etc/ssl/certs/sp.shib.lan_cert.pem
  SSLCertificateKeyFile   /etc/ssl/private/sp.shib.lan_key.pem
 
  SSLCertificateChainFile /etc/ssl/certs/shib.lan.pem
 
  SSLProtocol All -SSLv2 -SSLv3
  SSLHonorCipherOrder On
  SSLCipherSuite 'ECDH+AESGCM:DH+AESGCM:ECDH+AES256:DH+AES256:ECDH+AES128:DH+AES:RSA+AESGCM:RSA+AES:ECDH+3DES:DH+3DES:RSA+3DES:!aNULL:!eNULL:!LOW:!RC4:!MD5:!EXP:!PSK:!DSS:!SEED:!ECDSA:!CAMELLIA'

  ErrorLog ${APACHE_LOG_DIR}/error.log
  CustomLog ${APACHE_LOG_DIR}/access.log combined

  <Location /shibtest>
    AuthType shibboleth
    ShibRequireSession On
    require valid-user
    DirectoryIndex shib-info.php
  </Location>
 
  # optional (Metadata-Access at entityID-URL)
  Redirect seeother /shibboleth https://sp.shib.lan/Shibboleth.sso/Metadata
 
</VirtualHost>

<VirtualHost *:80>
  ServerAdmin admin@shib.lan
  ServerName sp.shib.lan
  ServerAlias sp.shib.lan 
  Redirect 301 / https://sp.shib.lan/
  ErrorLog ${APACHE_LOG_DIR}/error.log
  CustomLog ${APACHE_LOG_DIR}/access.log combined

  <Location /shibtest>
    Redirect 301 / https://sp.shib.lan/shibtest
  </Location>

</VirtualHost>
\end{lstlisting}

\subsubsection{/etc/shibboleth/shibboleth2.xml}
\begin{lstlisting}[language=xml]
<SPConfig xmlns="urn:mace:shibboleth:2.0:native:sp:config"
    xmlns:conf="urn:mace:shibboleth:2.0:native:sp:config"
    xmlns:saml="urn:oasis:names:tc:SAML:2.0:assertion"
    xmlns:samlp="urn:oasis:names:tc:SAML:2.0:protocol"    
    xmlns:md="urn:oasis:names:tc:SAML:2.0:metadata"
    clockSkew="180">

    <!--
    By default, in-memory StorageService, ReplayCache, ArtifactMap, and SessionCache
    are used. See example-shibboleth2.xml for samples of explicitly configuring them.
    -->

    <!--
    To customize behavior for specific resources on Apache, and to link vhosts or
    resources to ApplicationOverride settings below, use web server options/commands.
    See https://wiki.shibboleth.net/confluence/display/SHIB2/NativeSPConfigurationElements for help.
    
    For examples with the RequestMap XML syntax instead, see the example-shibboleth2.xml
    file, and the https://wiki.shibboleth.net/confluence/display/SHIB2/NativeSPRequestMapHowTo topic.
    -->

    <!-- The ApplicationDefaults element is where most of Shibboleth's SAML bits are defined. -->
    <ApplicationDefaults entityID="https://sp.shib.lan/shibboleth"
                         REMOTE_USER="eppn persistent-id targeted-id uniqueID"
                         signing="back" requireTransportAuth="false">

        <!--
        Controls session lifetimes, address checks, cookie handling, and the protocol handlers.
        You MUST supply an effectively unique handlerURL value for each of your applications.
        The value defaults to /Shibboleth.sso, and should be a relative path, with the SP computing
        a relative value based on the virtual host. Using handlerSSL="true", the default, will force
        the protocol to be https. You should also set cookieProps to "https" for SSL-only sites.
        Note that while we default checkAddress to "false", this has a negative impact on the
        security of your site. Stealing sessions via cookie theft is much easier with this disabled.
        -->
        <Sessions lifetime="28800" timeout="3600" relayState="ss:mem"
                  checkAddress="false" consistentAddress="true" handlerSSL="true" cookieProps="https">

            <!--
            Configures SSO for a default IdP. To allow for >1 IdP, remove
            entityID property and adjust discoveryURL to point to discovery service.
            (Set discoveryProtocol to "WAYF" for legacy Shibboleth WAYF support.)
            You can also override entityID on /Login query string, or in RequestMap/htaccess.
            -->
            <SSO entityID="https://idp.shib.lan/idp/shibboleth">
                SAML2 
            </SSO>

            <!-- SAML and local-only logout. -->
            <Logout>SAML2 Local</Logout>
            
            <!-- Extension service that generates "approximate" metadata based on SP configuration. -->
            <Handler type="MetadataGenerator" Location="/Metadata" signing="false"/>

            <!-- Status reporting service. -->
            <Handler type="Status" Location="/Status" acl="127.0.0.1 ::1"/>

            <!-- Session diagnostic service. -->
            <Handler type="Session" Location="/Session" showAttributeValues="true"/>

            <!-- JSON feed of discovery information. -->
            <Handler type="DiscoveryFeed" Location="/DiscoFeed"/>
        </Sessions>

        <!--
        Allows overriding of error template information/filenames. You can
        also add attributes with values that can be plugged into the templates.
        -->
        <Errors supportContact="root@localhost"
            helpLocation="/about.html"
            styleSheet="/shibboleth-sp/main.css"/>
        
        <MetadataProvider type="XML" url="https://idp.shib.lan/idp/shibboleth" />

        <!-- Example of remotely supplied batch of signed metadata. -->
        <!--
        <MetadataProvider type="XML" uri="http://federation.org/federation-metadata.xml"
              backingFilePath="federation-metadata.xml" reloadInterval="7200">
            <MetadataFilter type="RequireValidUntil" maxValidityInterval="2419200"/>
            <MetadataFilter type="Signature" certificate="fedsigner.pem"/>
        </MetadataProvider>
        -->

        <!-- Example of locally maintained metadata. -->
        <!--
        <MetadataProvider type="XML" file="partner-metadata.xml"/>
        -->

        <!-- Map to extract attributes from SAML assertions. -->
        <AttributeExtractor type="XML" validate="true" reloadChanges="false" path="attribute-map.xml"/>
        
        <!-- Use a SAML query if no attributes are supplied during SSO. -->
        <AttributeResolver type="Query" subjectMatch="true"/>

        <!-- Default filtering policy for recognized attributes, lets other data pass. -->
        <AttributeFilter type="XML" validate="true" path="attribute-policy.xml"/>

        <!-- Simple file-based resolver for using a single keypair. -->
        <CredentialResolver type="File" key="/etc/ssl/private/sp.shib.lan_key.pem" certificate="/etc/ssl/certs/sp.shib.lan_cert.pem"/>

        <!--
        The default settings can be overridden by creating ApplicationOverride elements (see
        the https://wiki.shibboleth.net/confluence/display/SHIB2/NativeSPApplicationOverride topic).
        Resource requests are mapped by web server commands, or the RequestMapper, to an
        applicationId setting.
        
        Example of a second application (for a second vhost) that has a different entityID.
        Resources on the vhost would map to an applicationId of "admin":
        -->
        <!--
        <ApplicationOverride id="admin" entityID="https://admin.example.org/shibboleth"/>
        -->
    </ApplicationDefaults>
    
    <!-- Policies that determine how to process and authenticate runtime messages. -->
    <SecurityPolicyProvider type="XML" validate="true" path="security-policy.xml"/>

    <!-- Low-level configuration about protocols and bindings available for use. -->
    <ProtocolProvider type="XML" validate="true" reloadChanges="false" path="protocols.xml"/>

</SPConfig>
\end{lstlisting}

\subsubsection{/etc/shibboleth/attribute-map.xml}
\begin{lstlisting}[language=xml]
<Attributes xmlns="urn:mace:shibboleth:2.0:attribute-map" xmlns:xsi="http://www.w3.org/2001/XMLSchema-instance">

    <!--
    The mappings are a mix of SAML 1.1 and SAML 2.0 attribute names agreed to within the Shibboleth
    community. The non-OID URNs are SAML 1.1 names and most of the OIDs are SAML 2.0 names, with a
    few exceptions for newer attributes where the name is the same for both versions. You will
    usually want to uncomment or map the names for both SAML versions as a unit.
    -->
    
    <!-- First some useful eduPerson attributes that many sites might use. -->
    
    <Attribute name="urn:mace:dir:attribute-def:eduPersonPrincipalName" id="eppn">
        <AttributeDecoder xsi:type="ScopedAttributeDecoder"/>
    </Attribute>
    <Attribute name="urn:oid:1.3.6.1.4.1.5923.1.1.1.6" id="eppn">
        <AttributeDecoder xsi:type="ScopedAttributeDecoder"/>
    </Attribute>
    
    <Attribute name="urn:mace:dir:attribute-def:eduPersonScopedAffiliation" id="affiliation">
        <AttributeDecoder xsi:type="ScopedAttributeDecoder" caseSensitive="false"/>
    </Attribute>
    <Attribute name="urn:oid:1.3.6.1.4.1.5923.1.1.1.9" id="affiliation">
        <AttributeDecoder xsi:type="ScopedAttributeDecoder" caseSensitive="false"/>
    </Attribute>
    
    <Attribute name="urn:mace:dir:attribute-def:eduPersonAffiliation" id="unscoped-affiliation">
        <AttributeDecoder xsi:type="StringAttributeDecoder" caseSensitive="false"/>
    </Attribute>
    <Attribute name="urn:oid:1.3.6.1.4.1.5923.1.1.1.1" id="unscoped-affiliation">
        <AttributeDecoder xsi:type="StringAttributeDecoder" caseSensitive="false"/>
    </Attribute>
    
    <Attribute name="urn:mace:dir:attribute-def:eduPersonEntitlement" id="entitlement"/>
    <Attribute name="urn:oid:1.3.6.1.4.1.5923.1.1.1.7" id="entitlement"/>

    <!-- A persistent id attribute that supports personalized anonymous access. -->
    
    <!-- First, the deprecated/incorrect version, decoded as a scoped string: -->
    <Attribute name="urn:mace:dir:attribute-def:eduPersonTargetedID" id="targeted-id">
        <AttributeDecoder xsi:type="ScopedAttributeDecoder"/>
        <!-- <AttributeDecoder xsi:type="NameIDFromScopedAttributeDecoder" formatter="$NameQualifier!$SPNameQualifier!$Name" defaultQualifiers="true"/> -->
    </Attribute>
    
    <!-- Second, an alternate decoder that will decode the incorrect form into the newer form. -->
    <!--
    <Attribute name="urn:mace:dir:attribute-def:eduPersonTargetedID" id="persistent-id">
        <AttributeDecoder xsi:type="NameIDFromScopedAttributeDecoder" formatter="$NameQualifier!$SPNameQualifier!$Name" defaultQualifiers="true"/>
    </Attribute>
    -->
    
    <!-- Third, the new version (note the OID-style name): -->
    <Attribute name="urn:oid:1.3.6.1.4.1.5923.1.1.1.10" id="persistent-id">
        <AttributeDecoder xsi:type="NameIDAttributeDecoder" formatter="$NameQualifier!$SPNameQualifier!$Name" defaultQualifiers="true"/>
    </Attribute>

    <!-- Fourth, the SAML 2.0 NameID Format: -->
    <Attribute name="urn:oasis:names:tc:SAML:2.0:nameid-format:persistent" id="persistent-id">
        <AttributeDecoder xsi:type="NameIDAttributeDecoder" formatter="$NameQualifier!$SPNameQualifier!$Name" defaultQualifiers="true"/>
    </Attribute>
    
    <!-- Some more eduPerson attributes, uncomment these to use them... -->
    
    <Attribute name="urn:mace:dir:attribute-def:eduPersonPrimaryAffiliation" id="primary-affiliation">
        <AttributeDecoder xsi:type="StringAttributeDecoder" caseSensitive="false"/>
    </Attribute>
    <Attribute name="urn:oid:1.3.6.1.4.1.5923.1.1.1.5" id="primary-affiliation">
        <AttributeDecoder xsi:type="StringAttributeDecoder" caseSensitive="false"/>
    </Attribute>
    <!--
    <Attribute name="urn:mace:dir:attribute-def:eduPersonNickname" id="nickname"/>
    <Attribute name="urn:mace:dir:attribute-def:eduPersonPrimaryOrgUnitDN" id="primary-orgunit-dn"/>
    <Attribute name="urn:mace:dir:attribute-def:eduPersonOrgUnitDN" id="orgunit-dn"/>
    <Attribute name="urn:mace:dir:attribute-def:eduPersonOrgDN" id="org-dn"/>

    <Attribute name="urn:oid:1.3.6.1.4.1.5923.1.1.1.5" id="primary-affiliation">
        <AttributeDecoder xsi:type="StringAttributeDecoder" caseSensitive="false"/>
    </Attribute>
    <Attribute name="urn:oid:1.3.6.1.4.1.5923.1.1.1.2" id="nickname"/>
    <Attribute name="urn:oid:1.3.6.1.4.1.5923.1.1.1.8" id="primary-orgunit-dn"/>
    <Attribute name="urn:oid:1.3.6.1.4.1.5923.1.1.1.4" id="orgunit-dn"/>
    <Attribute name="urn:oid:1.3.6.1.4.1.5923.1.1.1.3" id="org-dn"/>

    <Attribute name="urn:oid:1.3.6.1.4.1.5923.1.1.1.11" id="assurance"/>
    
    <Attribute name="urn:oid:1.3.6.1.4.1.5923.1.5.1.1" id="member"/>
    
    <Attribute name="urn:oid:1.3.6.1.4.1.5923.1.6.1.1" id="eduCourseOffering"/>
    <Attribute name="urn:oid:1.3.6.1.4.1.5923.1.6.1.2" id="eduCourseMember"/>
    -->

    <!-- DFN Attribute dfnEduPerson -->
    <Attribute name="urn:mace:dir:attribute-def:dfnEduPersonCostCenter" id="dfnEduPersonCostCenter">
        <AttributeDecoder xsi:type="StringAttributeDecoder" caseSensitive="false"/>
    </Attribute>
    <Attribute name="urn:oid:1.3.6.1.4.1.22177.400.1.1.3.1" id="dfnEduPersonCostCenter">
        <AttributeDecoder xsi:type="StringAttributeDecoder" caseSensitive="false"/>
    </Attribute>
    
    <Attribute name="urn:mace:dir:attribute-def:dfnEduPersonStudyBranch1" id="dfnEduPersonStudyBranch1">
        <AttributeDecoder xsi:type="StringAttributeDecoder" caseSensitive="false"/>
    </Attribute>
    <Attribute name="urn:oid:1.3.6.1.4.1.22177.400.1.1.3.2" id="dfnEduPersonStudyBranch1">
        <AttributeDecoder xsi:type="StringAttributeDecoder" caseSensitive="false"/>
    </Attribute>
    
    <Attribute name="urn:mace:dir:attribute-def:dfnEduPersonStudyBranch2" id="dfnEduPersonStudyBranch2">
        <AttributeDecoder xsi:type="StringAttributeDecoder" caseSensitive="false"/>
    </Attribute>
    <Attribute name="urn:oid:1.3.6.1.4.1.22177.400.1.1.3.3" id="dfnEduPersonStudyBranch2">
        <AttributeDecoder xsi:type="StringAttributeDecoder" caseSensitive="false"/>
    </Attribute>
    
    <Attribute name="urn:mace:dir:attribute-def:dfnEduPersonStudyBranch3" id="dfnEduPersonStudyBranch3">
        <AttributeDecoder xsi:type="StringAttributeDecoder" caseSensitive="false"/>
    </Attribute>
    <Attribute name="urn:oid:1.3.6.1.4.1.22177.400.1.1.3.4" id="dfnEduPersonStudyBranch3">
        <AttributeDecoder xsi:type="StringAttributeDecoder" caseSensitive="false"/>
    </Attribute>
    
    <Attribute name="urn:mace:dir:attribute-def:dfnEduPersonFieldOfStudyString" id="dfnEduPersonFieldOfStudyString">
        <AttributeDecoder xsi:type="StringAttributeDecoder" caseSensitive="false"/>
    </Attribute>
    <Attribute name="urn:oid:1.3.6.1.4.1.22177.400.1.1.3.5" id="dfnEduPersonFieldOfStudyString">
        <AttributeDecoder xsi:type="StringAttributeDecoder" caseSensitive="false"/>
    </Attribute>
    
    <!--
    <Attribute name="urn:mace:dir:attribute-def:dfnEduPersonFinalDegree" id="dfnEduPersonFinalDegree">
        <AttributeDecoder xsi:type="StringAttributeDecoder" caseSensitive="false"/>
    </Attribute>
    <Attribute name="urn:oid:1.3.6.1.4.1.22177.400.1.1.3.6" id="dfnEduPersonFinalDegree">
        <AttributeDecoder xsi:type="StringAttributeDecoder" caseSensitive="false"/>
    </Attribute>

    <Attribute name="urn:mace:dir:attribute-def:dfnEduPersonTypeOfStudy" id="dfnEduPersonTypeOfStudy">
        <AttributeDecoder xsi:type="StringAttributeDecoder" caseSensitive="false"/>
    </Attribute>
    <Attribute name="urn:oid:1.3.6.1.4.1.22177.400.1.1.3.7" id="dfnEduPersonTypeOfStudy">
        <AttributeDecoder xsi:type="StringAttributeDecoder" caseSensitive="false"/>
    </Attribute>

    <Attribute name="urn:mace:dir:attribute-def:dfnEduPersonTermsOfStudy" id="dfnEduPersonTermsOfStudy">
        <AttributeDecoder xsi:type="StringAttributeDecoder" caseSensitive="false"/>
    </Attribute>
    <Attribute name="urn:oid:1.3.6.1.4.1.22177.400.1.1.3.8" id="dfnEduPersonTermsOfStudy">
        <AttributeDecoder xsi:type="StringAttributeDecoder" caseSensitive="false"/>
    </Attribute>

    <Attribute name="urn:mace:dir:attribute-def:dfnEduPersonBranchAndDegree" id="dfnEduPersonBranchAndDegree">
        <AttributeDecoder xsi:type="StringAttributeDecoder" caseSensitive="false"/>
    </Attribute>
    <Attribute name="urn:oid:1.3.6.1.4.1.22177.400.1.1.3.9" id="dfnEduPersonBranchAndDegree">
        <AttributeDecoder xsi:type="StringAttributeDecoder" caseSensitive="false"/>
    </Attribute>

    <Attribute name="urn:mace:dir:attribute-def:dfnEduPersonBrachAndType" id="dfnEduPersonBrachAndType">
        <AttributeDecoder xsi:type="StringAttributeDecoder" caseSensitive="false"/>
    </Attribute>
    <Attribute name="urn:oid:1.3.6.1.4.1.22177.400.1.1.3.10" id="dfnEduPersonBrachAndType">
        <AttributeDecoder xsi:type="StringAttributeDecoder" caseSensitive="false"/>
    </Attribute>

    <Attribute name="urn:mace:dir:attribute-def:dfnEduPersonFeaturesOfStudy" id="dfnEduPersonFeaturesOfStudy">
        <AttributeDecoder xsi:type="StringAttributeDecoder" caseSensitive="false"/>
    </Attribute>
    <Attribute name="urn:oid:1.3.6.1.4.1.22177.400.1.1.3.11" id="dfnEduPersonFeaturesOfStudy">
        <AttributeDecoder xsi:type="StringAttributeDecoder" caseSensitive="false"/>
    </Attribute>
    -->

    <!-- Examples of LDAP-based attributes, uncomment to use these... -->
    
    <Attribute name="urn:mace:dir:attribute-def:cn" id="cn"/>
    <Attribute name="urn:mace:dir:attribute-def:sn" id="sn"/>
    <Attribute name="urn:mace:dir:attribute-def:givenName" id="givenName"/>
    <!--
    <Attribute name="urn:mace:dir:attribute-def:displayName" id="displayName"/>
    <Attribute name="urn:mace:dir:attribute-def:mail" id="mail"/>
    <Attribute name="urn:mace:dir:attribute-def:telephoneNumber" id="telephoneNumber"/>
    <Attribute name="urn:mace:dir:attribute-def:title" id="title"/>
    <Attribute name="urn:mace:dir:attribute-def:initials" id="initials"/>
    <Attribute name="urn:mace:dir:attribute-def:description" id="description"/>
    <Attribute name="urn:mace:dir:attribute-def:carLicense" id="carLicense"/>
    <Attribute name="urn:mace:dir:attribute-def:departmentNumber" id="departmentNumber"/>
    <Attribute name="urn:mace:dir:attribute-def:employeeNumber" id="employeeNumber"/>
    <Attribute name="urn:mace:dir:attribute-def:employeeType" id="employeeType"/>
    <Attribute name="urn:mace:dir:attribute-def:preferredLanguage" id="preferredLanguage"/>
    <Attribute name="urn:mace:dir:attribute-def:manager" id="manager"/>
    <Attribute name="urn:mace:dir:attribute-def:seeAlso" id="seeAlso"/>
    <Attribute name="urn:mace:dir:attribute-def:facsimileTelephoneNumber" id="facsimileTelephoneNumber"/>
    <Attribute name="urn:mace:dir:attribute-def:street" id="street"/>
    <Attribute name="urn:mace:dir:attribute-def:postOfficeBox" id="postOfficeBox"/>
    <Attribute name="urn:mace:dir:attribute-def:postalCode" id="postalCode"/>
    <Attribute name="urn:mace:dir:attribute-def:st" id="st"/>
    <Attribute name="urn:mace:dir:attribute-def:l" id="l"/>
    <Attribute name="urn:mace:dir:attribute-def:o" id="o"/>
    <Attribute name="urn:mace:dir:attribute-def:ou" id="ou"/>
    <Attribute name="urn:mace:dir:attribute-def:businessCategory" id="businessCategory"/>
    <Attribute name="urn:mace:dir:attribute-def:physicalDeliveryOfficeName" id="physicalDeliveryOfficeName"/>

    -->
    <Attribute name="urn:oid:2.5.4.3" id="cn"/>
    <Attribute name="urn:oid:2.5.4.4" id="sn"/>
    <Attribute name="urn:oid:2.5.4.42" id="givenName"/>
    <!--
    <Attribute name="urn:oid:2.16.840.1.113730.3.1.241" id="displayName"/>
    <Attribute name="urn:oid:0.9.2342.19200300.100.1.3" id="mail"/>
    <Attribute name="urn:oid:2.5.4.20" id="telephoneNumber"/>
    <Attribute name="urn:oid:2.5.4.12" id="title"/>
    <Attribute name="urn:oid:2.5.4.43" id="initials"/>
    <Attribute name="urn:oid:2.5.4.13" id="description"/>
    <Attribute name="urn:oid:2.16.840.1.113730.3.1.1" id="carLicense"/>
    <Attribute name="urn:oid:2.16.840.1.113730.3.1.2" id="departmentNumber"/>
    <Attribute name="urn:oid:2.16.840.1.113730.3.1.3" id="employeeNumber"/>
    <Attribute name="urn:oid:2.16.840.1.113730.3.1.4" id="employeeType"/>
    <Attribute name="urn:oid:2.16.840.1.113730.3.1.39" id="preferredLanguage"/>
    <Attribute name="urn:oid:0.9.2342.19200300.100.1.10" id="manager"/>
    <Attribute name="urn:oid:2.5.4.34" id="seeAlso"/>
    <Attribute name="urn:oid:2.5.4.23" id="facsimileTelephoneNumber"/>
    <Attribute name="urn:oid:2.5.4.9" id="street"/>
    <Attribute name="urn:oid:2.5.4.18" id="postOfficeBox"/>
    <Attribute name="urn:oid:2.5.4.17" id="postalCode"/>
    <Attribute name="urn:oid:2.5.4.8" id="st"/>
    <Attribute name="urn:oid:2.5.4.7" id="l"/>
    <Attribute name="urn:oid:2.5.4.10" id="o"/>
    <Attribute name="urn:oid:2.5.4.11" id="ou"/>
    <Attribute name="urn:oid:2.5.4.15" id="businessCategory"/>
    <Attribute name="urn:oid:2.5.4.19" id="physicalDeliveryOfficeName"/>
    -->

</Attributes>
\end{lstlisting}


\end{document}
