\section{Installation von Identity Provider und Service Provider}
\subsection{VirtualBox}
Da der IdP die IPs per DHCP im VM-Netz übernehmen soll, muss der
VirtualBox-interne DHCP Server für das Host-Only-Netzwerk deaktiviert werden.
Laufen bereits VMs auf dem Host-System, die den VirtualBox-DHCP für das interne
Netz nutzen, sollte ein neues Host-Only-Netzwerk angelegt werden:\newline
Datei $\rightarrow$ Globale EInstellungen $\rightarrow$ Netzwerk $\rightarrow$
Reiter "`Host-Only-Netzwerke"' $\rightarrow$ Host-Only-Netzwerk hinzufügen
(kleine Buttons rechts) bzw. für den vorhandenen den DHCP-Server deaktivieren
(über "`Host-Only-Netzwerk ändern"'-Button) $\rightarrow$ IP-Adresse des
Adapters: 192.168.100.1/24
\begin{figure}[h!]
  \centering
    \includegraphics[width=0.75\textwidth]{img/vbox-network-global.png}
    \caption{Host-Only-Netzwerk - globale Einstellungen VirtualBox}
\end{figure}

Danach ist natürlich darauf zu achten, dass die beiden Shibboleth-VMs als
zweiten Netzwerkadapter (neben dem NAT-Adapter) noch einen "`Host-only Adapter"'
erhalten. Dieser muss als virtuellen Host-Adapter natürlich den gerade in den
globalen Einstellungen konfigurierten Host-Only-Adapter nutzen.
\begin{figure}[h!]
  \centering
    \includegraphics[width=0.75\textwidth]{img/vbox-network-vm.png}
    \caption{Host-Only-Netzwerk in den VMs}
\end{figure}

\newpage
\subsection{Grundkonfiguration auf beiden Systemen}
\textcolor{red}{\textbf{Kennwörter für sudo (siehe auch Kapitel
\ref{sec:spickzettel} \nameref{sec:spickzettel}):
\newline IDP: idp\newline SP: sp}}

PPA für SublimeText3 hinzufügen aktuelle Patches einspielen (hier für shibidp):
\begin{lstlisting}
shib-idp@shibidp:~$ sudo add-apt-repository ppa:webupd8team/sublime-text-3
shib-idp@shibidp:~$ sudo apt-get update
shib-idp@shibidp:~$ sudo apt-get upgrade
\end{lstlisting}

Pakete installieren:
\begin{lstlisting}
shib-idp@shibidp:~$ sudo apt-get install chromium-browser openssh-server apache2 php5 php5-mysql mysql-server vim ntp sublime-text-installer
\end{lstlisting}
Für MySQL wird das MySQL-root-Passwort abgefragt, hier wurde "`idp"' gewählt.
\newline Optional kann noch der SSH-Server angepasst werden, die Config findet
sich unter \mbox{/etc/ssh/sshd\_config}. Die Grundkonfiguration ist aber
ausreichend.
\subsection{Identity Provider - Teil 1}
Im ersten Teil des IdP-Setups werden DNS, DHCP, LDAP, CA und der Shibboleth IdP
installiert und konfiguriert. Um die Basiskonfiguration zu testen wird
anschließend der SP aufgesetzt. Im zweiten Teil des IdP-Setups erfolgt dann der
Feinschliff und die Konfiguration des Attribute-Resolvers und -Filters.
\subsubsection{DNS}
\step{/etc/network/interfaces anpassen (KEIN Gateway für eth1 setzen, alle was
  nicht nach 192.168.100.0 geht wird über NAT geroutet. Das Gateway wird per
  DHCP vom VM-Wirtsrechner verteilt!):}
\begin{lstlisting}
shib-idp@shibidp:~$ cat /etc/network/interfaces
	# interfaces(5) file used by ifup(8) and ifdown(8)
	auto lo
	iface lo inet loopback
		auto eth0
	iface eth0 inet dhcp
		auto eth1
	iface eth1 inet static
		address 192.168.100.100
		netmask 255.255.255.0
		dns-nameservers 192.168.100.100
		dns-search shib.lan
\end{lstlisting}
\step{bind9 installieren und konfigurieren}
\begin{lstlisting}
shib-idp@shibidp:~$ sudo apt-get install bind9
shib-idp@shibidp:~$ sudo service bind9 stop
\end{lstlisting}
\step{da wir nur IPv4 nutzen kann eibe "`-4"' in der bind-config in die Optionen
übernommen werden:}
\begin{lstlisting}
shib-idp@shibidp:~$ cat /etc/default/bind9
	# run resolvconf?
	RESOLVCONF=no

	# startup options for the server
	OPTIONS="-u bind -4"
\end{lstlisting}
\step{Zonendateien für Forward-Lookup erstellen}
\begin{lstlisting}
shib-idp@shibidp:~$ cat /etc/bind/db.shib.lan
	;; db.shib.lan
	;; Forwardlookupzone für domainname
	;;
	$TTL 2D
	@       IN      SOA     shib.lan. mail.shib.lan. (
							2015051801      ; Serial
									8H      ; Refresh
									2H      ; Retry
									4W      ; Expire
									3H )    ; NX (TTL Negativ Cache)
	@                               IN      NS      shib.lan.
									IN      MX      10 mailserver.shib.lan.
									IN      A       192.168.100.100
	idp                             IN      A       192.168.100.100
	localhost                       IN      A       127.0.0.1
	sp                              IN      A       192.168.100.200
	www                             IN      A       192.168.100.100

\end{lstlisting}
\step{Zonendateien für Reverse-Lookup erstellen}
\begin{lstlisting}
shib-idp@shibidp:~$ cat /etc/bind/db.100.168.192
	;; db.100.168.192
	;; Reverselookupzone für shib.lan
	;;
	$TTL 2D
	@       IN      SOA     shib.lan. mail.shib-idp.lan. (
									2015051801      ; Serial
											8H      ; Refresh
											2H      ; Retry
											4W      ; Expire
											2D )    ; TTL Negative Cache

	@       IN      NS      shib.lan.

	100     IN      PTR     shib.lan.
	200     IN      PTR     sp.shib.lan.
\end{lstlisting}

\step{neuen DNS "`bekannt machen"':}
\begin{lstlisting}
shib-idp@shibidp:~$ cat /etc/bind/named.conf.local
	//
	// Do any local configuration here
	//

	// Consider adding the 1918 zones here, if they are not used in your
	// organization
	//include "/etc/bind/zones.rfc1918";

	zone "shib.lan" {
	type master;
	file "/etc/bind/db.shib.lan";
	};

	zone "100.168.192.in-addr.arpa" {
	type master;
	file "/etc/bind/db.100.168.192";
\end{lstlisting}
\step{lokalen Nameserver als Default festlegen (immer der erste, der abgefragt
wird, head wird immer zuerst eingebunden in die resolv.conf):}
\begin{lstlisting}
shib-idp@shibidp:~$ cat /etc/resolvconf/resolv.conf.d/head
	# Dynamic resolv.conf(5) file for glibc resolver(3) generated by resolvconf(8)
	#     DO NOT EDIT THIS FILE BY HAND -- YOUR CHANGES WILL BE OVERWRITTEN
	nameserver 192.168.100.100
\end{lstlisting}
\step{DNS-Weiterleitung aktivieren (forwarders { ... } einkommentieren, 8.8.8.8
(google) ggf. durch lokalen DNS (der Uni) ersetzen):}
\begin{lstlisting}
shib-idp@shibidp:~$ cat /etc/bind/named.conf.options
	options {
		directory "/var/cache/bind";

		// If there is a firewall between you and nameservers you want
		// to talk to, you may need to fix the firewall to allow multiple
		// ports to talk.  See http://www.kb.cert.org/vuls/id/800113

		// If your ISP provided one or more IP addresses for stable
		// nameservers, you probably want to use them as forwarders.
		// Uncomment the following block, and insert the addresses replacing
		// the all-0's placeholder.

		forwarders {
			8.8.8.8;
		};

		//========================================================================
		// If BIND logs error messages about the root key being expired,
		// you will need to update your keys.  See https://www.isc.org/bind-keys
		//========================================================================
		dnssec-validation auto;

		auth-nxdomain no;    # conform to RFC1035
		listen-on-v6 { any; };
	};
\end{lstlisting}
\step{entweder jetzt einfach das System neu durchstarten (sudo reboot now)
oder:}
\begin{lstlisting}
shib-idp@shibidp:~$ sudo ifdown eth0
shib-idp@shibidp:~$ sudo ifdown eth1
shib-idp@shibidp:~$ sudo ifup eth0
shib-idp@shibidp:~$ sudo ifup eth1
shib-idp@shibidp:~$ sudo service bind9 restart
shib-idp@shibidp:~$ sudo resolvconf -u
\end{lstlisting}
\step{Config checken:}
Routing sollte so aussehen (10er Netz ist das NAT vom Wirt, kann ggf. anders sein):
\begin{lstlisting}
shib-idp@shibidp:~$ route -n
	Kernel-IP-Routentabelle
	Ziel            Router          Genmask         Flags Metric Ref    Use Iface
	0.0.0.0         10.0.2.2        0.0.0.0         UG    0      0        0 eth0
	10.0.2.0        0.0.0.0         255.255.255.0   U     0      0        0 eth0
	169.254.0.0     0.0.0.0         255.255.0.0     U     1000   0        0 eth1
	192.168.100.0    0.0.0.0         255.255.255.0   U     0      0        0 eth1
\end{lstlisting}
Nameserver-Config (der zweite nameserver 192... Eintrag kommt aus der interfaces-Datei, kann dort ggf. auch weggelassen werden):
\begin{lstlisting}
shib-idp@shibidp:~$ cat /etc/resolv.conf
	# Dynamic resolv.conf(5) file for glibc resolver(3) generated by resolvconf(8)
	#     DO NOT EDIT THIS FILE BY HAND -- YOUR CHANGES WILL BE OVERWRITTEN
	nameserver 192.168.100.100
	nameserver 139.20.64.1
	nameserver 139.20.64.2
	nameserver 192.168.100.100
	search shib.lan
\end{lstlisting}
DNS-Auflösung mit dig testen (Ergebnisse sollten zumindest ähnlich sein, was auf jeden Fall funktionieren muss ist die Auflösung der Domain in eine IP):
 Test nach "`außen"' über NAT:
\begin{lstlisting}
shib-idp@shibidp:~$ dig www.linux.org
	; <<>> DiG 9.9.5-3ubuntu0.2-Ubuntu <<>> www.linux.org
	;; global options: +cmd
	;; Got answer:
	;; ->>HEADER<<- opcode: QUERY, status: NOERROR, id: 9156
	;; flags: qr rd ra; QUERY: 1, ANSWER: 2, AUTHORITY: 13, ADDITIONAL: 1

	;; OPT PSEUDOSECTION:
	; EDNS: version: 0, flags:; udp: 4096
	;; QUESTION SECTION:
	;www.linux.org.			IN	A

	;; ANSWER SECTION:
	www.linux.org.		12488	IN	CNAME	linux.org.
	linux.org.		1055	IN	A	107.170.40.56

	;; AUTHORITY SECTION:
	.			927	IN	NS	g.root-servers.net.
	.			927	IN	NS	i.root-servers.net.
	.			927	IN	NS	m.root-servers.net.
	.			927	IN	NS	b.root-servers.net.
	.			927	IN	NS	c.root-servers.net.
	.			927	IN	NS	j.root-servers.net.
	.			927	IN	NS	d.root-servers.net.
	.			927	IN	NS	f.root-servers.net.
	.			927	IN	NS	a.root-servers.net.
	.			927	IN	NS	h.root-servers.net.
	.			927	IN	NS	l.root-servers.net.
	.			927	IN	NS	k.root-servers.net.
	.			927	IN	NS	e.root-servers.net.

	;; Query time: 77 msec
	;; SERVER: 192.168.56.150#53(192.168.56.150)
	;; WHEN: Tue May 19 11:54:52 CEST 2015
	;; MSG SIZE  rcvd: 283

\end{lstlisting}
Test des lokalen Netzes (shib.lan)
\begin{lstlisting}
shib-idp@shibidp:~$ dig idp.shib.lan
	; <<>> DiG 9.9.5-3ubuntu0.2-Ubuntu <<>> idp.shib.lan
	;; global options: +cmd
	;; Got answer:
	;; ->>HEADER<<- opcode: QUERY, status: NOERROR, id: 57983
	;; flags: qr aa rd ra; QUERY: 1, ANSWER: 1, AUTHORITY: 1, ADDITIONAL: 2

	;; OPT PSEUDOSECTION:
	; EDNS: version: 0, flags:; udp: 4096
	;; QUESTION SECTION:
	;idp.shib.lan.			IN	A

	;; ANSWER SECTION:
	idp.shib.lan.		172800	IN	A	192.168.100.100

	;; AUTHORITY SECTION:
	shib.lan.		172800	IN	NS	shib.lan.

	;; ADDITIONAL SECTION:
	shib.lan.		172800	IN	A	192.168.100.100

	;; Query time: 1 msec
	;; SERVER: 192.168.100.100#53(192.168.100.100)
	;; WHEN: Tue May 19 11:56:47 CEST 2015
	;; MSG SIZE  rcvd: 87
\end{lstlisting}
Reverse-Lookup (idp.shib-lan hat im Beispiel 192.168.100.100)
\begin{lstlisting}
shib-idp@shibidp:~$ dig -x 192.168.100.100
	; <<>> DiG 9.9.5-3ubuntu0.2-Ubuntu <<>> -x 192.168.100.100
	;; global options: +cmd
	;; Got answer:
	;; ->>HEADER<<- opcode: QUERY, status: NOERROR, id: 341
	;; flags: qr aa rd ra; QUERY: 1, ANSWER: 1, AUTHORITY: 1, ADDITIONAL: 2

	;; OPT PSEUDOSECTION:
	; EDNS: version: 0, flags:; udp: 4096
	;; QUESTION SECTION:
	;100.100.168.192.in-addr.arpa.	IN	PTR

	;; ANSWER SECTION:
	100.100.168.192.in-addr.arpa. 172800 IN	PTR	shib.lan.

	;; AUTHORITY SECTION:
	100.168.192.in-addr.arpa. 172800	IN	NS	shib.lan.

	;; ADDITIONAL SECTION:
	shib.lan.		172800	IN	A	192.168.100.100

	;; Query time: 0 msec
	;; SERVER: 192.168.100.100#53(192.168.100.100)
	;; WHEN: Tue May 19 11:59:32 CEST 2015
	;; MSG SIZE  rcvd: 108
\end{lstlisting}
Reverse-Lookup für 192.168.100.200 sollte auf den noch aufzusetzenden SP mit sp.shib.lan zeigen:
\begin{lstlisting}
shib-idp@shibidp:~$ dig -x 192.168.100.200
	; <<>> DiG 9.9.5-3ubuntu0.2-Ubuntu <<>> -x 192.168.100.200
	;; global options: +cmd
	;; Got answer:
	;; ->>HEADER<<- opcode: QUERY, status: NOERROR, id: 8338
	;; flags: qr aa rd ra; QUERY: 1, ANSWER: 1, AUTHORITY: 1, ADDITIONAL: 2

	;; OPT PSEUDOSECTION:
	; EDNS: version: 0, flags:; udp: 4096
	;; QUESTION SECTION:
	;200.100.168.192.in-addr.arpa.	IN	PTR

	;; ANSWER SECTION:
	200.100.168.192.in-addr.arpa. 172800 IN	PTR	sp.shib.lan.

	;; AUTHORITY SECTION:
	100.168.192.in-addr.arpa. 172800	IN	NS	shib.lan.

	;; ADDITIONAL SECTION:
	shib.lan.		172800	IN	A	192.168.100.100

	;; Query time: 0 msec
	;; SERVER: 192.168.100.100#53(192.168.100.100)
	;; WHEN: Tue May 19 11:58:30 CEST 2015
	;; MSG SIZE  rcvd: 111
\end{lstlisting}
\step{Der DNS-Server kann jetzt auch auf dem VM-Host als Standard-Nameserver
für das Host-Only-Netzwerk eingetragen werden (das, in welchem sich Host, IdP
und SP bewegen sollen).}
\subsubsection{DHCP}
\step{DHCP3 Installieren:}
\begin{lstlisting}
shib-idp@shibidp:~$ sudo apt-get install isc-dhcp-server
\end{lstlisting}
\step{DHCP Config (/etc/dhcp/dhcpd.conf) - "`authoritative"' aktivieren,
Domainname setzen, Bereich "`subnet"' anlegen und host "`sp"' auf IP
192.168.100.200 fixieren (hier muss natürlich die MAC des eth1 Interfaces der SP-VM eingetragen werden):}
\begin{lstlisting}
shib-idp@shibidp:~$ cat /etc/dhcp/dhcpd.conf
	ddns-update-style none;
	option domain-name "shib.lan";
	option domain-name-servers 192.168.100.100;
	default-lease-time 600;
	max-lease-time 7200;
	authoritative;
	log-facility local7;
	subnet 192.168.100.0 netmask 255.255.255.0 {
		range 192.168.100.101 192.168.100.200;
		interface eth1;
	}

	host sp {
		hardware ethernet 08:00:27:9e:43:d3;
		fixed-address 192.168.100.200;
	}
\end{lstlisting}

\step{ DHCP starten:}
\begin{lstlisting}
shib-idp@shibidp:~$ sudo /etc/init.d/isc-dhcp-server start
\end{lstlisting}
\subsubsection{Certificate Authority aufsetzen und Zertifikate erstellen}
\step{CA erstellen (im Home-Verzeichnis möglich):}
\begin{lstlisting}
shib-idp@shibidp:~$ mkdir CA
shib-idp@shibidp:~$ cd CA
shib-idp@shibidp:~/CA$ /usr/lib/ssl/misc/CA.pl -newca
\end{lstlisting}
\begin{itemize}
 \item für Filename: <Enter> (neue erstellen)
 \item CA-Passphrase (auch für cakey.pem): +\$hib-idp.C@\_p@\$\$phr@5e**
 \item Country: DE
 \item State: Saxony
 \item Common Name: shib.lan
 \item Rest leer.
\end{itemize}

\step{eigene Zertifikate erstellen, die von der erstellten CA zertifiziert
sind:}
\anno{Anm.: es werden drei Zertifikate erstellt: eines für den PHP-LDAP-Admin,
eines für den Shibboleth IdP und ein "default"-Zertifikat (www.shib.lan)}
\anno{Anm.: vorher prüfen, ob das .rnd-File im Home-Verzeichnis dem Nutzer
gehört:}
\begin{lstlisting}
shib-idp@shibidp:~$ ll ~/ | grep .rnd
	-rw-------  1 root     root      1024 Mai 20 08:49 .rnd
\end{lstlisting}
Ist das nicht so (hier gehört das File root), dann sollte dies dem Nutzer (hier
shib-idp) übergeben werden:
\begin{lstlisting}
shib-idp@shibidp:~$ sudo chown shib-idp:shib-idp ~/.rnd
\end{lstlisting}
\step{neuen Request für LDAP-Zertifikat erstellen und signieren:}
\begin{lstlisting}
shib-idp@shibidp:~/CA$ /usr/lib/ssl/misc/CA.pl -newreq
\end{lstlisting}
\begin{itemize}
\item Hier genutzt:
\item Passphrase: +\$hib-idp.LD@P\_p@\$\$**
\item Country: DE
\item State: Saxony
\item Common Name: ldap.shib.lan
\item Rest leer.
\end{itemize}

\begin{lstlisting}
shib-idp@shibidp:~/CA$ /usr/lib/ssl/misc/CA.pl -sign
	Passphrase: CA-Passphrase (von oben)
	Sign Sertificae: y
	Commit: y
\end{lstlisting}
\step{im Verzeichnis CA sollte es jetzt so aussehen:}
\begin{lstlisting}
shib-idp@shibidp:~/CA$ ll
	drwxrwxr-x  3 shib-idp shib-idp 4096 Mai 20 10:28 ./
	drwxr-xr-x 20 shib-idp shib-idp 4096 Mai 20 10:09 ../
	drwxrwxr-x  6 shib-idp shib-idp 4096 Mai 20 10:29 demoCA/
	-rw-rw-r--  1 shib-idp shib-idp 4481 Mai 20 10:29 newcert.pem
	-rw-rw-r--  1 shib-idp shib-idp 1834 Mai 20 10:26 newkey.pem
	-rw-rw-r--  1 shib-idp shib-idp  985 Mai 20 10:26 newreq.pem
\end{lstlisting}
\step{für die Nutzung im Apache muss noch die Passphrase aus dem newkey.pem-File
entfernt werden:}
\begin{lstlisting}
shib-idp@shibidp:~/CA$ openssl rsa -in newkey.pem > ldap.shib.lan_key.pem
\end{lstlisting}
Passphrase: die vom LDAP-Key oben\newline
\step{Schlüssel und Key verschieben ("`Backup"'):}
\begin{lstlisting}
shib-idp@shibidp:~/CA$ mv ldap.shib.lan_key.pem demoCA/private/
shib-idp@shibidp:~/CA$ mv newcert.pem demoCA/certs/ldap.shib.lan_cert.pem
\end{lstlisting}
\step{Aufräumen:}
\begin{lstlisting}
shib-idp@shibidp:~/CA$ rm newreq.pem newkey.pem
\end{lstlisting}
\step{CA und LDAP-Cert bereitstellen:}
\begin{lstlisting}
shib-idp@shibidp:~/CA$ sudo cp demoCA/private/ldap.shib.lan_key.pem /etc/ssl/private/
shib-idp@shibidp:~/CA$ sudo cp demoCA/certs/ldap.shib.lan_cert.pem /etc/ssl/certs/
shib-idp@shibidp:~/CA$ sudo cp demoCA/cacert.pem /usr/share/ca-certificates/shib.lan/shib.lan.crt
\end{lstlisting}

\step{Das Zertifikat beinhalten noch "`Überhang"' der entfernt werden
muss:\newline Aus /etc/ssl/certs/ldap.shib.lan\_cert.pem muss alles entfernt werden, was
AUSSERHALB von BEGIN CERTIFICATE ... END CERTIFICATE steht, am Ende beinhaltet
die Datei nur noch etwas wie:}
\begin{lstlisting}
	-----BEGIN CERTIFICATE-----
	MIIDqzCCApOgAwIBAgIJAM857DKIFfSRMA0GCSqGSIb3DQEBCwUAMFQxCzAJBgNV
	.....
	DjW8g+KdwDa09JetID+kHiwto3NB3IN2YzUfr7nhSQ==
	-----END CERTIFICATE-----
\end{lstlisting}

\step{CA über ca-certificates importieren:}
\begin{lstlisting}
shib-idp@shibidp:~/CA$ sudo dpkg-reconfigure ca-certificates
\end{lstlisting}
neue Zertifizierungsstellen vertrauen: JA \newline
in der Liste shib,lan/shib.lan.crt raussuchen und anhaken (Leertaste), mit Enter weiter

\step{Zertifikate für www.shib.lan und idp.shib.lan erstellen:}
\begin{lstlisting}
shib-idp@shibidp:~/CA$ /usr/lib/ssl/misc/CA.pl -newreq
\end{lstlisting}
\begin{itemize}
\item Passphrase: +\$hib-idp.www\_p@\$\$**
\item Country: DE
\item State: Saxony
\item Common Name: ldap.shib.lan
\item Rest leer.
\end{itemize}
\begin{lstlisting}
			shib-idp@shibidp:~/CA$ /usr/lib/ssl/misc/CA.pl -sign
\end{lstlisting}
\begin{itemize}
  \item cakey-Passphrase com CA (oben)
  \item sign und commit: y
\end{itemize}
\begin{lstlisting}
shib-idp@shibidp:~/CA$ openssl rsa -in newkey.pem > www.shib.lan_key.pem
shib-idp@shibidp:~/CA$ mv www.shib.lan_key.pem demoCA/private/
shib-idp@shibidp:~/CA$ mv newcert.pem demoCA/certs/www.shib.lan_cert.pem
shib-idp@shibidp:~/CA$ rm new*
shib-idp@shibidp:~/CA$ /usr/lib/ssl/misc/CA.pl -newreq
\end{lstlisting}
\begin{itemize}
\item Passphrase: +\$hib-idp.idp\_p@\$\$**
\item Country: DE
\item State: Saxony
\item Common Name: ldap.shib.lan
\item Rest leer.
\end{itemize}
\begin{lstlisting}
shib-idp@shibidp:~/CA$ /usr/lib/ssl/misc/CA.pl -sign
\end{lstlisting}
\begin{itemize}
\item cakey-Passphrase com CA (oben)
\item sign und commit: y
\end{itemize}
\begin{lstlisting}
shib-idp@shibidp:~/CA$ openssl rsa -in newkey.pem > idp.shib.lan_key.pem
shib-idp@shibidp:~/CA$ mv idp.shib.lan_key.pem demoCA/private/
shib-idp@shibidp:~/CA$ mv newcert.pem demoCA/certs/idp.shib.lan_cert.pem
shib-idp@shibidp:~/CA$ rm new*
shib-idp@shibidp:~/CA$ sudo cp demoCA/certs/www.shib.lan_cert.pem /etc/ssl/certs/
shib-idp@shibidp:~/CA$ sudo cp demoCA/certs/idp.shib.lan_cert.pem /etc/ssl/certs/
shib-idp@shibidp:~/CA$ sudo cp demoCA/private/www.shib.lan_key.pem /etc/ssl/private/
shib-idp@shibidp:~/CA$ sudo cp demoCA/private/idp.shib.lan_key.pem
/etc/ssl/private/
\end{lstlisting}
\step{www.shib.lan\_cert.pem und
idp.shib.lan\_cert.pem (/etc/ssl/certs/) noch vom "`Überhang"' bereinigen (wie
oben)}

\step{ wird Chromium / Chrome zum Browsen verwendet, so sollte hier die CA als
 vertrauenswürdig hinzugefügt werden:}
\begin{itemize}
  \item Chromium:
  \begin{itemize}
    \item  Chromium in der VM: Einstellungen $\rightarrow$
  erweiterte Einstellungen $\rightarrow$ HTTPS/SSL (Zertifikate verwalten...) $\rightarrow$ Zertifizierungsstellen (Chromium) $\rightarrow$ Importieren... $\rightarrow$ cacert.pem aus /home/shib-idp/CA/demoCA/ importieren $\rightarrow$ Haken bei "`Diesem Zertifikat zur Identifizierung von Webseiten vertrauen"' $\rightarrow$ OK
  \item  Chromium auf VM-Host: analog, nur muss hier die cacert.pem noch per
  (Win)SCP oder Shared Folder für den Zugriff bereitgestellt
  werden)\footnote{\textbf{{\color{red} ACHTUNG:} DAMIT BEWERTET DAS
  WIRTS-SYSTEM VON DER DEMO-CA SIGNIERTE CERTS ALS ABSOLUT VERTRAUENSWÜRDIG! WENN MÖGLICH
  HIER DIE DEMO-CA NUR FÜR DIE VMs ALS VALIDE STAMMZERTIFIZIERUNGSSTELLE
  NUTZEN.
  AUF DEM VM-WIRT SOLLTE IN KAUF GENOMMEN WERDEN, IM BROWSER DIE WARNUNG
  WEGKLICKEN ZU MÜSSEN.}}
  \end{itemize}
\end{itemize}

\step{Mozilla: hier sollte es genügen, einmalig das Zertifikat herunterzuladen
(auf dem Hinweisbildschirm) und als Ausnahme zu genehmigen}
\step{ggf. ist noch ein Browser-Neustart nötig.}

\subsubsection{Apache Grundkonfig}
\step{Apache sollte per default schon für SSL vorbereitet sein, die ports.conf
sollte so aussehen:}
\begin{lstlisting}
shib-idp@shibidp:~$ cat /etc/apache2/ports.conf
	# If you just change the port or add more ports here, you will likely also
	# have to change the VirtualHost statement in
	# /etc/apache2/sites-enabled/000-default.conf

	Listen 80

	<IfModule ssl_module>
		Listen 443
	</IfModule>

	<IfModule mod_gnutls.c>
		Listen 443
	</IfModule>
\end{lstlisting}

\step{ggf. Apache SSL Modul aktivieren und Apache neustarten:}
\begin{lstlisting}
shib-idp@shibidp:~$ sudo a2enmod ssl
shib-idp@shibidp:~$ sudo service apache2 restart
\end{lstlisting}

\step{VHost für www.shib.lan:}
\begin{lstlisting}
shib-idp@shibidp:~$ cat /etc/apache2/sites-available/www.conf
	<VirtualHost*:443>
		ServerAdmin admin@shib.lan
		ServerName www.shib.lan
		ServerAlias www.shib.lan
		SSLEngine On
		SSLCertificateFile /etc/ssl/certs/www.shib.lan_cert.pem
		SSLCertificateKeyFile /etc/ssl/private/www.shib.lan_key.pem
		DocumentRoot "/var/www/html"
		<Directory "/var/www/html">
				Options FollowSymLinks
				AllowOverride AuthConfig
				Order allow,deny
				Allow from all
		</Directory>
		DirectoryIndex index.html
		ErrorLog ${APACHE_LOG_DIR}/error.log
		CustomLog ${APACHE_LOG_DIR}/access.log combined
	</VirtualHost>
\end{lstlisting}
\step{Vhost für Redirect von https://shib.lan und http://shib.lan auf
https://www.shib.lan}
\begin{lstlisting}
shib-idp@shibidp:~$ cat /etc/apache2/sites-available/redirect.conf
	<VirtualHost*:443>
	ServerName shib.lan
		ServerAlias shib.lan
		SSLEngine On
		SSLCertificateFile /etc/ssl/certs/www.shib.lan_cert.pem
		SSLCertificateKeyFile /etc/ssl/private/www.shib.lan_key.pem
		Redirect 301 / https://www.shib.lan/
		ErrorLog ${APACHE_LOG_DIR}/error.log
		CustomLog ${APACHE_LOG_DIR}/access.log combined
	</VirtualHost>

	<VirtualHost *:80>
		ServerAdmin admin@shib.lan
		ServerName shib.lan
		ServerAlias shib.lan
		Redirect 301 / https://www.shib.lan/
		ErrorLog ${APACHE_LOG_DIR}/error.log
		CustomLog ${APACHE_LOG_DIR}/access.log combined
	</VirtualHost>
\end{lstlisting}
\step{VHosts aktivieren und Apache neu starten}
\begin{lstlisting}
shib-idp@shibidp:~$ sudo a2ensite ldap.conf
shib-idp@shibidp:~$ sudo a2ensite www.conf
shib-idp@shibidp:~$ sudo a2ensite redirect.conf
shib-idp@shibidp:~$ sudo service apache2 restart
\end{lstlisting}

\subsubsection{OpenLDAP-Server}
\step{Pakete  slapd, ldap-utils installieren}
\begin{lstlisting}
shib-idp@shibidp:~$ sudo apt-get install slapd ldap-utils
\end{lstlisting}
\step{OpenLDAP konfigurieren:}
\begin{lstlisting}
shib-idp@shibidp:~$ sudo dpkg-reconfigure slapd
\end{lstlisting}
\begin{itemize}
	\item DNS: shib.lan
	\item Organisation: shib
	\item Password: idp
	\item Backend: hdb
	\item Löschen: nein
	\item Backup alte DB: ja
	\item LDAPv2 erlauben: nein
\end{itemize}

\step{TLS aktivieren:}
\begin{itemize}
  	\item Zertifikate in LDAP-Config laden: folgendes LDIF erzeugen und dieses
  ins Backend einfügen:
\end{itemize}
\begin{lstlisting}
shib-idp@shibidp:~$ cat /etc/ldap/tls.ldif
	\#\#\#\#\#\#\#\#\#\#\#\#\#\#\#\#\#\#\#\#\#\#\#\#\#\#\#\#\#\#\#\#\#\#\#\#\#\#\#\#\#\#\#\#\#\#\#\#\#\#\#\#\#\#\#\#\#\#\#
	\# CONFIGURATION for Support of TLS
	\#\#\#\#\#\#\#\#\#\#\#\#\#\#\#\#\#\#\#\#\#\#\#\#\#\#\#\#\#\#\#\#\#\#\#\#\#\#\#\#\#\#\#\#\#\#\#\#\#\#\#\#\#\#\#\#\#\#\#
	\# Add TLS supported access to user passwords for LDAP clients
	\# to the LDAP config.

	dn: cn=config
	changetype: modify
	add: olcTLSCACertificateFile
	olcTLSCACertificateFile: /etc/ssl/certs/ldap.shib.lan_cert.pem

	dn: cn=config
	changetype: modify
	add: olcTLSCertificateKeyFile
	olcTLSCertificateKeyFile: /etc/ssl/private/ldap.shib.lan_key.pem

	\#dn: cn=config
	\#changetype: modify
	\#delete: olcTLSCertificateFile

	dn: cn=config
	changetype: modify
	add: olcTLSCertificateFile
	olcTLSCertificateFile: /etc/ssl/certs/ldap.shib.lan_cert.pem

shib-idp@shibidp:~$ sudo ldapmodify -Y EXTERNAL -H ldapi:/// -f /etc/ldap/tls.ldif
\end{lstlisting}

\step{slapd (openldap) benötigt natürlich noch Rechte, um die
Zertifikatschlüssel auslesen zu dürfen:}
\begin{lstlisting}
shib-idp@shibidp:/etc/ldap$ sudo adduser openldap ssl-cert
\end{lstlisting}

\subsubsection*{(OPTIONAL phpldapadmin - man kann alle Arbeiten am LDAP auch
über das Apache Directory Studio wesentlich komfortabler erledigen)} * phpldapadmin installieren und Link erzeugen \footnote{Bug:
https://bugs.launchpad.net/ubuntu/+source/phpldapadmin/+bug/1321831}
\begin{lstlisting}
shib-idp@shibidp:~$ sudo ln -s /etc/apache2/conf-enabled /etc/apache2/conf.d
shib-idp@shibidp:~$ sudo apt-get install phpldapadmin
shib-idp@shibidp:~$ sudo rm /etc/apache2/conf.d
shib-idp@shibidp:~$ sudo service apache2 restart
\end{lstlisting}
* phpldapadmin für PHP >=5.5 patchen\footnote{
(http://sourceforge.net/u/nihilisticz/phpldapadmin/ci/7e53dab990748c546b79f0610c3a7a58431e9ebc/\#diff-1}:\newline
Diese Patches sind im aktuellen Paket schon angewandt, was fehlt ist
in /usr/share/phpldapadmin/lib/TemplateRender.php:
\begin{lstlisting}
Z.2469:
-- $default = $this->getServer()->getValue('appearance','password_hash');
++ $default = $this->getServer()->getValue('appearance','password_hash_custom');
\end{lstlisting}

* Apache VHost für phpldapadmin und SSL:
\begin{lstlisting}
shib-idp@shibidp:~$ cat /etc/apache2/sites-available/ldap.conf
	<VirtualHost *:443>
		ServerAdmin admin@shib.lan
		ServerName ldap.shib.lan
		ServerAlias ldap.shib.lan
		SSLEngine On
		SSLCertificateFile /etc/ssl/certs/ldap.shib.lan_cert.pem
		SSLCertificateKeyFile /etc/ssl/private/ldap.shib.lan_key.pem
		DocumentRoot "/usr/share/phpldapadmin/htdocs"
		<Directory "/usr/share/phpldapadmin">
				Options FollowSymLinks
				AllowOverride AuthConfig
				Order allow,deny
				Allow from all
		</Directory>
		DirectoryIndex index.php
		ErrorLog "/var/log/apache2/phpldapadmin_error.log"
		CustomLog "/var/log/apache2/phpldapadmin_access.log" combined
	</VirtualHost>

	<VirtualHost *:80>
			ServerAdmin admin@shib.lan
			ServerName ldap.shib.lan
			ServerAlias ldap.shib.lan
			Redirect 301 / https://ldap.shib.lan/
			ErrorLog ${APACHE_LOG_DIR}/error.log
			CustomLog ${APACHE_LOG_DIR}/access.log combined
	</VirtualHost>
\end{lstlisting}

* VHost aktivieren und Apache neu starten (oder auch reload, geht beides):
\begin{lstlisting}
shib-idp@shibidp:~$ sudo a2ensite ldap.conf
shib-idp@shibidp:~$ sudo service apache2 restart
\end{lstlisting}

* Schemata für eduPerson, schac, dfnEduPerson herunterladen:
\begin{lstlisting}
shib-idp@shibidp:~$ mkdir -p LDAP/schema
shib-idp@shibidp:~$ cd LDAP/schema
\end{lstlisting}
* folgende Schemata herunterladen:
\begin{itemize}
  \item
  https://wiki.refeds.org/download/attachments/1606048/schac-20150413-1.5.0.schema.txt
  \item https://spaces.internet2.edu/display/macedir/OpenLDAP+eduPerson
  \item
  https://www.aai.dfn.de/fileadmin/documents/attributes/200811/dfneduperson-1.0.schema.txt
\end{itemize}

* bis auf eduperson liegen die Schemata im "`schema"' Format vor, es wird jedoch LDIF benötigt. Daher ist hier eine Umwandlung nötig:
\newline * Anlegen einer schema\_convert.conf mit folgendem Inhalt:
\begin{lstlisting}
shib-idp@shibidp:~/LDAP/schema$ cat schema_convert.conf
	include /etc/ldap/schema/core.schema
	include /etc/ldap/schema/collective.schema
	include /etc/ldap/schema/corba.schema
	include /etc/ldap/schema/cosine.schema
	include /etc/ldap/schema/duaconf.schema
	include /etc/ldap/schema/dyngroup.schema
	include /etc/ldap/schema/inetorgperson.schema
	include /etc/ldap/schema/java.schema
	include /etc/ldap/schema/misc.schema
	include /etc/ldap/schema/nis.schema
	include /etc/ldap/schema/openldap.schema
	include /etc/ldap/schema/ppolicy.schema
	include /etc/ldap/schema/ldapns.schema
	include /etc/ldap/schema/schac-1.5.0.schema
	include /etc/ldap/schema/dfneduperson-1.0.schema
\end{lstlisting}

* dfneduperson und schac schema kopieren:
\begin{lstlisting}
	shib-idp@shibidp:~/LDAP/schema$ sudo cp dfneduperson-1.0.schema /etc/ldap/schema/
	shib-idp@shibidp:~/LDAP/schema$ sudo cp schac-1.5.0.schema /etc/ldap/schema/
\end{lstlisting}
* eduperson war schon LDIF, dies muss natürlich auch in's Schema:
\begin{lstlisting}
	shib-idp@shibidp:~/LDAP/schema$ sudo cp eduperson.ldif /etc/ldap/schema/
\end{lstlisting}
* konvertierung mittels slapcat:
\begin{lstlisting}
	shib-idp@shibidp:~/LDAP/schema$ mkdir ldif
	shib-idp@shibidp:~/LDAP/schema$ slapcat -f schema_convert.conf -F ./ldif -n0
\end{lstlisting}
* in den cn=schema - Ordner wechseln
\begin{lstlisting}
	shib-idp@shibidp:~/LDAP/schema$ cd ldif/cn\=config/cn\=schema/
\end{lstlisting}
* neu erstellte LDIFs nach /etc/ldap/schema kopieren (die Zahlen in den {} können variieren!):
\begin{lstlisting}
	shib-idp@shibidp:~/LDAP/schema/ldif/cn=config/cn=schema$ sudo mv cn\=\{13\}schac-1.ldif /etc/ldap/schema/schac-1.5.0.ldif
	shib-idp@shibidp:~/LDAP/schema/ldif/cn=config/cn=schema$ sudo mv cn\=\{14\}dfneduperson-1.ldif /etc/ldap/schema/dfneduperson-1.0.ldif
\end{lstlisting}
* die LDIFs müssen noch nachbearbeitet werden:
\begin{lstlisting}
	# Zeile 3 in dfneduperson-1.0.ldif muss geändert werden:
	Z3 -- dn: cn={14}dfneduperson-1
	Z3 ++ dn: cn=dfneduperson,cn=schema,cn=config
	\# Zeile 5 in dfneduperson-1.0.ldif:
	Z5 -- cn: {14}dfneduperson-1
	Z5 ++ cn: dfneduperson

	# die letzten 7 Zeilen müssen entfernt werden (Werte können variieren):
	Zn-6 -- structuralObjectClass: olcSchemaConfig
	Zn-5 -- entryUUID: 571ca564-932e-1034-8286-bdb5afc52f0d
	Zn-4 -- creatorsName: cn=config
	Zn-3 -- createTimestamp: 20150520112310Z
	Zn-2 -- entryCSN: 20150520112310.319976Z#000000#000#000000
	Zn-1 -- modifiersName: cn=config
	Zn   -- modifyTimestamp: 20150520112310Z
	# analog gilt für schac-1.5.0.ldif für Zeilen3, 5 und die letzten 7:
	Z3 -- dn: cn={13}schac-1
	Z3 ++ dn: cn=schac,cn=schema,cn=config
	Z5 -- cn: {13}schac-1
	Z5 ++ cn: schac
	Zn-6 -- structuralObjectClass: olcSchemaConfig
	Zn-5 -- entryUUID: 571c9ff6-932e-1034-8285-bdb5afc52f0d
	Zn-4 -- creatorsName: cn=config
	Zn-3 -- createTimestamp: 20150520112310Z
	Zn-2 -- entryCSN: 20150520112310.319976Z#000000#000#000000
	Zn-1 -- modifiersName: cn=config
	Zn   -- modifyTimestamp: 20150520112310Z
\end{lstlisting}
* LDIFs zum Schema von LDAP hinzufügen:
\begin{lstlisting}
	shib-idp@shibidp:~/LDAP/schema$ sudo ldapadd -Y EXTERNAL -H ldapi:/// -f /etc/ldap/schema/eduperson.ldif
	shib-idp@shibidp:~/LDAP/schema$ sudo ldapadd -Y EXTERNAL -H ldapi:/// -f /etc/ldap/schema/dfneduperson-1.0.ldif
	shib-idp@shibidp:~/LDAP/schema$ sudo ldapadd -Y EXTERNAL -H ldapi:/// -f /etc/ldap/schema/schac-1.5.0.ldif
\end{lstlisting}
* phpldapadmin konfigurieren (in /etc/phpldapadmin):
\begin{lstlisting}
	# Backup Orginal-Konfig:
		shib-idp@shibidp:/etc/phpldapadmin$ sudo cp config.php config.php.bak
	# Server-DN:
		Z300 -- $servers->setValue('server','base',array('dc=example,dc=com'));
		Z300 ++ $servers->setValue('server','base',array('dc=shib,dc=lan'));
		Z326 -- $servers->setValue('login','bind_id','cn=admin,dc=example,dc=com');
		Z326 ++ $servers->setValue('login','bind_id','cn=admin,dc=shib,dc=lan');
\end{lstlisting}
* das Web-Interface ist jetzt über https://ldap.shib.lan/ erreichbar, Login ist wie oben angegeben cn=admin,dc=shib,dc=lan mit Passwort idp

\subsubsection*{Apache Directory Studio - verwendung idealerweise auf dem
Hostrechner:}
\begin{itemize}
  \item aktuelle Version hier raussuchen
  https://directory.apache.org/studio/download/
  \item ist der DNS auf dem Host eingetragen, so kann einfach per ldap.shib.lan auf den ldap server zugegriffen werden
  \begin{itemize}
		\item user: cn=admin,dc=shib,dc=lan
		\item pass: idp
\end{itemize}
\end{itemize}
\subsubsection*{Baum und Beispielnutzer:}
Wurzel: dc=shib,dc=lan\newline
OUs: students, staff, extern\newline
\begin{center}
\footnotesize
	\begin{tabular}{| l | l | l | p{10cm} | }
	\hline
	uid & Passwort & ou & Attribute \\ \hline\hline
	alice & alice & students &
sn: Alison\newline
givenName: Alice\newline
cn: Alice Alison\newline
dfnEduPersonFieldOfStudyString: BWL\newline
dfnEduPersonStudyBranch3: 021\newline
dfnEduPersonStudyBranch2: 30\newline
dfnEduPersonStudyBranch1: 03\newline
eduPersonAffiliation: student\newline
eduPersonAffiliation: member\newline
eduPersonPrimaryAffiliation: student\newline
eduPersonEntitlement: urn:mace:dir:common-lib-terms
	 \\ \hline
	 bob & bob & students &
sn: Bobinson\newline
givenName: Bob\newline
cn: Bob Bobinson\newline
dfnEduPersonFieldOfStudyString: Maschinenbau\newline
dfnEduPersonStudyBranch1: 08\newline
dfnEduPersonStudyBranch2: 63\newline
dfnEduPersonStudyBranch3: 104\newline
eduPersonAffiliation: student\newline
eduPersonAffiliation: member\newline
eduPersonPrimaryAffiliation: student\newline
eduPersonEntitlement: urn:mace:dir:common-lib-terms\\ \hline
	ed & ed & students &
sn: Edison\newline
givenName: Ed\newline
cn: Ed Edison\newline
dfnEduPersonFieldOfStudyString: Verfahrenstechnik\newline
dfnEduPersonStudyBranch1: 08\newline
dfnEduPersonStudyBranch2: 63\newline
dfnEduPersonStudyBranch3: 226\newline
eduPersonAffiliation: student\newline
eduPersonAffiliation: member\newline
eduPersonPrimaryAffiliation: student\newline
eduPersonEntitlement: urn:mace:dir:common-lib-terms \\ \hline
	carl & carl & staff &
sn: Carlson\newline
givenName: Carl\newline
cn: Carl Carlson\newline
eduPersonAffiliation: member\newline
eduPersonAffiliation: staff\newline
eduPersonPrimaryAffiliation: staff\newline
dfnEduPersonCostCenter: 01-INFO\newline
eduPersonEntitlement: urn:mace:dir:common-lib-terms\\ \hline
	gunhild & gunhild & staff &
sn: Gunhildson\newline
givenName: Gundhild\newline
cn: Gunhild Gunhildson\newline
eduPersonAffiliation: member\newline
eduPersonAffiliation: staff\newline
eduPersonPrimaryAffiliation: staff\newline
dfnEduPersonCostCenter: 02-CHEM\newline
eduPersonEntitlement: urn:mace:dir:common-lib-terms\\ \hline
	nils & nils & extern &
sn: Nilson	\newline
givenName: Nilson\newline
cn: Nils Nilson\newline
eduPersonAffiliation: affiliate\newline
eduPersonPrimaryAffiliation: affiliate\\ \hline
	\end{tabular}
\end{center}
\subsubsection{Identity Provider}
\step{Apache Module installieren:}
\begin{lstlisting}
shib-idp@shibidp:~$ sudo a2enmod headers
shib-idp@shibidp:~$ sudo a2enmod proxy_ajp
shib-idp@shibidp:~$ sudo service apache2 restart
\end{lstlisting}
\step{Tomcat 7 installieren:}
\begin{lstlisting}
shib-idp@shibidp:~$ sudo apt-get install openjdk-7-jre
shib-idp@shibidp:~$ sudo apt-get install tomcat7
\end{lstlisting}
\step{Tomcat Konfiguration anpassen:}
* editieren der /etc/default/tomcat7:
\begin{lstlisting}
Z21 -- JAVA_OPTS="-Djava.awt.headless=true -Xmx128m -XX:+UseConcMarkSweepGC"
Z21 ++ JAVA_OPTS="-Djava.awt.headless=true -Xmx512m -XX:+UseConcMarkSweepGC"
Z32 -- #TOMCAT7_SECURITY=no
Z32 ++ TOMCAT7_SECURITY=no
\end{lstlisting}
* Rechte für SSL-Zertifikate:
\begin{lstlisting}
shib-idp@shibidp:~$ sudo adduser tomcat7 ssl-cert
\end{lstlisting}
\step{Identity Provider 2.4.4 installieren:}
\begin{lstlisting}
shib-idp@shibidp:~$ mkdir IdP
shib-idp@shibidp:~$ cd IdP/
shib-idp@shibidp:~/IdP$ wget http://shibboleth.net/downloads/identity-provider/2.4.4/shibboleth-identityprovider-2.4.4-bin.tar.gz
shib-idp@shibidp:~/IdP$ tar -xzf shibboleth-identityprovider-2.4.4-bin.tar.gz
shib-idp@shibidp:~/IdP$ cd shibboleth-identityprovider-2.4.4/
shib-idp@shibidp:~/IdP/shibboleth-identityprovider-2.4.4$ sudo JAVA_HOME=/usr/lib/jvm/java-7-openjdk-amd64/ ./install.sh
\end{lstlisting}
\begin{itemize}
	\item Installationverzeichnis: /opt/shibboleth-idp
	\item FQDN: idp.shib.lan
	\item Keystore Pass: *\$hib-idp.k3yst0r3\_p@\$\$**
\end{itemize}
\begin{lstlisting}
shib-idp@shibidp:~$ sudo chown tomcat7 /opt/shibboleth-idp/{metadata,logs}
\end{lstlisting}
\step{Apache und Tomcat Konfigurieren \footnote{
siehe
auch https://www.aai.dfn.de/dokumentation/identity-provider/konfiguration/}:}
* Apache-Config (VHost für IDP anlegen (/etc/apache2/sites-available)):
\begin{lstlisting}
shib-idp@shibidp:~$ cat /etc/apache2/sites-available/idp.conf
	<VirtualHost *:443>
	  ServerName              idp.shib.lan
	  SSLEngine               on
	  SSLCertificateFile      /etc/ssl/certs/idp.shib.lan_cert.pem
	  SSLCertificateKeyFile   /etc/ssl/private/idp.shib.lan_key.pem
	  SSLProtocol All -SSLv2 -SSLv3
	  SSLHonorCipherOrder On
	  SSLCipherSuite 'ECDH+AESGCM:DH+AESGCM:ECDH+AES256:DH+AES256:ECDH+AES128:DH+AES:RSA+AESGCM:RSA+AES:ECDH+3DES:DH+3DES:RSA+3DES:!aNULL:!eNULL:!LOW:!RC4:!MD5:!EXP:!PSK:!DSS:!SEED:!ECDSA:!CAMELLIA'

	  <Location /idp>
		Allow from all
		ProxyPass ajp://localhost:8009/idp
		#verhindern, dass die Login Seite in iFrames eingebunden wird
		Header always append X-FRAME-OPTIONS "DENY"
	  </Location>
	</VirtualHost>

	Listen 8443

	<VirtualHost *:8443>
	  ServerName              idp.hib.lan
	  SSLEngine               on
	  SSLCertificateFile      /opt/shibboleth-idp/credentials/idp.crt
	  SSLCertificateKeyFile   /opt/shibboleth-idp/credentials/idp.key
	  SSLProtocol All -SSLv2 -SSLv3
	  SSLHonorCipherOrder On
	  SSLCipherSuite 'ECDH+AESGCM:DH+AESGCM:ECDH+AES256:DH+AES256:ECDH+AES128:DH+AES:RSA+AESGCM:RSA+AES:ECDH+3DES:DH+3DES:RSA+3DES:!aNULL:!eNULL:!LOW:!RC4:!MD5:!EXP:!PSK:!DSS:!SEED:!ECDSA:!CAMELLIA'
	  SSLVerifyClient optional_no_ca
	  SSLVerifyDepth  10
	  SSLOptions      +StdEnvVars +ExportCertData

	  <Location /idp>
		Allow from all
		ProxyPass ajp://localhost:8009/idp
	  </Location>
	</VirtualHost>
\end{lstlisting}
\step{VHosts aktivieren:}
\begin{lstlisting}
shib-idp@shibidp:~$ sudo a2ensite idp.conf
shib-idp@shibidp:~$ sudo service apache2 reload
\end{lstlisting}
\step{Tomcat Konfiguration:}
* server.xml anpasssen (/etc/tomcat7/) und in den Abschnitt <Service name="Catalina"> (Z.56) einen neuen Connector einfügen:
\begin{lstlisting}
<!-- Define an AJP 1.3 Connector on port 8009 -->
<Connector port="8009" address="127.0.0.1"
  enableLookups="false"
  redirectPort="8443"
  protocol="AJP/1.3"
  maxPostSize="100000" />
\end{lstlisting}

* idp.xml erzeugen (in /etc/tomcat7/Catalina/localhost/)
\begin{lstlisting}
shib-idp@shibidp:~$ cat /etc/tomcat7/Catalina/localhost/idp.xml
<Context docBase="/opt/shibboleth-idp/war/idp.war"
		 privileged="true"
		 antiResourceLocking="false"
		 antiJARLocking="false"
		 unpackWAR="false"
		 swallowOutput="true" />
\end{lstlisting}
* Tomcat neu starten:
\begin{lstlisting}
shib-idp@shibidp:~$ sudo service tomcat7 restart
\end{lstlisting}
* kurz warten, bis der IdP komplett geladen ist.
* Aufrufen: https://idp.shib.lan/idp/profile/Status, hier sollte einfach
"`ok"' stehen - das IdP Servlet läuft!
\step{Logs bei Fehlern:}
\begin{itemize}
  \item IdP: /opt/shibboleth-idp/logs/idp-process.log
  \item Tomcat: /var/log/tomcat7/catalina.XXX.log (XXX = Datum)
\end{itemize}
\step{IdP konfigurieren:}
* service.xml anpassen (/opt/shibboleth-idp/conf) / polling Intervall für Auslesen des Attribute-Resolvers und -Filters auf 5 Minuten setzen für den Test:
\begin{lstlisting}
Z14: -- <srv:Service id="shibboleth.AttributeResolver" xsi:type="attribute-resolver:ShibbolethAttributeResolver">
Z14: ++ <srv:Service id="shibboleth.AttributeResolver" xsi:type="attribute-resolver:ShibbolethAttributeResolver" configurationResourcePollingFrequency="PT5M">

Z18: -- <srv:Service id="shibboleth.AttributeFilterEngine" xsi:type="attribute-afp:ShibbolethAttributeFilteringEngine">
Z18: ++ <srv:Service id="shibboleth.AttributeFilterEngine" xsi:type="attribute-afp:ShibbolethAttributeFilteringEngine" configurationResourcePollingFrequency="PT5M">
\end{lstlisting}

* relying-party.xml anpassen (/opt/shibboleth-idp/conf) / wir sind NICHT im
DFN-AAI mit dem Testsystem, daher unter <metadata:MetadataProvider
id="ShibbolethMetadata" xsi:type="metadata:ChainingMetadataProvider"> (Z. 76) zusätzlich einfügen:
\begin{lstlisting}
  <metadata:MetadataProvider id="sp.shib.lan" xsi:type="metadata:FileBackedHTTPMetadataProvider"
      metadataURL="https://sp.shib.lan/Shibboleth.sso/Metadata"
      backingFile="/opt/shibboleth-idp/metadata/sp-metadata.xml" />
\end{lstlisting}
* handler.xml (/opt/shibboleth-idp/conf) / Login-Methode unter <ph:LoginHandler>
konfigurieren:
\begin{lstlisting}
		Z115: --    <ph:LoginHandler xsi:type="ph:RemoteUser">
		Z116: --		<ph:AuthenticationMethod>urn:oasis:names:tc:SAML:2.0:ac:classes:unspecified</ph:AuthenticationMethod>
		Z117: --	</ph:LoginHandler>
		Z115: ++    <!-- <ph:LoginHandler xsi:type="ph:RemoteUser">
		Z116: ++		<ph:AuthenticationMethod>urn:oasis:names:tc:SAML:2.0:ac:classes:unspecified</ph:AuthenticationMethod>
		Z117: ++	</ph:LoginHandler> -->
		* "Einkommentieren" von UsernamePassword:
		Z129: -- <!--
		Z134: -- -->
\end{lstlisting}
* login.config (/opt/shibboleth-idp/conf) - LDAP Auth definieren:
\begin{lstlisting}
	Z29: -- /*
	Z30: --	edu.vt.middleware.ldap.jaas.LdapLoginModule required
	Z31: --	  ldapUrl="ldap://ldap.example.org"
	Z32: --	  baseDn="ou=people,dc=example,dc=org"
	Z33: --	  ssl="true"
	Z34: --	  userFilter="uid={0}";
	Z35: --	*/

	Z29: ++
	Z30: ++ edu.vt.middleware.ldap.jaas.LdapLoginModule required
	Z31: ++ 	  ldapUrl="ldap://ldap.shib.lan"
	Z32: ++ 	  baseDn="dc=shib,dc=lan"
	Z33: ++	      subtreeSearch="true"
	Z34: ++ 	  tls="true"
	Z35: ++ 	  bindDn="cn=admin,dc=shib,dc=lan"
	Z36: ++ 	  bindCredential="idp"
	Z37: ++ 	  userFilter="uid={0}";
\end{lstlisting}
* logging.xml (/opt/shibboleth-idp/conf) anpassen: (Debugging für LDAP setzen)
\begin{lstlisting}
	Z15: -- <logger name="edu.vt.middleware.ldap" level="WARN"/>
	Z15: ++ <logger name="edu.vt.middleware.ldap" level="DEBUG"/>
\end{lstlisting}
\step{Um jetzt den IdP testen zu können, muss hier der Tomcat neu gestartet
werden. Außerdem wird jetzt natürlich der Service Provider gebraucht.}
\begin{lstlisting}
shib-idp@shibidp:~$ sudo service tomcat7 restart
\end{lstlisting}

\subsection{Service Provider}
\subsubsection{Netzwerk}
\step{Interface auf DHCP für NAT und Host-Only (in der VM-Config müssen sich IdP
und SP im selben Host-Only-Netzwerk befinden)}
\begin{lstlisting}
shib-sp@shib-sp:~$ cat /etc/network/interfaces
	# interfaces(5) file used by ifup(8) and ifdown(8)
	auto lo
	iface lo inet loopback

	auto eth0
	iface eth0 inet dhcp

	auto eth1
	iface eth1 inet dhcp
\end{lstlisting}

\step{DNS auf Reihenfolge 192.168.100.100 forcen\footnote{Ansonsten
wird versucht, über eth0 aufzulösen, was natürlich nicht klappt. Hier muss noch eine Lösung
gefunden werden!}:}
* anpassen der dhclient.conf (/etc/dhcp) und neues DHCP-Lease holen:
\begin{lstlisting}
Z20: ++ supersede domain-name-servers 192.168.100.100;

shib-sp@shib-sp:~$ sudo ifdown eth0
shib-sp@shib-sp:~$ sudo ifdown eth1
shib-sp@shib-sp:~$ sudo ifup eth0
shib-sp@shib-sp:~$ sudo ifup eth1
\end{lstlisting}
\step{wenn der DHCP und DNS auf dem IdP korrekt konfiguriert ist, sollte
folgendes funktionieren:}
* DNS-Abfrage über NAT:
\begin{lstlisting}
shib-sp@shib-sp:~$ dig google.de
	; <<>> DiG 9.9.5-3ubuntu0.2-Ubuntu <<>> google.de
	;; global options: +cmd
	;; Got answer:
	;; ->>HEADER<<- opcode: QUERY, status: NOERROR, id: 21335
	;; flags: qr rd ra; QUERY: 1, ANSWER: 4, AUTHORITY: 13, ADDITIONAL: 1

	;; OPT PSEUDOSECTION:
	; EDNS: version: 0, flags:; udp: 4096
	;; QUESTION SECTION:
	;google.de.			IN	A

	;; ANSWER SECTION:
	google.de.		117	IN	A	173.194.32.223
	google.de.		117	IN	A	173.194.32.215
	google.de.		117	IN	A	173.194.32.207
	google.de.		117	IN	A	173.194.32.216

	;; AUTHORITY SECTION:
	.			18625	IN	NS	c.root-servers.net.
	.			18625	IN	NS	d.root-servers.net.
	.			18625	IN	NS	i.root-servers.net.
	.			18625	IN	NS	b.root-servers.net.
	.			18625	IN	NS	h.root-servers.net.
	.			18625	IN	NS	e.root-servers.net.
	.			18625	IN	NS	m.root-servers.net.
	.			18625	IN	NS	g.root-servers.net.
	.			18625	IN	NS	j.root-servers.net.
	.			18625	IN	NS	l.root-servers.net.
	.			18625	IN	NS	a.root-servers.net.
	.			18625	IN	NS	f.root-servers.net.
	.			18625	IN	NS	k.root-servers.net.

	;; Query time: 9 msec
	;; SERVER: 192.168.100.100#53(192.168.100.100)
	;; WHEN: Tue May 19 12:59:48 CEST 2015
	;; MSG SIZE  rcvd: 313
\end{lstlisting}
* DNS-Abfrage auf shib.lan:
\begin{lstlisting}
shib-sp@shib-sp:~$ dig shib.lan
	; <<>> DiG 9.9.5-3ubuntu0.2-Ubuntu <<>> shib.lan
	;; global options: +cmd
	;; Got answer:
	;; ->>HEADER<<- opcode: QUERY, status: NOERROR, id: 16736
	;; flags: qr aa rd ra; QUERY: 1, ANSWER: 1, AUTHORITY: 1, ADDITIONAL: 1

	;; OPT PSEUDOSECTION:
	; EDNS: version: 0, flags:; udp: 4096
	;; QUESTION SECTION:
	;shib.lan.			IN	A

	;; ANSWER SECTION:
	shib.lan.		172800	IN	A	192.168.100.100

	;; AUTHORITY SECTION:
	shib.lan.		172800	IN	NS	shib.lan.

	;; Query time: 0 msec
	;; SERVER: 192.168.100.100#53(192.168.100.100)
	;; WHEN: Tue May 19 12:59:52 CEST 2015
	;; MSG SIZE  rcvd: 67
\end{lstlisting}
* DNS auf sp.shib.lan (SP-VM selbst):
\begin{lstlisting}
shib-sp@shib-sp:~$ dig sp.shib.lan
	; <<>> DiG 9.9.5-3ubuntu0.2-Ubuntu <<>> sp.shib.lan
	;; global options: +cmd
	;; Got answer:
	;; ->>HEADER<<- opcode: QUERY, status: NOERROR, id: 23654
	;; flags: qr aa rd ra; QUERY: 1, ANSWER: 1, AUTHORITY: 1, ADDITIONAL: 2

	;; OPT PSEUDOSECTION:
	; EDNS: version: 0, flags:; udp: 4096
	;; QUESTION SECTION:
	;sp.shib.lan.			IN	A

	;; ANSWER SECTION:
	sp.shib.lan.		172800	IN	A	192.168.100.200

	;; AUTHORITY SECTION:
	shib.lan.		172800	IN	NS	shib.lan.

	;; ADDITIONAL SECTION:
	shib.lan.		172800	IN	A	192.168.100.100

	;; Query time: 0 msec
	;; SERVER: 192.168.100.100#53(192.168.100.100)
	;; WHEN: Tue May 19 13:00:01 CEST 2015
	;; MSG SIZE  rcvd: 86
\end{lstlisting}

\subsubsection{Apache und Shibboleth Daemon (shibd)}
\step{Apache + Shibboleth-SP Daemon (shibd) installieren}
\begin{lstlisting}
shib-sp@shib-sp:~$ sudo apt-get install apache2
shib-sp@shib-sp:~$ sudo a2enmod ssl
shib-sp@shib-sp:~$ sudo apt-get install libapache2-mod-shib2
shib-sp@shib-sp:~$ sudo adduser _shibd ssl-cert
shib-sp@shib-sp:~$ sudo service apache2 restart
shib-sp@shib-sp:~$ sudo service shibd restart
\end{lstlisting}
\step{wir brauchen noch ein gültiges Zertifikat für den SP. Zufälligerweise
haben wir ja im IdP eine eigene CA...}
* im Homeverzeichnis im SP:
\begin{lstlisting}
shib-sp@shib-sp:~$ mkdir certs
\end{lstlisting}

\step{$\rightarrow$ Wechseln zur IdP-VM}
* Wechseln nach CA im Home-Verzeichnis (cd ~/CA)
\begin{lstlisting}
shib-idp@shibidp:~/CA$ /usr/lib/ssl/misc/CA.pl -newreq
\end{lstlisting}
\begin{itemize}
\item Passphrase: +\$hib-idp.sp\_p@\$\$**
\item Country: DE
\item State: Saxony
\item Common Name: ldap.shib.lan
\item Rest leer.
\end{itemize}
\begin{lstlisting}
shib-idp@shibidp:~/CA$ /usr/lib/ssl/misc/CA.pl -sign
\end{lstlisting}
\begin{itemize}
\item cakey-Passphrase vom CA: +\$hib-idp.C@\_p@\$\$phr@5e**
\item sign und commit: y
\end{itemize}
\begin{lstlisting}
shib-idp@shibidp:~/CA$ openssl rsa -in newkey.pem > sp.shib.lan_key.pem
shib-idp@shibidp:~/CA$ mv sp.shib.lan_key.pem demoCA/private/
shib-idp@shibidp:~/CA$ mv newcert.pem demoCA/certs/sp.shib.lan_cert.pem
shib-idp@shibidp:~/CA$ rm new*
shib-idp@shibidp:~/CA$ scp demoCA/private/sp.shib.lan_key.pem shib-sp@sp.shib.lan:/home/shib-sp/certs
	# connect: yes
	# passwort: sp
shib-idp@shibidp:~/CA$ scp demoCA/certs/sp.shib.lan_cert.pem  shib-sp@sp.shib.lan:/home/shib-sp/certs
shib-idp@shibidp:~/CA$ scp demoCA/cacert.pem  shib-sp@sp.shib.lan:/home/shib-sp/certs/shib.lan.crt
\end{lstlisting}

\step{$\rightarrow$ Wechseln zur SP-VM}
\step{Zertifikate installieren}
* aus dem Zertifikat (sp.shib.lan\_cert.pem) wieder alles AUSSERHALB von BEGIN
RSA PRIVATE KEY und END RSA PRIVATE KEY entfernen, danach:
\begin{lstlisting}
shib-sp@shib-sp:~$ sudo cp certs/sp.shib.lan\_key.pem /etc/ssl/private/
shib-sp@shib-sp:~$ sudo cp certs/sp.shib.lan\_cert.pem /etc/ssl/certs/
shib-sp@shib-sp:~$ sudo mkdir /usr/share/ca-certificates/shib.lan
shib-sp@shib-sp:~$ sudo cp certs/shib.lan.crt /usr/share/ca-certificates/shib.lan/
shib-sp@shib-sp:~$ sudo dpkg-reconfigure ca-certificates
	# yes
	# shib.lan suchen und anhaken -> OK
\end{lstlisting}

* VHost für sp.shib.lan erzeugen:
\begin{lstlisting}
shib-sp@shib-sp:~$ cat /etc/apache2/sites-available/sp.conf
	<VirtualHost *:443>
	  ServerName              sp.shib.lan
	  ServerAdmin             webmaster@example.org
	  DocumentRoot            /var/www/html
	  SSLEngine on
	  SSLCertificateFile      /etc/ssl/certs/sp.shib.lan_cert.pem
	  SSLCertificateKeyFile   /etc/ssl/private/sp.shib.lan_key.pem

	  SSLCertificateChainFile /etc/ssl/certs/shib.lan.pem

	  SSLProtocol All -SSLv2 -SSLv3
	  SSLHonorCipherOrder On
	  SSLCipherSuite 'ECDH+AESGCM:DH+AESGCM:ECDH+AES256:DH+AES256:ECDH+AES128:DH+AES:RSA+AESGCM:RSA+AES:ECDH+3DES:DH+3DES:RSA+3DES:!aNULL:!eNULL:!LOW:!RC4:!MD5:!EXP:!PSK:!DSS:!SEED:!ECDSA:!CAMELLIA'

	  ErrorLog ${APACHE_LOG_DIR}/error.log
	  CustomLog ${APACHE_LOG_DIR}/access.log combined

	  <Location /shibtest>
		AuthType shibboleth
		ShibRequireSession On
		require valid-user
		DirectoryIndex shib-info.php
	  </Location>

	  # optional (Metadata-Access at entityID-URL)
	  Redirect seeother /shibboleth https://sp.shib.lan/Shibboleth.sso/Metadata

	</VirtualHost>

	<VirtualHost *:80>
	  ServerAdmin admin@shib.lan
	  ServerName sp.shib.lan
	  ServerAlias sp.shib.lan
	  Redirect 301 / https://sp.shib.lan/
	  ErrorLog ${APACHE_LOG_DIR}/error.log
	  CustomLog ${APACHE_LOG_DIR}/access.log combined

	  <Location /shibtest>
		Redirect 301 / https://sp.shib.lan/shibtest
	  </Location>

	</VirtualHost>
\end{lstlisting}
* Verzeichnis shibtest im Apache-www-Root erstellen und einfache
Beispiel-index.html erzeugen:
\begin{lstlisting}
shib-sp@shib-sp:~$ sudo mkdir /var/www/html/shibtest
shib-sp@shib-sp:~$ cat /var/www/html/shibtest/index.html
	<html>
	<head>
	<title>SHIBTEST</title>
	</head>
	<body>
	<h1>SHIBTEST</h1>
	</body>
	</html>

shib-sp@shib-sp:~$ sudo a2ensite sp.conf
shib-sp@shib-sp:~$ sudo service apache2 restart
\end{lstlisting}

* SP konfigurieren (shibboleth2.xml  in /etc/shibboleth/)
\begin{lstlisting}
Z23: -- <ApplicationDefaults entityID="https://sp.example.org/shibboleth"
Z24: --      REMOTE_USER="eppn persistent-id targeted-id"
Z23: ++ <ApplicationDefaults entityID="https://sp.shib.lan/shibboleth"
Z24: ++      REMOTE_USER="eppn persistent-id targeted-id uniqueID"
Z25: ++      signing="back" requireTransportAuth="false">

Z35: -- <Sessions lifetime="28800" timeout="3600" relayState="ss:mem"
Z56: --           checkAddress="false" handlerSSL="false" cookieProps="http">
Z35: ++ <Sessions lifetime="28800" timeout="3600" relayState="ss:mem"
Z36: ++           checkAddress="false" consistentAddress="true" handlerSSL="true" cookieProps="https">

Z44: -- <SSO entityID="https://idp.example.org/idp/shibboleth"
Z45: --   discoveryProtocol="SAMLDS" discoveryURL="https://ds.example.org/DS/WAYF">
Z46: --  SAML2 SAML1
Z47: -- </SSO>

Z44: ++ <SSO entityID="https://idp.shib.lan/idp/shibboleth">
Z45: ++ 	SAML2
Z46: ++ </SSO>

Z59: -- <Handler type="Session" Location="/Session" showAttributeValues="false"/>
Z59: ++ <Handler type="Session" Location="/Session" showAttributeValues="true"/>

Z77: ++ <MetadataProvider type="XML" url="https://idp.xacml.lan/idp/shibboleth" />

Z98: -- <CredentialResolver type="File" key="sp-key.pem" certificate="sp-cert.pem"/>
Z98: ++ <CredentialResolver type="File" key="/etc/ssl/private/sp.shib.lan_key.pem" certificate="/etc/ssl/certs/sp.shib.lan_cert.pem"/>
\end{lstlisting}

* shibd neu starten:
\begin{lstlisting}
	shib-sp@shib-sp:~$ sudo service shibd restart
\end{lstlisting}
* jetzt sollte der Login über Shibboleth funktionieren:
\begin{itemize}
\item https://sp.shib.lan/shibtest/ \footnote{Achtung: shibtest ist nur für SSL mit Shibboleth-Auth, normales http geht auch ohne! (Es ist ja nur ein Test ;-) )}
\item User:Password: alice:alice
\end{itemize}

\subsection{Identity Provider - Teil 2}
\subsubsection{uApprove}
\step{$\rightarrow$ Wechsel in IdP-VM}
\step{uApprove 2.6.0 herunterladen und entpacken}
\begin{lstlisting}
shib-idp@shibidp:~$ mkdir IdP/uApprove
shib-idp@shibidp:~$ cd IdP/uApprove/
shib-idp@shibidp:~/IdP/uApprove$ wget
https://forge.switch.ch/redmine/attachments/download/1823/uApprove-2.6.0.zip shib-idp@shibidp:~/uApprove$ unzip uApprove-2.6.0.zip
shib-idp@shibidp:~/IdP/uApprove$ cd uApprove-2.6.0/
\end{lstlisting}
\step{uApprove Bibliotheken in die IdP Quellen kopieren}
\begin{lstlisting}
shib-idp@shibidp:~/IdP/uApprove/uApprove-2.6.0$ cp lib/*.jar /home/shib-idp/IdP/shibboleth-identityprovider-2.4.4/lib/
shib-idp@shibidp:~/IdP/uApprove/uApprove-2.6.0$ cp lib/jdbc/*.jar /home/shib-idp/IdP/shibboleth-identityprovider-2.4.4/lib/
shib-idp@shibidp:~/IdP/uApprove/uApprove-2.6.0$ cp lib/jdbc/optional/mysql-connector-java-5.1.34.jar /home/shib-idp/IdP/shibboleth-identityprovider-2.4.4/lib/
\end{lstlisting}
\subsubsection*{uApprove Konfiguration, Web-App bereit stellen und SQL-DB
bauen:}
\step{Config-Templates kopieren}
\begin{lstlisting}
shib-idp@shibidp:~/IdP/uApprove/uApprove-2.6.0$ sudo cp manual/configuration/uApprove.properties /opt/shibboleth-idp/conf/
shib-idp@shibidp:~/IdP/uApprove/uApprove-2.6.0$ sudo cp manual/configuration/uApprove.xml /opt/shibboleth-idp/conf/
\end{lstlisting}
\step{Webapp kopieren:}
\begin{lstlisting}
shib-idp@shibidp:~/IdP/uApprove/uApprove-2.6.0$ mkdir /home/shib-idp/IdP/shibboleth-identityprovider-2.4.4/src/main/webapp/uApprove
shib-idp@shibidp:~/IdP/uApprove/uApprove-2.6.0$ cp webapp/* /home/shib-idp/IdP/shibboleth-identityprovider-2.4.4/src/main/webapp/uApprove/
\end{lstlisting}
\step{Datenbank vorbereiten:}
\begin{lstlisting}
shib-idp@shibidp:~/IdP/uApprove/uApprove-2.6.0$ mysql -uroot -p
	# Passwort: idp
\end{lstlisting}
\begin{lstlisting}[language=sql]
mysql> set names 'utf8';
mysql> set character set utf8;
mysql> charset utf8;
mysql> create database if not exists uApprove character set=utf8;
mysql> create user 'uApprove'@'localhost' identified by 'idp';
mysql> grant insert, select, update, delete on uApprove.* to
   'uApprove'@'localhost';
mysql> flush privileges;
mysql> create table ToUAcceptance (userId varchar(104) not null, version varchar(104) not null, fingerprint varchar(256) not null, acceptanceDate timestamp default current_timestamp not null, primary key (userId,version));
mysql> create table AttributeReleaseConsent(userId varchar(104) not null, relyingPartyId varchar(104) not null, attributeId varchar(104) not null, valuesHash varchar(256) not null, consentDate timestamp default current_timestamp not null, primary key (userId, relyingPartyId, attributeId));
mysql> show fields from AttributeReleaseConsent;
mysql> quit
\end{lstlisting}
\step{uApprove konfigurieren:}
* web.xml anpassen (/home/shib-idp/IdP/shibboleth-identityprovider-2.4.4/src/main/webapp/WEB-INF/)
\begin{lstlisting}
Z14: -- <param-value>$IDP_HOME$/conf/internal.xml; $IDP_HOME$/conf/service.xml;</param-value>
Z14: ++ <param-value>$IDP_HOME$/conf/internal.xml; $IDP_HOME$/conf/service.xml; $IDP_HOME$/conf/uApprove.xml;</param-value>

ab Z53ff: hinzufügen:
<!-- uApprove -->
<filter>
	<filter-name>uApprove</filter-name>
	<filter-class>ch.SWITCH.aai.uApprove.Intercepter</filter-class>
</filter>
<filter-mapping>
	<filter-name>uApprove</filter-name>
	<url-pattern>/profile/Shibboleth/SSO</url-pattern>
	<url-pattern>/profile/SAML1/SOAP/AttributeQuery</url-pattern>
	<url-pattern>/profile/SAML1/SOAP/ArtifactResolution</url-pattern>
	<url-pattern>/profile/SAML2/POST/SSO</url-pattern>
	<url-pattern>/profile/SAML2/POST-SimpleSign/SSO</url-pattern>
	<url-pattern>/profile/SAML2/Redirect/SSO</url-pattern>
	<url-pattern>/profile/SAML2/Unsolicited/SSO</url-pattern>
	<url-pattern>/Authn/UserPassword</url-pattern>
</filter-mapping>

<servlet>
	<servlet-name>uApprove - Terms Of Use</servlet-name>
	<servlet-class>ch.SWITCH.aai.uApprove.tou.ToUServlet</servlet-class>
</servlet>

<servlet-mapping>
	<servlet-name>uApprove - Terms Of Use</servlet-name>
	<url-pattern>/uApprove/TermsOfUse</url-pattern>
</servlet-mapping>

<servlet>
	<servlet-name>uApprove - Attribute Release</servlet-name>
	<servlet-class>ch.SWITCH.aai.uApprove.ar.AttributeReleaseServlet</servlet-class>
</servlet>

<servlet-mapping>
	<servlet-name>uApprove - Attribute Release</servlet-name>
	<url-pattern>/uApprove/AttributeRelease</url-pattern>
</servlet-mapping>
\end{lstlisting}
* uApprove.xml anpassen (/opt/shibboleth-idp/conf/)
\begin{lstlisting}
Z10: -- <context:property-placeholder location="classpath:/configuration/uApprove.properties" />
Z10: ++ <context:property-placeholder location="file:/opt/shibboleth-idp/conf/uApprove.properties" />
\end{lstlisting}
* uApprove.properties anpassen (/opt/shibboleth-idp/conf/)
\begin{lstlisting}
Z26: -- view.defaultLocale          = en
Z26: ++ view.defaultLocale          = de

Z38: -- database.password           = secret
Z38: ++ database.password           = idp
\end{lstlisting}
* Checkbox, um Zustimmung zu uApprove am IdP-Portal aufheben zu können
einfügen, dazu login.jsp anpassen (/home/shib-idp/IdP/shibboleth-identityprovider-2.4.4/src/main/webapp)
\begin{lstlisting}
Z50: ++ <section>
Z51: ++	 <input type="checkbox" name="uApprove.consent-revocation" value="true"/>
Z52: ++	 Clear data release consent for this service
Z53: ++	</section>
\end{lstlisting}
\subsubsection*{uApprove deployen}
\step{Deployment geht mit einem neuen Bauen der .war Files einher, der IdP wird
 daher neu installiert, wobei die alte Config beibehalten wird. Es wird nur
 das .war File neu erstellt (und kopiert):}
 \begin{lstlisting}
shib-idp@shibidp:~/IdP/shibboleth-identityprovider-2.4.4$ sudo JAVA_HOME=/usr/lib/jvm/java-7-openjdk-amd64/ ./install.sh
	# Installation nach: /opt/shibboleth-idp/ (default, also Enter)
	# bestehende Config überschreiben: no (!!!)
\end{lstlisting}
\step{uApprove Logger zu logging.xml hinzufügen (/opt/shibboleth-idp/conf):}
\begin{lstlisting}
Z17: ++ <!-- uApprove Logger -->
Z18: ++ <logger name="ch.SWITCH.aai.uApprove" level="DEBUG"/>
\end{lstlisting}
\step{abschließend Tomcat neu starten}
\begin{lstlisting}
shib-idp@shibidp:~/IdP/shibboleth-identityprovider-2.4.4$ sudo service tomcat7 restart
\end{lstlisting}

\subsubsection{attribut-resolver.xml und attribute-filter.xml anpassen}
\step{anpassen der attribute-resolver.xml (/opt/shibboleth-idp/conf) und
folgende Attribute aus der Auskommentierung entfernen:}
\begin{itemize}
\item uid
\item commonName
\item surname
\item givenName
\item eduPersonAffiliation
\item eduPersonEntitlement
\item eduPersonPrimaryAffiliation
\item eduPersonPrincipalName
\item eduPersonScopedAffiliation
\end{itemize}
* neu kommen die dfnEduPerson-Attribute hinzu (hier schon fertig aus- und
einkommentiert), ab Z.261 einzufügen:
\begin{lstlisting}[language=xml]
<!-- DFN Attribute -->
<resolver:AttributeDefinition xsi:type="ad:Simple" id="dfnEduPersonCostCenter" sourceAttributeID="dfnEduPersonCostCenter">
	<resolver:Dependency ref="myLDAP" />
	<resolver:AttributeEncoder xsi:type="enc:SAML1String" name="urn:mace:dir:attribute-def:dfnEduPersonCostCenter" />
	<resolver:AttributeEncoder xsi:type="enc:SAML2String" name="urn:oid:1.3.6.1.4.1.22177.400.1.1.3.1" friendlyName="dfnEduPersonCostCenter" />
</resolver:AttributeDefinition>

<resolver:AttributeDefinition xsi:type="ad:Simple" id="dfnEduPersonStudyBranch1" sourceAttributeID="dfnEduPersonStudyBranch1">
	<resolver:Dependency ref="myLDAP" />
	<resolver:AttributeEncoder xsi:type="enc:SAML1String" name="urn:mace:dir:attribute-def:dfnEduPersonStudyBranch1" />
	<resolver:AttributeEncoder xsi:type="enc:SAML2String" name="urn:oid:1.3.6.1.4.1.22177.400.1.1.3.2" friendlyName="dfnEduPersonStudyBranch1" />
</resolver:AttributeDefinition>

<resolver:AttributeDefinition xsi:type="ad:Simple" id="dfnEduPersonStudyBranch2" sourceAttributeID="dfnEduPersonStudyBranch2">
	<resolver:Dependency ref="myLDAP" />
	<resolver:AttributeEncoder xsi:type="enc:SAML1String" name="urn:mace:dir:attribute-def:dfnEduPersonStudyBranch2" />
	<resolver:AttributeEncoder xsi:type="enc:SAML2String" name="urn:oid:1.3.6.1.4.1.22177.400.1.1.3.3" friendlyName="dfnEduPersonStudyBranch2" />
</resolver:AttributeDefinition>

<resolver:AttributeDefinition xsi:type="ad:Simple" id="dfnEduPersonStudyBranch3" sourceAttributeID="dfnEduPersonStudyBranch3">
	<resolver:Dependency ref="myLDAP" />
	<resolver:AttributeEncoder xsi:type="enc:SAML1String" name="urn:mace:dir:attribute-def:dfnEduPersonStudyBranch2" />
	<resolver:AttributeEncoder xsi:type="enc:SAML2String" name="urn:oid:1.3.6.1.4.1.22177.400.1.1.3.4" friendlyName="dfnEduPersonStudyBranch3" />
</resolver:AttributeDefinition>

<resolver:AttributeDefinition xsi:type="ad:Simple" id="dfnEduPersonFieldOfStudyString" sourceAttributeID="dfnEduPersonFieldOfStudyString">
	<resolver:Dependency ref="myLDAP" />
	<resolver:AttributeEncoder xsi:type="enc:SAML1String" name="urn:mace:dir:attribute-def:dfnEduPersonFieldOfStudyString" />
	<resolver:AttributeEncoder xsi:type="enc:SAML2String" name="urn:oid:1.3.6.1.4.1.22177.400.1.1.3.5" friendlyName="dfnEduPersonFieldOfStudyString" />
</resolver:AttributeDefinition>
<!--
<resolver:AttributeDefinition xsi:type="ad:Simple" id="dfnEduPersonFinalDegree" sourceAttributeID="dfnEduPersonFinalDegree">
	<resolver:Dependency ref="myLDAP" />
	<resolver:AttributeEncoder xsi:type="enc:SAML1String" name="urn:mace:dir:attribute-def:dfnEduPersonFinalDegree" />
	<resolver:AttributeEncoder xsi:type="enc:SAML2String" name="urn:oid:1.3.6.1.4.1.22177.400.1.1.3.6" friendlyName="dfnEduPersonFinalDegree" />
</resolver:AttributeDefinition>

<resolver:AttributeDefinition xsi:type="ad:Simple" id="dfnEduPersonTypeOfStudy" sourceAttributeID="dfnEduPersonTypeOfStudy">
	<resolver:Dependency ref="myLDAP" />
	<resolver:AttributeEncoder xsi:type="enc:SAML1String" name="urn:mace:dir:attribute-def:dfnEduPersonTypeOfStudy" />
	<resolver:AttributeEncoder xsi:type="enc:SAML2String" name="urn:oid:1.3.6.1.4.1.22177.400.1.1.3.7" friendlyName="dfnEduPersonTypeOfStudy" />
</resolver:AttributeDefinition>

<resolver:AttributeDefinition xsi:type="ad:Simple" id="dfnEduPersonTermsOfStudy" sourceAttributeID="dfnEduPersonTermsOfStudy">
	<resolver:Dependency ref="myLDAP" />
	<resolver:AttributeEncoder xsi:type="enc:SAML1String" name="urn:mace:dir:attribute-def:dfnEduPersonTermsOfStudy" />
	<resolver:AttributeEncoder xsi:type="enc:SAML2String" name="urn:oid:1.3.6.1.4.1.22177.400.1.1.3.8" friendlyName="dfnEduPersonTermsOfStudy" />
</resolver:AttributeDefinition>

<resolver:AttributeDefinition xsi:type="ad:Simple" id="dfnEduPersonBranchAndDegree" sourceAttributeID="dfnEduPersonBranchAndDegree">
	<resolver:Dependency ref="myLDAP" />
	<resolver:AttributeEncoder xsi:type="enc:SAML1String" name="urn:mace:dir:attribute-def:dfnEduPersonBranchAndDegree" />
	<resolver:AttributeEncoder xsi:type="enc:SAML2String" name="urn:oid:1.3.6.1.4.1.22177.400.1.1.3.9" friendlyName="dfnEduPersonBranchAndDegree" />
</resolver:AttributeDefinition>

<resolver:AttributeDefinition xsi:type="ad:Simple" id="dfnEduPersonBrachAndType" sourceAttributeID="dfnEduPersonBrachAndType">
	<resolver:Dependency ref="myLDAP" />
	<resolver:AttributeEncoder xsi:type="enc:SAML1String" name="urn:mace:dir:attribute-def:dfnEduPersonBrachAndType" />
	<resolver:AttributeEncoder xsi:type="enc:SAML2String" name="urn:oid:1.3.6.1.4.1.22177.400.1.1.3.10" friendlyName="dfnEduPersonBrachAndType" />
</resolver:AttributeDefinition>

<resolver:AttributeDefinition xsi:type="ad:Simple" id="dfnEduPersonFeaturesOfStudy" sourceAttributeID="dfnEduPersonFeaturesOfStudy">
	<resolver:Dependency ref="myLDAP" />
	<resolver:AttributeEncoder xsi:type="enc:SAML1String" name="urn:mace:dir:attribute-def:dfnEduPersonFeaturesOfStudy" />
	<resolver:AttributeEncoder xsi:type="enc:SAML2String" name="urn:oid:1.3.6.1.4.1.22177.400.1.1.3.11" friendlyName="dfnEduPersonFeaturesOfStudy" />
</resolver:AttributeDefinition>
-->
\end{lstlisting}
* LDAP-Connector einfügen (Anpassung des "`Example LDAP Connector"', einkommentieren nicht vergessen):
\begin{lstlisting}[language=xml]
<resolver:DataConnector id="myLDAP" xsi:type="dc:LDAPDirectory"
	ldapURL="ldap://ldap.shib.lan"
	baseDN="dc=shib,dc=lan"
	principal="uid=shib_ldap,dc=shib,dc=lan"
	principalCredential="shib_ldap">
	<dc:FilterTemplate>
		<![CDATA[
			(uid=$requestContext.principalName)
		]]>
	</dc:FilterTemplate>
</resolver:DataConnector>
\end{lstlisting}
\step{Anpassung attribute-filter.xml  (/opt/shibboleth-idp/conf)}
* folgende Policy ab Zeile 26 einfügen:
\begin{lstlisting}[language=xml]
<afp:AttributeFilterPolicy id="sp.shib.lan-Policy">

	<afp:PolicyRequirementRule xsi:type="basic:AttributeRequesterString" value="https://sp.shib.lan/shibboleth" />

	<afp:AttributeRule attributeID="eduPersonAffiliation">
		<afp:PermitValueRule xsi:type="basic:OR">
			<basic:Rule xsi:type="basic:AttributeValueString" value="faculty" ignoreCase="true" />
			<basic:Rule xsi:type="basic:AttributeValueString" value="student" ignoreCase="true" />
			<basic:Rule xsi:type="basic:AttributeValueString" value="staff" ignoreCase="true" />
			<basic:Rule xsi:type="basic:AttributeValueString" value="alum" ignoreCase="true" />
			<basic:Rule xsi:type="basic:AttributeValueString" value="member" ignoreCase="true" />
			<basic:Rule xsi:type="basic:AttributeValueString" value="affiliate" ignoreCase="true" />
			<basic:Rule xsi:type="basic:AttributeValueString" value="employee" ignoreCase="true" />
			<basic:Rule xsi:type="basic:AttributeValueString" value="library-walk-in" ignoreCase="true" />
		</afp:PermitValueRule>
	</afp:AttributeRule>

	<afp:AttributeRule attributeID="eduPersonPrimaryAffiliation">
		<afp:PermitValueRule xsi:type="basic:OR">
			<basic:Rule xsi:type="basic:AttributeValueString" value="faculty" ignoreCase="true" />
			<basic:Rule xsi:type="basic:AttributeValueString" value="student" ignoreCase="true" />
			<basic:Rule xsi:type="basic:AttributeValueString" value="staff" ignoreCase="true" />
			<basic:Rule xsi:type="basic:AttributeValueString" value="alum" ignoreCase="true" />
			<basic:Rule xsi:type="basic:AttributeValueString" value="member" ignoreCase="true" />
			<basic:Rule xsi:type="basic:AttributeValueString" value="affiliate" ignoreCase="true" />
			<basic:Rule xsi:type="basic:AttributeValueString" value="employee" ignoreCase="true" />
			<basic:Rule xsi:type="basic:AttributeValueString" value="library-walk-in" ignoreCase="true" />
		</afp:PermitValueRule>
	</afp:AttributeRule>

	<afp:AttributeRule attributeID="commonName">
		<afp:PermitValueRule xsi:type="basic:ANY" />
	</afp:AttributeRule>
	<afp:AttributeRule attributeID="surname">
		<afp:PermitValueRule xsi:type="basic:ANY" />
	</afp:AttributeRule>
	<afp:AttributeRule attributeID="givenName">
		<afp:PermitValueRule xsi:type="basic:ANY" />
	</afp:AttributeRule>
	<afp:AttributeRule attributeID="eduPersonPrincipalName">
		<afp:PermitValueRule xsi:type="basic:ANY" />
	</afp:AttributeRule>
	<afp:AttributeRule attributeID="eduPersonScopedAffiliation">
		<afp:PermitValueRule xsi:type="basic:ANY" />
	</afp:AttributeRule>
	<afp:AttributeRule attributeID="eduPersonPrimaryAffiliation">
		<afp:PermitValueRule xsi:type="basic:ANY" />
	</afp:AttributeRule>
	<afp:AttributeRule attributeID="dfnEduPersonStudyBranch1">
		<afp:PermitValueRule xsi:type="basic:ANY" />
	</afp:AttributeRule>
	<afp:AttributeRule attributeID="dfnEduPersonStudyBranch2">
		<afp:PermitValueRule xsi:type="basic:ANY" />
	</afp:AttributeRule>
	<afp:AttributeRule attributeID="dfnEduPersonStudyBranch3">
		<afp:PermitValueRule xsi:type="basic:ANY" />
	</afp:AttributeRule>
	<afp:AttributeRule attributeID="dfnEduPersonFieldOfStudyString">
		<afp:PermitValueRule xsi:type="basic:ANY" />
	</afp:AttributeRule>
	<afp:AttributeRule attributeID="eduPersonEntitlement">
		<afp:PermitValueRule xsi:type="basic:ANY" />
	</afp:AttributeRule>
	<afp:AttributeRule attributeID="dfnEduPersonCostCenter">
		<afp:PermitValueRule xsi:type="basic:ANY" />
	</afp:AttributeRule>

</afp:AttributeFilterPolicy>
\end{lstlisting}
\step{Abschließend Tomcat neu starten:}
\begin{lstlisting}
shib-idp@shibidp:~$ sudo service tomcat7 restart
\end{lstlisting}
\subsubsection*{Attribute-Mapping auf dem SP anpassen:}
$\rightarrow$ Wechsel zur SP-VM
\step{alle Attribute, die vom IdP kommen, müssen im SP noch gemappt werden. Dazu
die attribute-map.xml (/etc/shibboleth) anpassen:}
* einkommentieren von:
\begin{itemize}
\item eduPersonPrimaryAffiliation
\item cn
\item sn
\item givenName
\end{itemize}
\step{Attribute sowohl im alten NameID- als auch OID-Format einkommentieren!}
* zusätzlich ist für die eduPersonScopedAffiliation die oid-Variante einzufügen:
\begin{lstlisting}
<Attribute name="urn:oid:1.3.6.1.4.1.5923.1.1.1.5" id="primary-affiliation">
	<AttributeDecoder xsi:type="StringAttributeDecoder" caseSensitive="false"/>
</Attribute>
\end{lstlisting}
* außerdem neu hinzufügen:
\begin{lstlisting}[language=xml]
<!-- DFN Attribute dfnEduPerson -->
<Attribute name="urn:mace:dir:attribute-def:dfnEduPersonCostCenter" id="dfnEduPersonCostCenter">
	<AttributeDecoder xsi:type="StringAttributeDecoder" caseSensitive="false"/>
</Attribute>
<Attribute name="urn:oid:1.3.6.1.4.1.22177.400.1.1.3.1" id="dfnEduPersonCostCenter">
	<AttributeDecoder xsi:type="StringAttributeDecoder" caseSensitive="false"/>
</Attribute>

<Attribute name="urn:mace:dir:attribute-def:dfnEduPersonStudyBranch1" id="dfnEduPersonStudyBranch1">
	<AttributeDecoder xsi:type="StringAttributeDecoder" caseSensitive="false"/>
</Attribute>
<Attribute name="urn:oid:1.3.6.1.4.1.22177.400.1.1.3.2" id="dfnEduPersonStudyBranch1">
	<AttributeDecoder xsi:type="StringAttributeDecoder" caseSensitive="false"/>
</Attribute>

<Attribute name="urn:mace:dir:attribute-def:dfnEduPersonStudyBranch2" id="dfnEduPersonStudyBranch2">
	<AttributeDecoder xsi:type="StringAttributeDecoder" caseSensitive="false"/>
</Attribute>
<Attribute name="urn:oid:1.3.6.1.4.1.22177.400.1.1.3.3" id="dfnEduPersonStudyBranch2">
	<AttributeDecoder xsi:type="StringAttributeDecoder" caseSensitive="false"/>
</Attribute>

<Attribute name="urn:mace:dir:attribute-def:dfnEduPersonStudyBranch3" id="dfnEduPersonStudyBranch3">
	<AttributeDecoder xsi:type="StringAttributeDecoder" caseSensitive="false"/>
</Attribute>
<Attribute name="urn:oid:1.3.6.1.4.1.22177.400.1.1.3.4" id="dfnEduPersonStudyBranch3">
	<AttributeDecoder xsi:type="StringAttributeDecoder" caseSensitive="false"/>
</Attribute>

<Attribute name="urn:mace:dir:attribute-def:dfnEduPersonFieldOfStudyString" id="dfnEduPersonFieldOfStudyString">
	<AttributeDecoder xsi:type="StringAttributeDecoder" caseSensitive="false"/>
</Attribute>
<Attribute name="urn:oid:1.3.6.1.4.1.22177.400.1.1.3.5" id="dfnEduPersonFieldOfStudyString">
	<AttributeDecoder xsi:type="StringAttributeDecoder" caseSensitive="false"/>
</Attribute>

<!--
<Attribute name="urn:mace:dir:attribute-def:dfnEduPersonFinalDegree" id="dfnEduPersonFinalDegree">
	<AttributeDecoder xsi:type="StringAttributeDecoder" caseSensitive="false"/>
</Attribute>
<Attribute name="urn:oid:1.3.6.1.4.1.22177.400.1.1.3.6" id="dfnEduPersonFinalDegree">
	<AttributeDecoder xsi:type="StringAttributeDecoder" caseSensitive="false"/>
</Attribute>

<Attribute name="urn:mace:dir:attribute-def:dfnEduPersonTypeOfStudy" id="dfnEduPersonTypeOfStudy">
	<AttributeDecoder xsi:type="StringAttributeDecoder" caseSensitive="false"/>
</Attribute>
<Attribute name="urn:oid:1.3.6.1.4.1.22177.400.1.1.3.7" id="dfnEduPersonTypeOfStudy">
	<AttributeDecoder xsi:type="StringAttributeDecoder" caseSensitive="false"/>
</Attribute>

<Attribute name="urn:mace:dir:attribute-def:dfnEduPersonTermsOfStudy" id="dfnEduPersonTermsOfStudy">
	<AttributeDecoder xsi:type="StringAttributeDecoder" caseSensitive="false"/>
</Attribute>
<Attribute name="urn:oid:1.3.6.1.4.1.22177.400.1.1.3.8" id="dfnEduPersonTermsOfStudy">
	<AttributeDecoder xsi:type="StringAttributeDecoder" caseSensitive="false"/>
</Attribute>

<Attribute name="urn:mace:dir:attribute-def:dfnEduPersonBranchAndDegree" id="dfnEduPersonBranchAndDegree">
	<AttributeDecoder xsi:type="StringAttributeDecoder" caseSensitive="false"/>
</Attribute>
<Attribute name="urn:oid:1.3.6.1.4.1.22177.400.1.1.3.9" id="dfnEduPersonBranchAndDegree">
	<AttributeDecoder xsi:type="StringAttributeDecoder" caseSensitive="false"/>
</Attribute>

<Attribute name="urn:mace:dir:attribute-def:dfnEduPersonBrachAndType" id="dfnEduPersonBrachAndType">
	<AttributeDecoder xsi:type="StringAttributeDecoder" caseSensitive="false"/>
</Attribute>
<Attribute name="urn:oid:1.3.6.1.4.1.22177.400.1.1.3.10" id="dfnEduPersonBrachAndType">
	<AttributeDecoder xsi:type="StringAttributeDecoder" caseSensitive="false"/>
</Attribute>

<Attribute name="urn:mace:dir:attribute-def:dfnEduPersonFeaturesOfStudy" id="dfnEduPersonFeaturesOfStudy">
	<AttributeDecoder xsi:type="StringAttributeDecoder" caseSensitive="false"/>
</Attribute>
<Attribute name="urn:oid:1.3.6.1.4.1.22177.400.1.1.3.11" id="dfnEduPersonFeaturesOfStudy">
	<AttributeDecoder xsi:type="StringAttributeDecoder" caseSensitive="false"/>
</Attribute>
-->
\end{lstlisting}

\step{Auf den shibtest im SP ( https://sp.shib.lan/shibtest ) kann jetzt per IdP
Authentifizierung und uApprove zugegriffen werden}

\step{Optional: Anpassung der shibtest-Seite auf dem SP für die Ausgabe der
übertragenen Shibboleth-Attribute:}
$\rightarrow$ Wechsel in die SP-VM:
\step{sp.conf anpassen (/etc/apache2/sites-available)}
\begin{lstlisting}
Z19: ++    DirectoryIndex shib-info.php
\end{lstlisting}
\step{shib-info.php in /var/www/html/shibtest anlegen:}
\begin{lstlisting}[language=html]
shib-sp@shib-sp:/var/www/html/shibtest$ cat shib-info.php
	<html>
	<head>
	<title>SHIB-INFO</title>
	</head>
	<body>
	<h1>SHIBTEST</h1>
	<h2>$_SERVER:</h2>
	<?php
		foreach($_SERVER as $key => $value)
		{
			echo "<pre>".$key.": ".$value."</pre>";
		}
	?>
	</body>
\end{lstlisting}

\subsection{optionale Einstellungen}
\subsubsection{Statusseiten für IdP und SP verfügbar machen}
ACHTUNG: Das Verfügbarmachen der Statusseiten von "`außerhalb"' ist für den
Produktivbetrieb natürlich nicht sinnvoll, da hier z.T. interne Informationen
angezeigt werden!
\subsubsection*{IdP}
\step{Anpassung der web.xml (/home/shib-idp/IdP/shibboleth-identityprovider-2.4.4/src/main/webapp/WEB-INF)}:
\begin{lstlisting}
<!-- Servlet for displaying IdP status. -->
 <servlet>
   <servlet-name>Status</servlet-name>
    <servlet-class>edu.internet2.middleware.shibboleth.idp.StatusServlet</servlet-class>
     <!-- Space separated list of CIDR blocks allowed to access the status page -->
      <init-param>
        <param-name>AllowedIPs</param-name>
--      <param-value>127.0.0.1/32 ::1/128</param-value>
++      <param-value>127.0.0.1/32 ::1/128 192.168.100.0/24</param-value>
      </init-param>
    <load-on-startup>2</load-on-startup>
  </servlet>
\end{lstlisting}
\step{IdP neu deployen und Tomcat neu starten:}
\begin{lstlisting}
shib-idp@shibidp:~/IdP/shibboleth-identityprovider-2.4.4$ sudo JAVA\_HOME=/usr/lib/jvm/java-7-openjdk-amd64/ ./install.sh
	# bestehende config NICHT überschreiben!
shib-idp@shibidp:~$ sudo service tomcat7 restart
\end{lstlisting}
\step{IdP-Status-Seite: https://idp.shib.lan/idp/status}
\subsubsection{SP}
\step{shibboleth2.xml (/etc/shibboleth) anpassen:}
\begin{lstlisting}
-- <Handler type="Status" Location="/Status" acl="127.0.0.1 ::1"/>
++ <Handler type="Status" Location="/Status" acl="127.0.0.1 ::1  192.168.100.0/24"/>
\end{lstlisting}
\step{shibd neu starten:}
\begin{lstlisting}
shib-sp@shib-sp:~$ sudo service shibd restart
\end{lstlisting}
\step{SP-Status-Seite: https://sp.shib.lan/Shibboleth.sso/Status}
